% Options for packages loaded elsewhere
\PassOptionsToPackage{unicode}{hyperref}
\PassOptionsToPackage{hyphens}{url}
%
\documentclass[
]{article}
\usepackage{amsmath,amssymb}
\usepackage{iftex}
\ifPDFTeX
  \usepackage[T1]{fontenc}
  \usepackage[utf8]{inputenc}
  \usepackage{textcomp} % provide euro and other symbols
\else % if luatex or xetex
  \usepackage{unicode-math} % this also loads fontspec
  \defaultfontfeatures{Scale=MatchLowercase}
  \defaultfontfeatures[\rmfamily]{Ligatures=TeX,Scale=1}
\fi
\usepackage{lmodern}
\ifPDFTeX\else
  % xetex/luatex font selection
\fi
% Use upquote if available, for straight quotes in verbatim environments
\IfFileExists{upquote.sty}{\usepackage{upquote}}{}
\IfFileExists{microtype.sty}{% use microtype if available
  \usepackage[]{microtype}
  \UseMicrotypeSet[protrusion]{basicmath} % disable protrusion for tt fonts
}{}
\makeatletter
\@ifundefined{KOMAClassName}{% if non-KOMA class
  \IfFileExists{parskip.sty}{%
    \usepackage{parskip}
  }{% else
    \setlength{\parindent}{0pt}
    \setlength{\parskip}{6pt plus 2pt minus 1pt}}
}{% if KOMA class
  \KOMAoptions{parskip=half}}
\makeatother
\usepackage{xcolor}
\usepackage[margin=1in]{geometry}
\usepackage{color}
\usepackage{fancyvrb}
\newcommand{\VerbBar}{|}
\newcommand{\VERB}{\Verb[commandchars=\\\{\}]}
\DefineVerbatimEnvironment{Highlighting}{Verbatim}{commandchars=\\\{\}}
% Add ',fontsize=\small' for more characters per line
\usepackage{framed}
\definecolor{shadecolor}{RGB}{248,248,248}
\newenvironment{Shaded}{\begin{snugshade}}{\end{snugshade}}
\newcommand{\AlertTok}[1]{\textcolor[rgb]{0.94,0.16,0.16}{#1}}
\newcommand{\AnnotationTok}[1]{\textcolor[rgb]{0.56,0.35,0.01}{\textbf{\textit{#1}}}}
\newcommand{\AttributeTok}[1]{\textcolor[rgb]{0.13,0.29,0.53}{#1}}
\newcommand{\BaseNTok}[1]{\textcolor[rgb]{0.00,0.00,0.81}{#1}}
\newcommand{\BuiltInTok}[1]{#1}
\newcommand{\CharTok}[1]{\textcolor[rgb]{0.31,0.60,0.02}{#1}}
\newcommand{\CommentTok}[1]{\textcolor[rgb]{0.56,0.35,0.01}{\textit{#1}}}
\newcommand{\CommentVarTok}[1]{\textcolor[rgb]{0.56,0.35,0.01}{\textbf{\textit{#1}}}}
\newcommand{\ConstantTok}[1]{\textcolor[rgb]{0.56,0.35,0.01}{#1}}
\newcommand{\ControlFlowTok}[1]{\textcolor[rgb]{0.13,0.29,0.53}{\textbf{#1}}}
\newcommand{\DataTypeTok}[1]{\textcolor[rgb]{0.13,0.29,0.53}{#1}}
\newcommand{\DecValTok}[1]{\textcolor[rgb]{0.00,0.00,0.81}{#1}}
\newcommand{\DocumentationTok}[1]{\textcolor[rgb]{0.56,0.35,0.01}{\textbf{\textit{#1}}}}
\newcommand{\ErrorTok}[1]{\textcolor[rgb]{0.64,0.00,0.00}{\textbf{#1}}}
\newcommand{\ExtensionTok}[1]{#1}
\newcommand{\FloatTok}[1]{\textcolor[rgb]{0.00,0.00,0.81}{#1}}
\newcommand{\FunctionTok}[1]{\textcolor[rgb]{0.13,0.29,0.53}{\textbf{#1}}}
\newcommand{\ImportTok}[1]{#1}
\newcommand{\InformationTok}[1]{\textcolor[rgb]{0.56,0.35,0.01}{\textbf{\textit{#1}}}}
\newcommand{\KeywordTok}[1]{\textcolor[rgb]{0.13,0.29,0.53}{\textbf{#1}}}
\newcommand{\NormalTok}[1]{#1}
\newcommand{\OperatorTok}[1]{\textcolor[rgb]{0.81,0.36,0.00}{\textbf{#1}}}
\newcommand{\OtherTok}[1]{\textcolor[rgb]{0.56,0.35,0.01}{#1}}
\newcommand{\PreprocessorTok}[1]{\textcolor[rgb]{0.56,0.35,0.01}{\textit{#1}}}
\newcommand{\RegionMarkerTok}[1]{#1}
\newcommand{\SpecialCharTok}[1]{\textcolor[rgb]{0.81,0.36,0.00}{\textbf{#1}}}
\newcommand{\SpecialStringTok}[1]{\textcolor[rgb]{0.31,0.60,0.02}{#1}}
\newcommand{\StringTok}[1]{\textcolor[rgb]{0.31,0.60,0.02}{#1}}
\newcommand{\VariableTok}[1]{\textcolor[rgb]{0.00,0.00,0.00}{#1}}
\newcommand{\VerbatimStringTok}[1]{\textcolor[rgb]{0.31,0.60,0.02}{#1}}
\newcommand{\WarningTok}[1]{\textcolor[rgb]{0.56,0.35,0.01}{\textbf{\textit{#1}}}}
\usepackage{graphicx}
\makeatletter
\def\maxwidth{\ifdim\Gin@nat@width>\linewidth\linewidth\else\Gin@nat@width\fi}
\def\maxheight{\ifdim\Gin@nat@height>\textheight\textheight\else\Gin@nat@height\fi}
\makeatother
% Scale images if necessary, so that they will not overflow the page
% margins by default, and it is still possible to overwrite the defaults
% using explicit options in \includegraphics[width, height, ...]{}
\setkeys{Gin}{width=\maxwidth,height=\maxheight,keepaspectratio}
% Set default figure placement to htbp
\makeatletter
\def\fps@figure{htbp}
\makeatother
\setlength{\emergencystretch}{3em} % prevent overfull lines
\providecommand{\tightlist}{%
  \setlength{\itemsep}{0pt}\setlength{\parskip}{0pt}}
\setcounter{secnumdepth}{-\maxdimen} % remove section numbering
\ifLuaTeX
  \usepackage{selnolig}  % disable illegal ligatures
\fi
\IfFileExists{bookmark.sty}{\usepackage{bookmark}}{\usepackage{hyperref}}
\IfFileExists{xurl.sty}{\usepackage{xurl}}{} % add URL line breaks if available
\urlstyle{same}
\hypersetup{
  pdftitle={Bellabeat Data Analytics},
  pdfauthor={Mayssem Hannachi 2BD1},
  hidelinks,
  pdfcreator={LaTeX via pandoc}}

\title{Bellabeat Data Analytics}
\author{Mayssem Hannachi 2BD1}
\date{2024-05-07}

\begin{document}
\maketitle

\hypertarget{section-1-overview}{%
\subsection{Section 1: Overview}\label{section-1-overview}}

\hypertarget{what-is-bellbeat}{%
\subsubsection{What is Bellbeat?}\label{what-is-bellbeat}}

Bellabeat is a company that makes high-tech health products for women.
By collecting data on activity, sleep, stress, and reproductive health,
Bellabeat helps women understand their health and habits better.

Bellabeat plans to analyze data from smart device usage to see how
people use non-Bellabeat devices. The findings will help shape
Bellabeat's marketing strategies and spark new ideas for their products
and services.

This case study will concentrate on enhancing Ivy, Bellabeat's wellness
and health tracking device.

\textbf{Ivy -} A health tracker in the form of an elegant bracelet,
monitoring not only activity, sleep, and stress but also heart rate,
respiratory rate, cardiac coherence, and physical and mental activity.

\hypertarget{questions-to-consider}{%
\paragraph{Questions to Consider}\label{questions-to-consider}}

This case study will address three key questions:

\begin{enumerate}
\def\labelenumi{\arabic{enumi}.}
\item
  What are the popular trends in smart device usage?
\item
  How do these trends connect with Bellabeat's target audience?
\item
  How can these trends shape Bellabeat's marketing approach?
\end{enumerate}

\hypertarget{section-2-data-source}{%
\subsection{Section 2: Data Source}\label{section-2-data-source}}

In this case study, we will use the FitBit Fitness Tracker Data
available on Kaggle. This dataset was gathered from a survey on Amazon
Mechanical Turk from March 12, 2016, to May 12, 2016. It includes
detailed tracking data from 30 Fitbit users who agreed to share their
minute-by-minute physical activity, heart rate, and sleep data. The data
reveals patterns in daily activity, steps, and heart rate, highlighting
user habits. The dataset consists of 18 CSV files, formatted in both
long and wide styles.

The dataset contains data from only 30 participants, which may not fully
represent the more than 31 million Fitbit users as of 2020. Although a
sample size of 30 meets the minimum requirement under the Central Limit
Theorem (CLT) to be considered valid, a larger sample would be
preferable to enhance the reliability of the research findings.

\hypertarget{section-3-process}{%
\subsection{Section 3: Process}\label{section-3-process}}

In this section, we will clean the selected data to make sure it is
complete, accurate, and free from errors and outliers.

\hypertarget{installing-and-loading-packages}{%
\subsubsection{3.1 Installing and Loading
Packages}\label{installing-and-loading-packages}}

First, we will install and then load the necessary packages to utilize
their features for the analysis.

\begin{Shaded}
\begin{Highlighting}[]
\CommentTok{\# Set a CRAN mirror}
\FunctionTok{options}\NormalTok{(}\AttributeTok{repos =} \FunctionTok{list}\NormalTok{(}\AttributeTok{CRAN =} \StringTok{"https://cloud.r{-}project.org/"}\NormalTok{))}

\CommentTok{\# List of required packages}
\NormalTok{packages }\OtherTok{\textless{}{-}} \FunctionTok{c}\NormalTok{(}\StringTok{"tidyverse"}\NormalTok{, }\StringTok{"dplyr"}\NormalTok{, }\StringTok{"readr"}\NormalTok{, }\StringTok{"janitor"}\NormalTok{, }\StringTok{"ggplot2"}\NormalTok{, }\StringTok{"lubridate"}\NormalTok{, }\StringTok{"skimr"}\NormalTok{, }\StringTok{"gridExtra"}\NormalTok{)}

\CommentTok{\# Function to check and install missing packages}
\NormalTok{check\_and\_install }\OtherTok{\textless{}{-}} \ControlFlowTok{function}\NormalTok{(pkg)\{}
  \ControlFlowTok{if}\NormalTok{ (}\SpecialCharTok{!}\FunctionTok{require}\NormalTok{(pkg, }\AttributeTok{character.only =} \ConstantTok{TRUE}\NormalTok{)) \{}
    \FunctionTok{install.packages}\NormalTok{(pkg, }\AttributeTok{dependencies =} \ConstantTok{TRUE}\NormalTok{)}
    \FunctionTok{library}\NormalTok{(pkg, }\AttributeTok{character.only =} \ConstantTok{TRUE}\NormalTok{)}
\NormalTok{  \}}
\NormalTok{\}}

\CommentTok{\# Apply the function to each package}
\FunctionTok{sapply}\NormalTok{(packages, check\_and\_install)}
\end{Highlighting}
\end{Shaded}

\begin{verbatim}
## Loading required package: tidyverse
\end{verbatim}

\begin{verbatim}
## -- Attaching core tidyverse packages ------------------------ tidyverse 2.0.0 --
## v dplyr     1.1.4     v readr     2.1.5
## v forcats   1.0.0     v stringr   1.5.1
## v ggplot2   3.5.1     v tibble    3.2.1
## v lubridate 1.9.3     v tidyr     1.3.1
## v purrr     1.0.2     
## -- Conflicts ------------------------------------------ tidyverse_conflicts() --
## x dplyr::filter() masks stats::filter()
## x dplyr::lag()    masks stats::lag()
## i Use the conflicted package (<http://conflicted.r-lib.org/>) to force all conflicts to become errors
## Loading required package: janitor
## 
## 
## Attaching package: 'janitor'
## 
## 
## The following objects are masked from 'package:stats':
## 
##     chisq.test, fisher.test
## 
## 
## Loading required package: skimr
## 
## Loading required package: gridExtra
## 
## 
## Attaching package: 'gridExtra'
## 
## 
## The following object is masked from 'package:dplyr':
## 
##     combine
\end{verbatim}

\begin{verbatim}
## $tidyverse
## NULL
## 
## $dplyr
## NULL
## 
## $readr
## NULL
## 
## $janitor
## NULL
## 
## $ggplot2
## NULL
## 
## $lubridate
## NULL
## 
## $skimr
## NULL
## 
## $gridExtra
## NULL
\end{verbatim}

Check if all our packages have successfully been installed

\begin{Shaded}
\begin{Highlighting}[]
\CommentTok{\# List of packages you need}
\NormalTok{required\_packages }\OtherTok{\textless{}{-}} \FunctionTok{c}\NormalTok{(}\StringTok{"tidyverse"}\NormalTok{, }\StringTok{"dplyr"}\NormalTok{, }\StringTok{"readr"}\NormalTok{, }\StringTok{"janitor"}\NormalTok{, }\StringTok{"ggplot2"}\NormalTok{, }\StringTok{"lubridate"}\NormalTok{, }\StringTok{"skimr"}\NormalTok{, }\StringTok{"gridExtra"}\NormalTok{)}

\CommentTok{\# Check each package and print whether it is loaded successfully}
\FunctionTok{sapply}\NormalTok{(required\_packages, }\ControlFlowTok{function}\NormalTok{(pkg) \{}
  \ControlFlowTok{if}\NormalTok{ (}\SpecialCharTok{!}\FunctionTok{require}\NormalTok{(pkg, }\AttributeTok{character.only =} \ConstantTok{TRUE}\NormalTok{)) \{}
    \FunctionTok{cat}\NormalTok{(}\StringTok{"Package not installed:"}\NormalTok{, pkg, }\StringTok{"}\SpecialCharTok{\textbackslash{}n}\StringTok{"}\NormalTok{)}
\NormalTok{  \} }\ControlFlowTok{else}\NormalTok{ \{}
    \FunctionTok{cat}\NormalTok{(}\StringTok{"Package loaded successfully:"}\NormalTok{, pkg, }\StringTok{"}\SpecialCharTok{\textbackslash{}n}\StringTok{"}\NormalTok{)}
\NormalTok{  \}}
\NormalTok{\})}
\end{Highlighting}
\end{Shaded}

\begin{verbatim}
## Package loaded successfully: tidyverse 
## Package loaded successfully: dplyr 
## Package loaded successfully: readr 
## Package loaded successfully: janitor 
## Package loaded successfully: ggplot2 
## Package loaded successfully: lubridate 
## Package loaded successfully: skimr 
## Package loaded successfully: gridExtra
\end{verbatim}

\begin{verbatim}
## $tidyverse
## NULL
## 
## $dplyr
## NULL
## 
## $readr
## NULL
## 
## $janitor
## NULL
## 
## $ggplot2
## NULL
## 
## $lubridate
## NULL
## 
## $skimr
## NULL
## 
## $gridExtra
## NULL
\end{verbatim}

----\textgreater{} It appears that all the required packages are
successfully loaded into our R environment.

\hypertarget{importing-datasets-and-assigning-new-names}{%
\subsubsection{3.2 Importing Datasets and Assigning New
Names}\label{importing-datasets-and-assigning-new-names}}

\begin{Shaded}
\begin{Highlighting}[]
\NormalTok{daily\_activity }\OtherTok{\textless{}{-}} \FunctionTok{read\_csv}\NormalTok{(}\StringTok{"/Users/macbookair/Documents/ISAMM/sem2/R/Bellabeat Case Study Project/Fitabase Data 4.12.16{-}5.12.16/dailyActivity\_merged.csv"}\NormalTok{)}
\end{Highlighting}
\end{Shaded}

\begin{verbatim}
## Rows: 940 Columns: 15
## -- Column specification --------------------------------------------------------
## Delimiter: ","
## chr  (1): ActivityDate
## dbl (14): Id, TotalSteps, TotalDistance, TrackerDistance, LoggedActivitiesDi...
## 
## i Use `spec()` to retrieve the full column specification for this data.
## i Specify the column types or set `show_col_types = FALSE` to quiet this message.
\end{verbatim}

\begin{Shaded}
\begin{Highlighting}[]
\NormalTok{daily\_calories }\OtherTok{\textless{}{-}} \FunctionTok{read\_csv}\NormalTok{(}\StringTok{"/Users/macbookair/Documents/ISAMM/sem2/R/Bellabeat Case Study Project/Fitabase Data 4.12.16{-}5.12.16/dailyCalories\_merged.csv"}\NormalTok{)}
\end{Highlighting}
\end{Shaded}

\begin{verbatim}
## Rows: 940 Columns: 3
## -- Column specification --------------------------------------------------------
## Delimiter: ","
## chr (1): ActivityDay
## dbl (2): Id, Calories
## 
## i Use `spec()` to retrieve the full column specification for this data.
## i Specify the column types or set `show_col_types = FALSE` to quiet this message.
\end{verbatim}

\begin{Shaded}
\begin{Highlighting}[]
\NormalTok{daily\_intensities }\OtherTok{\textless{}{-}} \FunctionTok{read\_csv}\NormalTok{(}\StringTok{"/Users/macbookair/Documents/ISAMM/sem2/R/Bellabeat Case Study Project/Fitabase Data 4.12.16{-}5.12.16/dailyIntensities\_merged.csv"}\NormalTok{)}
\end{Highlighting}
\end{Shaded}

\begin{verbatim}
## Rows: 940 Columns: 10
## -- Column specification --------------------------------------------------------
## Delimiter: ","
## chr (1): ActivityDay
## dbl (9): Id, SedentaryMinutes, LightlyActiveMinutes, FairlyActiveMinutes, Ve...
## 
## i Use `spec()` to retrieve the full column specification for this data.
## i Specify the column types or set `show_col_types = FALSE` to quiet this message.
\end{verbatim}

\begin{Shaded}
\begin{Highlighting}[]
\NormalTok{daily\_steps }\OtherTok{\textless{}{-}} \FunctionTok{read\_csv}\NormalTok{(}\StringTok{"/Users/macbookair/Documents/ISAMM/sem2/R/Bellabeat Case Study Project/Fitabase Data 4.12.16{-}5.12.16/dailySteps\_merged.csv"}\NormalTok{)}
\end{Highlighting}
\end{Shaded}

\begin{verbatim}
## Rows: 940 Columns: 3
## -- Column specification --------------------------------------------------------
## Delimiter: ","
## chr (1): ActivityDay
## dbl (2): Id, StepTotal
## 
## i Use `spec()` to retrieve the full column specification for this data.
## i Specify the column types or set `show_col_types = FALSE` to quiet this message.
\end{verbatim}

\begin{Shaded}
\begin{Highlighting}[]
\NormalTok{daily\_sleep }\OtherTok{\textless{}{-}} \FunctionTok{read\_csv}\NormalTok{(}\StringTok{"/Users/macbookair/Documents/ISAMM/sem2/R/Bellabeat Case Study Project/Fitabase Data 4.12.16{-}5.12.16/sleepDay\_merged.csv"}\NormalTok{)}
\end{Highlighting}
\end{Shaded}

\begin{verbatim}
## Rows: 413 Columns: 5
## -- Column specification --------------------------------------------------------
## Delimiter: ","
## chr (1): SleepDay
## dbl (4): Id, TotalSleepRecords, TotalMinutesAsleep, TotalTimeInBed
## 
## i Use `spec()` to retrieve the full column specification for this data.
## i Specify the column types or set `show_col_types = FALSE` to quiet this message.
\end{verbatim}

\begin{Shaded}
\begin{Highlighting}[]
\NormalTok{heartrate\_seconds }\OtherTok{\textless{}{-}} \FunctionTok{read\_csv}\NormalTok{(}\StringTok{"/Users/macbookair/Documents/ISAMM/sem2/R/Bellabeat Case Study Project/Fitabase Data 4.12.16{-}5.12.16/heartrate\_seconds\_merged.csv"}\NormalTok{)}
\end{Highlighting}
\end{Shaded}

\begin{verbatim}
## Rows: 2483658 Columns: 3
## -- Column specification --------------------------------------------------------
## Delimiter: ","
## chr (1): Time
## dbl (2): Id, Value
## 
## i Use `spec()` to retrieve the full column specification for this data.
## i Specify the column types or set `show_col_types = FALSE` to quiet this message.
\end{verbatim}

\begin{Shaded}
\begin{Highlighting}[]
\NormalTok{hourly\_calories }\OtherTok{\textless{}{-}} \FunctionTok{read\_csv}\NormalTok{(}\StringTok{"/Users/macbookair/Documents/ISAMM/sem2/R/Bellabeat Case Study Project/Fitabase Data 4.12.16{-}5.12.16/hourlyCalories\_merged.csv"}\NormalTok{)}
\end{Highlighting}
\end{Shaded}

\begin{verbatim}
## Rows: 22099 Columns: 3
## -- Column specification --------------------------------------------------------
## Delimiter: ","
## chr (1): ActivityHour
## dbl (2): Id, Calories
## 
## i Use `spec()` to retrieve the full column specification for this data.
## i Specify the column types or set `show_col_types = FALSE` to quiet this message.
\end{verbatim}

\begin{Shaded}
\begin{Highlighting}[]
\NormalTok{hourly\_intensities }\OtherTok{\textless{}{-}} \FunctionTok{read\_csv}\NormalTok{(}\StringTok{"/Users/macbookair/Documents/ISAMM/sem2/R/Bellabeat Case Study Project/Fitabase Data 4.12.16{-}5.12.16/hourlyIntensities\_merged.csv"}\NormalTok{)}
\end{Highlighting}
\end{Shaded}

\begin{verbatim}
## Rows: 22099 Columns: 4
## -- Column specification --------------------------------------------------------
## Delimiter: ","
## chr (1): ActivityHour
## dbl (3): Id, TotalIntensity, AverageIntensity
## 
## i Use `spec()` to retrieve the full column specification for this data.
## i Specify the column types or set `show_col_types = FALSE` to quiet this message.
\end{verbatim}

\begin{Shaded}
\begin{Highlighting}[]
\NormalTok{hourly\_steps }\OtherTok{\textless{}{-}} \FunctionTok{read\_csv}\NormalTok{(}\StringTok{"/Users/macbookair/Documents/ISAMM/sem2/R/Bellabeat Case Study Project/Fitabase Data 4.12.16{-}5.12.16/hourlySteps\_merged.csv"}\NormalTok{)}
\end{Highlighting}
\end{Shaded}

\begin{verbatim}
## Rows: 22099 Columns: 3
## -- Column specification --------------------------------------------------------
## Delimiter: ","
## chr (1): ActivityHour
## dbl (2): Id, StepTotal
## 
## i Use `spec()` to retrieve the full column specification for this data.
## i Specify the column types or set `show_col_types = FALSE` to quiet this message.
\end{verbatim}

\begin{Shaded}
\begin{Highlighting}[]
\NormalTok{minute\_calories\_narrow }\OtherTok{\textless{}{-}} \FunctionTok{read\_csv}\NormalTok{(}\StringTok{"/Users/macbookair/Documents/ISAMM/sem2/R/Bellabeat Case Study Project/Fitabase Data 4.12.16{-}5.12.16/minuteCaloriesNarrow\_merged.csv"}\NormalTok{)}
\end{Highlighting}
\end{Shaded}

\begin{verbatim}
## Rows: 1325580 Columns: 3
## -- Column specification --------------------------------------------------------
## Delimiter: ","
## chr (1): ActivityMinute
## dbl (2): Id, Calories
## 
## i Use `spec()` to retrieve the full column specification for this data.
## i Specify the column types or set `show_col_types = FALSE` to quiet this message.
\end{verbatim}

\begin{Shaded}
\begin{Highlighting}[]
\NormalTok{minute\_calories\_wide }\OtherTok{\textless{}{-}} \FunctionTok{read\_csv}\NormalTok{(}\StringTok{"/Users/macbookair/Documents/ISAMM/sem2/R/Bellabeat Case Study Project/Fitabase Data 4.12.16{-}5.12.16/minuteCaloriesWide\_merged.csv"}\NormalTok{)}
\end{Highlighting}
\end{Shaded}

\begin{verbatim}
## Rows: 21645 Columns: 62
## -- Column specification --------------------------------------------------------
## Delimiter: ","
## chr  (1): ActivityHour
## dbl (61): Id, Calories00, Calories01, Calories02, Calories03, Calories04, Ca...
## 
## i Use `spec()` to retrieve the full column specification for this data.
## i Specify the column types or set `show_col_types = FALSE` to quiet this message.
\end{verbatim}

\begin{Shaded}
\begin{Highlighting}[]
\NormalTok{minute\_intensities\_narrow }\OtherTok{\textless{}{-}} \FunctionTok{read\_csv}\NormalTok{(}\StringTok{"/Users/macbookair/Documents/ISAMM/sem2/R/Bellabeat Case Study Project/Fitabase Data 4.12.16{-}5.12.16/minuteIntensitiesNarrow\_merged.csv"}\NormalTok{)}
\end{Highlighting}
\end{Shaded}

\begin{verbatim}
## Rows: 1325580 Columns: 3
## -- Column specification --------------------------------------------------------
## Delimiter: ","
## chr (1): ActivityMinute
## dbl (2): Id, Intensity
## 
## i Use `spec()` to retrieve the full column specification for this data.
## i Specify the column types or set `show_col_types = FALSE` to quiet this message.
\end{verbatim}

\begin{Shaded}
\begin{Highlighting}[]
\NormalTok{minute\_intensities\_wide }\OtherTok{\textless{}{-}} \FunctionTok{read\_csv}\NormalTok{(}\StringTok{"/Users/macbookair/Documents/ISAMM/sem2/R/Bellabeat Case Study Project/Fitabase Data 4.12.16{-}5.12.16/minuteIntensitiesWide\_merged.csv"}\NormalTok{)}
\end{Highlighting}
\end{Shaded}

\begin{verbatim}
## Rows: 21645 Columns: 62
## -- Column specification --------------------------------------------------------
## Delimiter: ","
## chr  (1): ActivityHour
## dbl (61): Id, Intensity00, Intensity01, Intensity02, Intensity03, Intensity0...
## 
## i Use `spec()` to retrieve the full column specification for this data.
## i Specify the column types or set `show_col_types = FALSE` to quiet this message.
\end{verbatim}

\begin{Shaded}
\begin{Highlighting}[]
\NormalTok{minute\_METs\_narrow }\OtherTok{\textless{}{-}} \FunctionTok{read\_csv}\NormalTok{(}\StringTok{"/Users/macbookair/Documents/ISAMM/sem2/R/Bellabeat Case Study Project/Fitabase Data 4.12.16{-}5.12.16/minuteMETsNarrow\_merged.csv"}\NormalTok{)}
\end{Highlighting}
\end{Shaded}

\begin{verbatim}
## Rows: 1325580 Columns: 3
## -- Column specification --------------------------------------------------------
## Delimiter: ","
## chr (1): ActivityMinute
## dbl (2): Id, METs
## 
## i Use `spec()` to retrieve the full column specification for this data.
## i Specify the column types or set `show_col_types = FALSE` to quiet this message.
\end{verbatim}

\begin{Shaded}
\begin{Highlighting}[]
\NormalTok{minute\_sleep }\OtherTok{\textless{}{-}} \FunctionTok{read\_csv}\NormalTok{(}\StringTok{"/Users/macbookair/Documents/ISAMM/sem2/R/Bellabeat Case Study Project/Fitabase Data 4.12.16{-}5.12.16/minuteSleep\_merged.csv"}\NormalTok{)}
\end{Highlighting}
\end{Shaded}

\begin{verbatim}
## Rows: 188521 Columns: 4
## -- Column specification --------------------------------------------------------
## Delimiter: ","
## chr (1): date
## dbl (3): Id, value, logId
## 
## i Use `spec()` to retrieve the full column specification for this data.
## i Specify the column types or set `show_col_types = FALSE` to quiet this message.
\end{verbatim}

\begin{Shaded}
\begin{Highlighting}[]
\NormalTok{minute\_steps\_narrow }\OtherTok{\textless{}{-}} \FunctionTok{read\_csv}\NormalTok{(}\StringTok{"/Users/macbookair/Documents/ISAMM/sem2/R/Bellabeat Case Study Project/Fitabase Data 4.12.16{-}5.12.16/minuteStepsNarrow\_merged.csv"}\NormalTok{)}
\end{Highlighting}
\end{Shaded}

\begin{verbatim}
## Rows: 1325580 Columns: 3
## -- Column specification --------------------------------------------------------
## Delimiter: ","
## chr (1): ActivityMinute
## dbl (2): Id, Steps
## 
## i Use `spec()` to retrieve the full column specification for this data.
## i Specify the column types or set `show_col_types = FALSE` to quiet this message.
\end{verbatim}

\begin{Shaded}
\begin{Highlighting}[]
\NormalTok{minute\_steps\_wide }\OtherTok{\textless{}{-}} \FunctionTok{read\_csv}\NormalTok{(}\StringTok{"/Users/macbookair/Documents/ISAMM/sem2/R/Bellabeat Case Study Project/Fitabase Data 4.12.16{-}5.12.16/minuteStepsWide\_merged.csv"}\NormalTok{)}
\end{Highlighting}
\end{Shaded}

\begin{verbatim}
## Rows: 21645 Columns: 62
## -- Column specification --------------------------------------------------------
## Delimiter: ","
## chr  (1): ActivityHour
## dbl (61): Id, Steps00, Steps01, Steps02, Steps03, Steps04, Steps05, Steps06,...
## 
## i Use `spec()` to retrieve the full column specification for this data.
## i Specify the column types or set `show_col_types = FALSE` to quiet this message.
\end{verbatim}

\begin{Shaded}
\begin{Highlighting}[]
\NormalTok{weight\_log\_info }\OtherTok{\textless{}{-}} \FunctionTok{read\_csv}\NormalTok{(}\StringTok{"/Users/macbookair/Documents/ISAMM/sem2/R/Bellabeat Case Study Project/Fitabase Data 4.12.16{-}5.12.16/weightLogInfo\_merged.csv"}\NormalTok{)}
\end{Highlighting}
\end{Shaded}

\begin{verbatim}
## Rows: 67 Columns: 8
## -- Column specification --------------------------------------------------------
## Delimiter: ","
## chr (1): Date
## dbl (6): Id, WeightKg, WeightPounds, Fat, BMI, LogId
## lgl (1): IsManualReport
## 
## i Use `spec()` to retrieve the full column specification for this data.
## i Specify the column types or set `show_col_types = FALSE` to quiet this message.
\end{verbatim}

\hypertarget{exploring-the-datasets}{%
\subsubsection{3.3 Exploring the
Datasets}\label{exploring-the-datasets}}

To get a quick idea of what is in the datasets, the glimpse() function
will be used. The datasets will be divided into daily, hourly, minute,
and second dataframes.

\hypertarget{daily-dataframes}{%
\paragraph{Daily DataFrames}\label{daily-dataframes}}

\begin{Shaded}
\begin{Highlighting}[]
\CommentTok{\# A quick preview of Daily Datasets}
\FunctionTok{print}\NormalTok{(}\StringTok{"{-}{-}{-}{-}{-}Daily Activity{-}{-}{-}{-}{-}"}\NormalTok{)}
\end{Highlighting}
\end{Shaded}

\begin{verbatim}
## [1] "-----Daily Activity-----"
\end{verbatim}

\begin{Shaded}
\begin{Highlighting}[]
\FunctionTok{glimpse}\NormalTok{(daily\_activity)}
\end{Highlighting}
\end{Shaded}

\begin{verbatim}
## Rows: 940
## Columns: 15
## $ Id                       <dbl> 1503960366, 1503960366, 1503960366, 150396036~
## $ ActivityDate             <chr> "4/12/2016", "4/13/2016", "4/14/2016", "4/15/~
## $ TotalSteps               <dbl> 13162, 10735, 10460, 9762, 12669, 9705, 13019~
## $ TotalDistance            <dbl> 8.50, 6.97, 6.74, 6.28, 8.16, 6.48, 8.59, 9.8~
## $ TrackerDistance          <dbl> 8.50, 6.97, 6.74, 6.28, 8.16, 6.48, 8.59, 9.8~
## $ LoggedActivitiesDistance <dbl> 0, 0, 0, 0, 0, 0, 0, 0, 0, 0, 0, 0, 0, 0, 0, ~
## $ VeryActiveDistance       <dbl> 1.88, 1.57, 2.44, 2.14, 2.71, 3.19, 3.25, 3.5~
## $ ModeratelyActiveDistance <dbl> 0.55, 0.69, 0.40, 1.26, 0.41, 0.78, 0.64, 1.3~
## $ LightActiveDistance      <dbl> 6.06, 4.71, 3.91, 2.83, 5.04, 2.51, 4.71, 5.0~
## $ SedentaryActiveDistance  <dbl> 0, 0, 0, 0, 0, 0, 0, 0, 0, 0, 0, 0, 0, 0, 0, ~
## $ VeryActiveMinutes        <dbl> 25, 21, 30, 29, 36, 38, 42, 50, 28, 19, 66, 4~
## $ FairlyActiveMinutes      <dbl> 13, 19, 11, 34, 10, 20, 16, 31, 12, 8, 27, 21~
## $ LightlyActiveMinutes     <dbl> 328, 217, 181, 209, 221, 164, 233, 264, 205, ~
## $ SedentaryMinutes         <dbl> 728, 776, 1218, 726, 773, 539, 1149, 775, 818~
## $ Calories                 <dbl> 1985, 1797, 1776, 1745, 1863, 1728, 1921, 203~
\end{verbatim}

\begin{Shaded}
\begin{Highlighting}[]
\FunctionTok{print}\NormalTok{(}\StringTok{"{-}{-}{-}{-}{-}Daily Calories{-}{-}{-}{-}{-}"}\NormalTok{)}
\end{Highlighting}
\end{Shaded}

\begin{verbatim}
## [1] "-----Daily Calories-----"
\end{verbatim}

\begin{Shaded}
\begin{Highlighting}[]
\FunctionTok{glimpse}\NormalTok{(daily\_calories)}
\end{Highlighting}
\end{Shaded}

\begin{verbatim}
## Rows: 940
## Columns: 3
## $ Id          <dbl> 1503960366, 1503960366, 1503960366, 1503960366, 1503960366~
## $ ActivityDay <chr> "4/12/2016", "4/13/2016", "4/14/2016", "4/15/2016", "4/16/~
## $ Calories    <dbl> 1985, 1797, 1776, 1745, 1863, 1728, 1921, 2035, 1786, 1775~
\end{verbatim}

\begin{Shaded}
\begin{Highlighting}[]
\FunctionTok{print}\NormalTok{(}\StringTok{"{-}{-}{-}{-}{-}Daily Intensities{-}{-}{-}{-}{-}"}\NormalTok{)}
\end{Highlighting}
\end{Shaded}

\begin{verbatim}
## [1] "-----Daily Intensities-----"
\end{verbatim}

\begin{Shaded}
\begin{Highlighting}[]
\FunctionTok{glimpse}\NormalTok{(daily\_intensities)}
\end{Highlighting}
\end{Shaded}

\begin{verbatim}
## Rows: 940
## Columns: 10
## $ Id                       <dbl> 1503960366, 1503960366, 1503960366, 150396036~
## $ ActivityDay              <chr> "4/12/2016", "4/13/2016", "4/14/2016", "4/15/~
## $ SedentaryMinutes         <dbl> 728, 776, 1218, 726, 773, 539, 1149, 775, 818~
## $ LightlyActiveMinutes     <dbl> 328, 217, 181, 209, 221, 164, 233, 264, 205, ~
## $ FairlyActiveMinutes      <dbl> 13, 19, 11, 34, 10, 20, 16, 31, 12, 8, 27, 21~
## $ VeryActiveMinutes        <dbl> 25, 21, 30, 29, 36, 38, 42, 50, 28, 19, 66, 4~
## $ SedentaryActiveDistance  <dbl> 0, 0, 0, 0, 0, 0, 0, 0, 0, 0, 0, 0, 0, 0, 0, ~
## $ LightActiveDistance      <dbl> 6.06, 4.71, 3.91, 2.83, 5.04, 2.51, 4.71, 5.0~
## $ ModeratelyActiveDistance <dbl> 0.55, 0.69, 0.40, 1.26, 0.41, 0.78, 0.64, 1.3~
## $ VeryActiveDistance       <dbl> 1.88, 1.57, 2.44, 2.14, 2.71, 3.19, 3.25, 3.5~
\end{verbatim}

\begin{Shaded}
\begin{Highlighting}[]
\FunctionTok{print}\NormalTok{(}\StringTok{"{-}{-}{-}{-}{-}Daily Steps{-}{-}{-}{-}{-}"}\NormalTok{)}
\end{Highlighting}
\end{Shaded}

\begin{verbatim}
## [1] "-----Daily Steps-----"
\end{verbatim}

\begin{Shaded}
\begin{Highlighting}[]
\FunctionTok{glimpse}\NormalTok{(daily\_steps)}
\end{Highlighting}
\end{Shaded}

\begin{verbatim}
## Rows: 940
## Columns: 3
## $ Id          <dbl> 1503960366, 1503960366, 1503960366, 1503960366, 1503960366~
## $ ActivityDay <chr> "4/12/2016", "4/13/2016", "4/14/2016", "4/15/2016", "4/16/~
## $ StepTotal   <dbl> 13162, 10735, 10460, 9762, 12669, 9705, 13019, 15506, 1054~
\end{verbatim}

\begin{Shaded}
\begin{Highlighting}[]
\FunctionTok{print}\NormalTok{(}\StringTok{"{-}{-}{-}{-}{-}Daily Sleep{-}{-}{-}{-}{-}"}\NormalTok{)}
\end{Highlighting}
\end{Shaded}

\begin{verbatim}
## [1] "-----Daily Sleep-----"
\end{verbatim}

\begin{Shaded}
\begin{Highlighting}[]
\FunctionTok{glimpse}\NormalTok{(daily\_sleep)}
\end{Highlighting}
\end{Shaded}

\begin{verbatim}
## Rows: 413
## Columns: 5
## $ Id                 <dbl> 1503960366, 1503960366, 1503960366, 1503960366, 150~
## $ SleepDay           <chr> "4/12/2016 12:00:00 AM", "4/13/2016 12:00:00 AM", "~
## $ TotalSleepRecords  <dbl> 1, 2, 1, 2, 1, 1, 1, 1, 1, 1, 1, 1, 1, 1, 1, 1, 1, ~
## $ TotalMinutesAsleep <dbl> 327, 384, 412, 340, 700, 304, 360, 325, 361, 430, 2~
## $ TotalTimeInBed     <dbl> 346, 407, 442, 367, 712, 320, 377, 364, 384, 449, 3~
\end{verbatim}

\begin{Shaded}
\begin{Highlighting}[]
\FunctionTok{print}\NormalTok{(}\StringTok{"{-}{-}{-}{-}{-}Weight Log Info{-}{-}{-}{-}{-}"}\NormalTok{)}
\end{Highlighting}
\end{Shaded}

\begin{verbatim}
## [1] "-----Weight Log Info-----"
\end{verbatim}

\begin{Shaded}
\begin{Highlighting}[]
\FunctionTok{glimpse}\NormalTok{(weight\_log\_info)}
\end{Highlighting}
\end{Shaded}

\begin{verbatim}
## Rows: 67
## Columns: 8
## $ Id             <dbl> 1503960366, 1503960366, 1927972279, 2873212765, 2873212~
## $ Date           <chr> "5/2/2016 11:59:59 PM", "5/3/2016 11:59:59 PM", "4/13/2~
## $ WeightKg       <dbl> 52.6, 52.6, 133.5, 56.7, 57.3, 72.4, 72.3, 69.7, 70.3, ~
## $ WeightPounds   <dbl> 115.9631, 115.9631, 294.3171, 125.0021, 126.3249, 159.6~
## $ Fat            <dbl> 22, NA, NA, NA, NA, 25, NA, NA, NA, NA, NA, NA, NA, NA,~
## $ BMI            <dbl> 22.65, 22.65, 47.54, 21.45, 21.69, 27.45, 27.38, 27.25,~
## $ IsManualReport <lgl> TRUE, TRUE, FALSE, TRUE, TRUE, TRUE, TRUE, TRUE, TRUE, ~
## $ LogId          <dbl> 1.462234e+12, 1.462320e+12, 1.460510e+12, 1.461283e+12,~
\end{verbatim}

\hypertarget{hourly-dataframes}{%
\paragraph{Hourly DataFrames}\label{hourly-dataframes}}

\begin{Shaded}
\begin{Highlighting}[]
\CommentTok{\# A quick preview of Hourly Datasets}
\FunctionTok{print}\NormalTok{(}\StringTok{"{-}{-}{-}{-}{-}Hourly Calories{-}{-}{-}{-}{-}"}\NormalTok{)}
\end{Highlighting}
\end{Shaded}

\begin{verbatim}
## [1] "-----Hourly Calories-----"
\end{verbatim}

\begin{Shaded}
\begin{Highlighting}[]
\FunctionTok{glimpse}\NormalTok{(hourly\_calories)}
\end{Highlighting}
\end{Shaded}

\begin{verbatim}
## Rows: 22,099
## Columns: 3
## $ Id           <dbl> 1503960366, 1503960366, 1503960366, 1503960366, 150396036~
## $ ActivityHour <chr> "4/12/2016 12:00:00 AM", "4/12/2016 1:00:00 AM", "4/12/20~
## $ Calories     <dbl> 81, 61, 59, 47, 48, 48, 48, 47, 68, 141, 99, 76, 73, 66, ~
\end{verbatim}

\begin{Shaded}
\begin{Highlighting}[]
\FunctionTok{print}\NormalTok{(}\StringTok{"{-}{-}{-}{-}{-}Hourly Intensities{-}{-}{-}{-}{-}"}\NormalTok{)}
\end{Highlighting}
\end{Shaded}

\begin{verbatim}
## [1] "-----Hourly Intensities-----"
\end{verbatim}

\begin{Shaded}
\begin{Highlighting}[]
\FunctionTok{glimpse}\NormalTok{(hourly\_intensities)}
\end{Highlighting}
\end{Shaded}

\begin{verbatim}
## Rows: 22,099
## Columns: 4
## $ Id               <dbl> 1503960366, 1503960366, 1503960366, 1503960366, 15039~
## $ ActivityHour     <chr> "4/12/2016 12:00:00 AM", "4/12/2016 1:00:00 AM", "4/1~
## $ TotalIntensity   <dbl> 20, 8, 7, 0, 0, 0, 0, 0, 13, 30, 29, 12, 11, 6, 36, 5~
## $ AverageIntensity <dbl> 0.333333, 0.133333, 0.116667, 0.000000, 0.000000, 0.0~
\end{verbatim}

\begin{Shaded}
\begin{Highlighting}[]
\FunctionTok{print}\NormalTok{(}\StringTok{"{-}{-}{-}{-}{-}Hourly Steps{-}{-}{-}{-}{-}"}\NormalTok{)}
\end{Highlighting}
\end{Shaded}

\begin{verbatim}
## [1] "-----Hourly Steps-----"
\end{verbatim}

\begin{Shaded}
\begin{Highlighting}[]
\FunctionTok{glimpse}\NormalTok{(hourly\_steps)}
\end{Highlighting}
\end{Shaded}

\begin{verbatim}
## Rows: 22,099
## Columns: 3
## $ Id           <dbl> 1503960366, 1503960366, 1503960366, 1503960366, 150396036~
## $ ActivityHour <chr> "4/12/2016 12:00:00 AM", "4/12/2016 1:00:00 AM", "4/12/20~
## $ StepTotal    <dbl> 373, 160, 151, 0, 0, 0, 0, 0, 250, 1864, 676, 360, 253, 2~
\end{verbatim}

\hypertarget{minute-dataframes}{%
\paragraph{Minute DataFrames}\label{minute-dataframes}}

\begin{Shaded}
\begin{Highlighting}[]
\CommentTok{\# A quick preview of Minute Datasets}
\FunctionTok{print}\NormalTok{(}\StringTok{"{-}{-}{-}{-}{-}Minute Calories{-}{-}{-}{-}{-}"}\NormalTok{)}
\end{Highlighting}
\end{Shaded}

\begin{verbatim}
## [1] "-----Minute Calories-----"
\end{verbatim}

\begin{Shaded}
\begin{Highlighting}[]
\FunctionTok{glimpse}\NormalTok{(minute\_calories\_narrow)}
\end{Highlighting}
\end{Shaded}

\begin{verbatim}
## Rows: 1,325,580
## Columns: 3
## $ Id             <dbl> 1503960366, 1503960366, 1503960366, 1503960366, 1503960~
## $ ActivityMinute <chr> "4/12/2016 12:00:00 AM", "4/12/2016 12:01:00 AM", "4/12~
## $ Calories       <dbl> 0.7865, 0.7865, 0.7865, 0.7865, 0.7865, 0.9438, 0.9438,~
\end{verbatim}

\begin{Shaded}
\begin{Highlighting}[]
\FunctionTok{print}\NormalTok{(}\StringTok{"{-}{-}{-}{-}{-}Minute Intensities{-}{-}{-}{-}{-}"}\NormalTok{)}
\end{Highlighting}
\end{Shaded}

\begin{verbatim}
## [1] "-----Minute Intensities-----"
\end{verbatim}

\begin{Shaded}
\begin{Highlighting}[]
\FunctionTok{glimpse}\NormalTok{(minute\_intensities\_narrow)}
\end{Highlighting}
\end{Shaded}

\begin{verbatim}
## Rows: 1,325,580
## Columns: 3
## $ Id             <dbl> 1503960366, 1503960366, 1503960366, 1503960366, 1503960~
## $ ActivityMinute <chr> "4/12/2016 12:00:00 AM", "4/12/2016 12:01:00 AM", "4/12~
## $ Intensity      <dbl> 0, 0, 0, 0, 0, 0, 0, 0, 0, 0, 0, 0, 0, 0, 0, 0, 0, 0, 0~
\end{verbatim}

\begin{Shaded}
\begin{Highlighting}[]
\FunctionTok{print}\NormalTok{(}\StringTok{"{-}{-}{-}{-}{-}Minute METs{-}{-}{-}{-}{-}"}\NormalTok{)}
\end{Highlighting}
\end{Shaded}

\begin{verbatim}
## [1] "-----Minute METs-----"
\end{verbatim}

\begin{Shaded}
\begin{Highlighting}[]
\FunctionTok{glimpse}\NormalTok{(minute\_METs\_narrow)}
\end{Highlighting}
\end{Shaded}

\begin{verbatim}
## Rows: 1,325,580
## Columns: 3
## $ Id             <dbl> 1503960366, 1503960366, 1503960366, 1503960366, 1503960~
## $ ActivityMinute <chr> "4/12/2016 12:00:00 AM", "4/12/2016 12:01:00 AM", "4/12~
## $ METs           <dbl> 10, 10, 10, 10, 10, 12, 12, 12, 12, 12, 12, 12, 10, 10,~
\end{verbatim}

\begin{Shaded}
\begin{Highlighting}[]
\FunctionTok{print}\NormalTok{(}\StringTok{"{-}{-}{-}{-}{-}Minute Sleep{-}{-}{-}{-}{-}"}\NormalTok{)}
\end{Highlighting}
\end{Shaded}

\begin{verbatim}
## [1] "-----Minute Sleep-----"
\end{verbatim}

\begin{Shaded}
\begin{Highlighting}[]
\FunctionTok{glimpse}\NormalTok{(minute\_sleep)}
\end{Highlighting}
\end{Shaded}

\begin{verbatim}
## Rows: 188,521
## Columns: 4
## $ Id    <dbl> 1503960366, 1503960366, 1503960366, 1503960366, 1503960366, 1503~
## $ date  <chr> "4/12/2016 2:47:30 AM", "4/12/2016 2:48:30 AM", "4/12/2016 2:49:~
## $ value <dbl> 3, 2, 1, 1, 1, 1, 1, 2, 2, 2, 3, 3, 3, 3, 3, 2, 1, 1, 1, 1, 1, 1~
## $ logId <dbl> 11380564589, 11380564589, 11380564589, 11380564589, 11380564589,~
\end{verbatim}

\begin{Shaded}
\begin{Highlighting}[]
\FunctionTok{print}\NormalTok{(}\StringTok{"{-}{-}{-}{-}{-}Minute Steps{-}{-}{-}{-}{-}"}\NormalTok{)}
\end{Highlighting}
\end{Shaded}

\begin{verbatim}
## [1] "-----Minute Steps-----"
\end{verbatim}

\begin{Shaded}
\begin{Highlighting}[]
\FunctionTok{glimpse}\NormalTok{(minute\_steps\_narrow)}
\end{Highlighting}
\end{Shaded}

\begin{verbatim}
## Rows: 1,325,580
## Columns: 3
## $ Id             <dbl> 1503960366, 1503960366, 1503960366, 1503960366, 1503960~
## $ ActivityMinute <chr> "4/12/2016 12:00:00 AM", "4/12/2016 12:01:00 AM", "4/12~
## $ Steps          <dbl> 0, 0, 0, 0, 0, 0, 0, 0, 0, 0, 0, 0, 0, 0, 0, 0, 0, 0, 0~
\end{verbatim}

\hypertarget{second-dataframe}{%
\paragraph{Second DataFrame}\label{second-dataframe}}

\begin{Shaded}
\begin{Highlighting}[]
\CommentTok{\# A quick preview of Second Datasets}
\FunctionTok{print}\NormalTok{(}\StringTok{"{-}{-}{-}{-}{-}Heartrate Seconds{-}{-}{-}{-}{-}"}\NormalTok{)}
\end{Highlighting}
\end{Shaded}

\begin{verbatim}
## [1] "-----Heartrate Seconds-----"
\end{verbatim}

\begin{Shaded}
\begin{Highlighting}[]
\FunctionTok{glimpse}\NormalTok{(heartrate\_seconds)}
\end{Highlighting}
\end{Shaded}

\begin{verbatim}
## Rows: 2,483,658
## Columns: 3
## $ Id    <dbl> 2022484408, 2022484408, 2022484408, 2022484408, 2022484408, 2022~
## $ Time  <chr> "4/12/2016 7:21:00 AM", "4/12/2016 7:21:05 AM", "4/12/2016 7:21:~
## $ Value <dbl> 97, 102, 105, 103, 101, 95, 91, 93, 94, 93, 92, 89, 83, 61, 60, ~
\end{verbatim}

\hypertarget{data-cleaning}{%
\subsubsection{3.4 Data Cleaning}\label{data-cleaning}}

After exploring the datasets, various data integrity issues have been
identified, highlighting the need for thorough data cleaning measures.
This process is essential to ensure the quality and accuracy of the data
before proceeding with any further analysis.

\hypertarget{standardizing-naming-conventions-and-formatting-date-variables}{%
\paragraph{3.4.1 Standardizing Naming Conventions and Formatting Date
Variables}\label{standardizing-naming-conventions-and-formatting-date-variables}}

\begin{enumerate}
\def\labelenumi{\arabic{enumi}.}
\tightlist
\item
  \textbf{Issue: Inconsistent Naming Conventions}

  \begin{itemize}
  \tightlist
  \item
    The problem identified here is that there is inconsistency in how
    names are formatted in the dataframes. Naming conventions are
    important because they help in maintaining uniformity and clarity in
    code, making it easier to read and manage. In this case, most of the
    dataframe names are in ``UpperCamelCase'' format. This format starts
    each new word with a capital letter and joins them without spaces,
    like \texttt{UpperCamelCase}.
  \item
    However, the name ``minute\_sleep'' breaks this pattern by using a
    mix of UpperCamelCase (``minute'') and lowercase (``sleep'')
    formats.
  \item
    the issue can be seen when we ran the command
    \texttt{glimpse(minute\_sleep)}
  \end{itemize}
\end{enumerate}

----\textgreater{} \textbf{Solution: Standardizing Variable Names} - To
resolve the inconsistency, the proposed solution is to use the
\texttt{clean\_names()} function. This function is typically used to
standardize the naming format of variables across all dataframes by
converting them to a consistent format. - The chosen format for
standardization is ``snake\_case''. In snake\_case format, words are
typically all lowercase and separated by underscores, like
\texttt{snake\_case}.

\begin{enumerate}
\def\labelenumi{\arabic{enumi}.}
\setcounter{enumi}{1}
\tightlist
\item
  \textbf{Issue: Different Names for Variables with the Same Meaning}

  \begin{itemize}
  \tightlist
  \item
    In the described scenario, although most of the variable names are
    consistent across the dataframes (like ``Id''), the variables that
    represent dates are named differently in each dataframe. Examples
    given include ``ActivityDate,'' ``ActivityDay,'' and ``Time.'' These
    variations can create confusion and complicate data processing tasks
    such as merging dataframes, as each dataframe refers to the date
    with a different title.
  \end{itemize}
\end{enumerate}

----\textgreater{} \textbf{Solution: Standardizing Variable Names} - To
address this inconsistency, the solution is to rename these variables so
that they all have the same title. Specifically, the variables
``ActivityDate,'' ``ActivityDay,'' and ``Time'' will all be renamed to
``activity\_date.'' - Choosing ``activity\_date'' as a standard name for
all date columns across the datasets ensures uniformity. This
standardization is particularly helpful when performing data merging.
When datasets are merged, it is essential that the corresponding columns
that are supposed to be identical across tables are indeed identical in
name and format. By standardizing the date columns to
``activity\_date,'' it ensures that when the datasets are combined,
there will be no ambiguity or error in aligning data related to dates.

\begin{enumerate}
\def\labelenumi{\arabic{enumi}.}
\setcounter{enumi}{2}
\tightlist
\item
  \textbf{Issue: Date Column's char Format}
\end{enumerate}

-Right now, the date information in the column is saved as text, which
makes it hard to work with. To fix this, we'll make new columns where
the dates will be changed into standard date or date-time formats using
the \texttt{mdy()} and \texttt{mdy\_hms()} functions. This will make
sure the dates look the same across all the datasets.

\hypertarget{daily-dataframes-1}{%
\subparagraph{Daily DataFrames}\label{daily-dataframes-1}}

\textbf{1. Load and Clean Daily Activity Data}

\begin{Shaded}
\begin{Highlighting}[]
\CommentTok{\# Load daily\_activity data and standardize column names to lowercase with underscores using clean\_names()}
\NormalTok{clean\_daily\_activity }\OtherTok{\textless{}{-}}\NormalTok{ daily\_activity }\SpecialCharTok{\%\textgreater{}\%} \FunctionTok{clean\_names}\NormalTok{() }
\CommentTok{\# Convert the activity\_date column from MM/DD/YYYY format to a Date object}
\NormalTok{clean\_daily\_activity}\SpecialCharTok{$}\NormalTok{activity\_date\_ymd }\OtherTok{\textless{}{-}} \FunctionTok{mdy}\NormalTok{(clean\_daily\_activity}\SpecialCharTok{$}\NormalTok{activity\_date)}
\CommentTok{\# Print the structure of the cleaned daily activity data}
\FunctionTok{print}\NormalTok{(}\StringTok{"{-}{-}{-}{-}{-}Clean Daily Activity{-}{-}{-}{-}{-}"}\NormalTok{)}
\end{Highlighting}
\end{Shaded}

\begin{verbatim}
## [1] "-----Clean Daily Activity-----"
\end{verbatim}

\begin{Shaded}
\begin{Highlighting}[]
\FunctionTok{glimpse}\NormalTok{(clean\_daily\_activity)}
\end{Highlighting}
\end{Shaded}

\begin{verbatim}
## Rows: 940
## Columns: 16
## $ id                         <dbl> 1503960366, 1503960366, 1503960366, 1503960~
## $ activity_date              <chr> "4/12/2016", "4/13/2016", "4/14/2016", "4/1~
## $ total_steps                <dbl> 13162, 10735, 10460, 9762, 12669, 9705, 130~
## $ total_distance             <dbl> 8.50, 6.97, 6.74, 6.28, 8.16, 6.48, 8.59, 9~
## $ tracker_distance           <dbl> 8.50, 6.97, 6.74, 6.28, 8.16, 6.48, 8.59, 9~
## $ logged_activities_distance <dbl> 0, 0, 0, 0, 0, 0, 0, 0, 0, 0, 0, 0, 0, 0, 0~
## $ very_active_distance       <dbl> 1.88, 1.57, 2.44, 2.14, 2.71, 3.19, 3.25, 3~
## $ moderately_active_distance <dbl> 0.55, 0.69, 0.40, 1.26, 0.41, 0.78, 0.64, 1~
## $ light_active_distance      <dbl> 6.06, 4.71, 3.91, 2.83, 5.04, 2.51, 4.71, 5~
## $ sedentary_active_distance  <dbl> 0, 0, 0, 0, 0, 0, 0, 0, 0, 0, 0, 0, 0, 0, 0~
## $ very_active_minutes        <dbl> 25, 21, 30, 29, 36, 38, 42, 50, 28, 19, 66,~
## $ fairly_active_minutes      <dbl> 13, 19, 11, 34, 10, 20, 16, 31, 12, 8, 27, ~
## $ lightly_active_minutes     <dbl> 328, 217, 181, 209, 221, 164, 233, 264, 205~
## $ sedentary_minutes          <dbl> 728, 776, 1218, 726, 773, 539, 1149, 775, 8~
## $ calories                   <dbl> 1985, 1797, 1776, 1745, 1863, 1728, 1921, 2~
## $ activity_date_ymd          <date> 2016-04-12, 2016-04-13, 2016-04-14, 2016-0~
\end{verbatim}

\textbf{2. Load and Clean Daily Calories Data}

\begin{Shaded}
\begin{Highlighting}[]
\CommentTok{\# Load daily\_calories data, rename ActivityDay column to activity\_date, and clean column names}
\NormalTok{clean\_daily\_calories }\OtherTok{\textless{}{-}}\NormalTok{ daily\_calories }\SpecialCharTok{\%\textgreater{}\%} 
    \FunctionTok{rename}\NormalTok{(}\AttributeTok{activity\_date =}\NormalTok{ ActivityDay) }\SpecialCharTok{\%\textgreater{}\%} 
    \FunctionTok{clean\_names}\NormalTok{()}
\CommentTok{\# Convert the activity\_date column to a Date object}
\NormalTok{clean\_daily\_calories}\SpecialCharTok{$}\NormalTok{activity\_date\_ymd }\OtherTok{\textless{}{-}} \FunctionTok{mdy}\NormalTok{(clean\_daily\_calories}\SpecialCharTok{$}\NormalTok{activity\_date)}
\CommentTok{\# Print the structure of the cleaned daily calories data}
\FunctionTok{print}\NormalTok{(}\StringTok{"{-}{-}{-}{-}{-}Clean Daily Calories{-}{-}{-}{-}{-}"}\NormalTok{)}
\end{Highlighting}
\end{Shaded}

\begin{verbatim}
## [1] "-----Clean Daily Calories-----"
\end{verbatim}

\begin{Shaded}
\begin{Highlighting}[]
\FunctionTok{glimpse}\NormalTok{(clean\_daily\_calories)}
\end{Highlighting}
\end{Shaded}

\begin{verbatim}
## Rows: 940
## Columns: 4
## $ id                <dbl> 1503960366, 1503960366, 1503960366, 1503960366, 1503~
## $ activity_date     <chr> "4/12/2016", "4/13/2016", "4/14/2016", "4/15/2016", ~
## $ calories          <dbl> 1985, 1797, 1776, 1745, 1863, 1728, 1921, 2035, 1786~
## $ activity_date_ymd <date> 2016-04-12, 2016-04-13, 2016-04-14, 2016-04-15, 201~
\end{verbatim}

\textbf{3. Load and Clean Daily Intensities Data}

\begin{Shaded}
\begin{Highlighting}[]
\CommentTok{\# Similar to daily calories, rename and clean columns}
\NormalTok{clean\_daily\_intensities }\OtherTok{\textless{}{-}}\NormalTok{ daily\_intensities }\SpecialCharTok{\%\textgreater{}\%} 
    \FunctionTok{rename}\NormalTok{(}\AttributeTok{activity\_date =}\NormalTok{ ActivityDay) }\SpecialCharTok{\%\textgreater{}\%} 
    \FunctionTok{clean\_names}\NormalTok{()}
\CommentTok{\# Convert date from string to Date object}
\NormalTok{clean\_daily\_intensities}\SpecialCharTok{$}\NormalTok{activity\_date\_ymd }\OtherTok{\textless{}{-}} \FunctionTok{mdy}\NormalTok{(clean\_daily\_intensities}\SpecialCharTok{$}\NormalTok{activity\_date)}
\CommentTok{\# Print the structure of cleaned data}
\FunctionTok{print}\NormalTok{(}\StringTok{"{-}{-}{-}{-}{-}Clean Daily Intensities{-}{-}{-}{-}{-}"}\NormalTok{)}
\end{Highlighting}
\end{Shaded}

\begin{verbatim}
## [1] "-----Clean Daily Intensities-----"
\end{verbatim}

\begin{Shaded}
\begin{Highlighting}[]
\FunctionTok{glimpse}\NormalTok{(clean\_daily\_intensities)}
\end{Highlighting}
\end{Shaded}

\begin{verbatim}
## Rows: 940
## Columns: 11
## $ id                         <dbl> 1503960366, 1503960366, 1503960366, 1503960~
## $ activity_date              <chr> "4/12/2016", "4/13/2016", "4/14/2016", "4/1~
## $ sedentary_minutes          <dbl> 728, 776, 1218, 726, 773, 539, 1149, 775, 8~
## $ lightly_active_minutes     <dbl> 328, 217, 181, 209, 221, 164, 233, 264, 205~
## $ fairly_active_minutes      <dbl> 13, 19, 11, 34, 10, 20, 16, 31, 12, 8, 27, ~
## $ very_active_minutes        <dbl> 25, 21, 30, 29, 36, 38, 42, 50, 28, 19, 66,~
## $ sedentary_active_distance  <dbl> 0, 0, 0, 0, 0, 0, 0, 0, 0, 0, 0, 0, 0, 0, 0~
## $ light_active_distance      <dbl> 6.06, 4.71, 3.91, 2.83, 5.04, 2.51, 4.71, 5~
## $ moderately_active_distance <dbl> 0.55, 0.69, 0.40, 1.26, 0.41, 0.78, 0.64, 1~
## $ very_active_distance       <dbl> 1.88, 1.57, 2.44, 2.14, 2.71, 3.19, 3.25, 3~
## $ activity_date_ymd          <date> 2016-04-12, 2016-04-13, 2016-04-14, 2016-0~
\end{verbatim}

\textbf{4. Load and Clean Daily Steps Data}

\begin{Shaded}
\begin{Highlighting}[]
\CommentTok{\# Load daily\_steps, rename ActivityDay and StepTotal, and clean names}
\NormalTok{clean\_daily\_steps }\OtherTok{\textless{}{-}}\NormalTok{ daily\_steps }\SpecialCharTok{\%\textgreater{}\%} 
    \FunctionTok{rename}\NormalTok{(}\AttributeTok{activity\_date =}\NormalTok{ ActivityDay,}
           \AttributeTok{total\_steps =}\NormalTok{ StepTotal) }\SpecialCharTok{\%\textgreater{}\%} 
    \FunctionTok{clean\_names}\NormalTok{()}
\CommentTok{\# Convert the activity\_date to a Date object}
\NormalTok{clean\_daily\_steps}\SpecialCharTok{$}\NormalTok{activity\_date\_ymd }\OtherTok{\textless{}{-}} \FunctionTok{mdy}\NormalTok{(clean\_daily\_steps}\SpecialCharTok{$}\NormalTok{activity\_date)}
\CommentTok{\# Print the structure of the cleaned data}
\FunctionTok{print}\NormalTok{(}\StringTok{"{-}{-}{-}{-}{-}Clean Daily Steps{-}{-}{-}{-}{-}"}\NormalTok{)}
\end{Highlighting}
\end{Shaded}

\begin{verbatim}
## [1] "-----Clean Daily Steps-----"
\end{verbatim}

\begin{Shaded}
\begin{Highlighting}[]
\FunctionTok{glimpse}\NormalTok{(clean\_daily\_steps)}
\end{Highlighting}
\end{Shaded}

\begin{verbatim}
## Rows: 940
## Columns: 4
## $ id                <dbl> 1503960366, 1503960366, 1503960366, 1503960366, 1503~
## $ activity_date     <chr> "4/12/2016", "4/13/2016", "4/14/2016", "4/15/2016", ~
## $ total_steps       <dbl> 13162, 10735, 10460, 9762, 12669, 9705, 13019, 15506~
## $ activity_date_ymd <date> 2016-04-12, 2016-04-13, 2016-04-14, 2016-04-15, 201~
\end{verbatim}

\textbf{5. Load and Clean Daily Sleep Data}

\begin{Shaded}
\begin{Highlighting}[]
\CommentTok{\# Rename and clean daily\_sleep data, considering sleep dates might include time}
\NormalTok{clean\_daily\_sleep }\OtherTok{\textless{}{-}}\NormalTok{ daily\_sleep }\SpecialCharTok{\%\textgreater{}\%} 
    \FunctionTok{rename}\NormalTok{(}\AttributeTok{activity\_date =}\NormalTok{ SleepDay) }\SpecialCharTok{\%\textgreater{}\%} 
    \FunctionTok{clean\_names}\NormalTok{() }
\CommentTok{\# Convert the activity\_date to a datetime object, considering possible time and timezone}
\NormalTok{clean\_daily\_sleep}\SpecialCharTok{$}\NormalTok{activity\_date\_ymd }\OtherTok{\textless{}{-}} \FunctionTok{mdy\_hms}\NormalTok{(clean\_daily\_sleep}\SpecialCharTok{$}\NormalTok{activity\_date, }\AttributeTok{tz=}\FunctionTok{Sys.timezone}\NormalTok{())}
\CommentTok{\# Print the structure of the cleaned sleep data}
\FunctionTok{print}\NormalTok{(}\StringTok{"{-}{-}{-}{-}{-}Clean Daily Sleep{-}{-}{-}{-}{-}"}\NormalTok{)}
\end{Highlighting}
\end{Shaded}

\begin{verbatim}
## [1] "-----Clean Daily Sleep-----"
\end{verbatim}

\begin{Shaded}
\begin{Highlighting}[]
\FunctionTok{glimpse}\NormalTok{(clean\_daily\_sleep)}
\end{Highlighting}
\end{Shaded}

\begin{verbatim}
## Rows: 413
## Columns: 6
## $ id                   <dbl> 1503960366, 1503960366, 1503960366, 1503960366, 1~
## $ activity_date        <chr> "4/12/2016 12:00:00 AM", "4/13/2016 12:00:00 AM",~
## $ total_sleep_records  <dbl> 1, 2, 1, 2, 1, 1, 1, 1, 1, 1, 1, 1, 1, 1, 1, 1, 1~
## $ total_minutes_asleep <dbl> 327, 384, 412, 340, 700, 304, 360, 325, 361, 430,~
## $ total_time_in_bed    <dbl> 346, 407, 442, 367, 712, 320, 377, 364, 384, 449,~
## $ activity_date_ymd    <dttm> 2016-04-12, 2016-04-13, 2016-04-15, 2016-04-16, ~
\end{verbatim}

\textbf{6. Load and Clean Weight Log Data}

\begin{Shaded}
\begin{Highlighting}[]
\CommentTok{\# Rename, reformat, and clean the weight\_log\_info data}
\NormalTok{clean\_weight\_log\_info }\OtherTok{\textless{}{-}}\NormalTok{ weight\_log\_info }\SpecialCharTok{\%\textgreater{}\%} 
    \FunctionTok{rename}\NormalTok{(}\AttributeTok{activity\_date =}\NormalTok{ Date,}
           \AttributeTok{weight\_kg =}\NormalTok{ WeightKg,}
           \AttributeTok{weight\_lb =}\NormalTok{ WeightPounds,}
           \AttributeTok{manual\_report =}\NormalTok{ IsManualReport) }\SpecialCharTok{\%\textgreater{}\%} 
    \FunctionTok{clean\_names}\NormalTok{()}
\CommentTok{\# Convert the activity\_date to a datetime object considering timezone}
\NormalTok{clean\_weight\_log\_info}\SpecialCharTok{$}\NormalTok{activity\_date\_ymdhms }\OtherTok{\textless{}{-}} \FunctionTok{mdy\_hms}\NormalTok{(clean\_weight\_log\_info}\SpecialCharTok{$}\NormalTok{activity\_date, }\AttributeTok{tz=}\FunctionTok{Sys.timezone}\NormalTok{())}
\CommentTok{\# Print the structure of the cleaned weight log data}
\FunctionTok{print}\NormalTok{(}\StringTok{"{-}{-}{-}{-}{-}Clean Weight Log Info{-}{-}{-}{-}{-}"}\NormalTok{)}
\end{Highlighting}
\end{Shaded}

\begin{verbatim}
## [1] "-----Clean Weight Log Info-----"
\end{verbatim}

\begin{Shaded}
\begin{Highlighting}[]
\FunctionTok{glimpse}\NormalTok{(clean\_weight\_log\_info)}
\end{Highlighting}
\end{Shaded}

\begin{verbatim}
## Rows: 67
## Columns: 9
## $ id                   <dbl> 1503960366, 1503960366, 1927972279, 2873212765, 2~
## $ activity_date        <chr> "5/2/2016 11:59:59 PM", "5/3/2016 11:59:59 PM", "~
## $ weight_kg            <dbl> 52.6, 52.6, 133.5, 56.7, 57.3, 72.4, 72.3, 69.7, ~
## $ weight_lb            <dbl> 115.9631, 115.9631, 294.3171, 125.0021, 126.3249,~
## $ fat                  <dbl> 22, NA, NA, NA, NA, 25, NA, NA, NA, NA, NA, NA, N~
## $ bmi                  <dbl> 22.65, 22.65, 47.54, 21.45, 21.69, 27.45, 27.38, ~
## $ manual_report        <lgl> TRUE, TRUE, FALSE, TRUE, TRUE, TRUE, TRUE, TRUE, ~
## $ log_id               <dbl> 1.462234e+12, 1.462320e+12, 1.460510e+12, 1.46128~
## $ activity_date_ymdhms <dttm> 2016-05-02 23:59:59, 2016-05-03 23:59:59, 2016-0~
\end{verbatim}

\hypertarget{hourly-dataframes-1}{%
\subparagraph{Hourly DataFrames}\label{hourly-dataframes-1}}

\textbf{1. Cleaning Hourly Calories Data}

\begin{Shaded}
\begin{Highlighting}[]
\CommentTok{\# Load hourly\_calories data and standardize column names using clean\_names(), which converts all column names to lowercase with underscores.}
\NormalTok{clean\_hourly\_calories }\OtherTok{\textless{}{-}}\NormalTok{ hourly\_calories }\SpecialCharTok{\%\textgreater{}\%} 
  \FunctionTok{clean\_names}\NormalTok{()}
\CommentTok{\# Convert the activity\_hour column from a string to a POSIXct datetime object, which includes date and time, adjusting for the system\textquotesingle{}s timezone.}
\NormalTok{clean\_hourly\_calories}\SpecialCharTok{$}\NormalTok{activity\_hour\_ymdhms }\OtherTok{\textless{}{-}} \FunctionTok{mdy\_hms}\NormalTok{(clean\_hourly\_calories}\SpecialCharTok{$}\NormalTok{activity\_hour, }\AttributeTok{tz=}\FunctionTok{Sys.timezone}\NormalTok{())}
\CommentTok{\# Print a statement to indicate the clean state of the data.}
\FunctionTok{print}\NormalTok{(}\StringTok{"{-}{-}{-}{-}{-}Clean Hourly Calories{-}{-}{-}{-}{-}"}\NormalTok{)}
\end{Highlighting}
\end{Shaded}

\begin{verbatim}
## [1] "-----Clean Hourly Calories-----"
\end{verbatim}

\begin{Shaded}
\begin{Highlighting}[]
\CommentTok{\# Use glimpse() to provide a concise summary of the dataframe, displaying the structure and a preview of the data.}
\FunctionTok{glimpse}\NormalTok{(clean\_hourly\_calories)}
\end{Highlighting}
\end{Shaded}

\begin{verbatim}
## Rows: 22,099
## Columns: 4
## $ id                   <dbl> 1503960366, 1503960366, 1503960366, 1503960366, 1~
## $ activity_hour        <chr> "4/12/2016 12:00:00 AM", "4/12/2016 1:00:00 AM", ~
## $ calories             <dbl> 81, 61, 59, 47, 48, 48, 48, 47, 68, 141, 99, 76, ~
## $ activity_hour_ymdhms <dttm> 2016-04-12 00:00:00, 2016-04-12 01:00:00, 2016-0~
\end{verbatim}

\textbf{2. Cleaning Hourly Intensities Data}

\begin{Shaded}
\begin{Highlighting}[]
\CommentTok{\# Load hourly\_intensities data and apply clean\_names() for consistent naming.}
\NormalTok{clean\_hourly\_intensities }\OtherTok{\textless{}{-}}\NormalTok{ hourly\_intensities }\SpecialCharTok{\%\textgreater{}\%} 
  \FunctionTok{clean\_names}\NormalTok{()}
\CommentTok{\# Convert the activity\_hour field from a string to a datetime object, applying the system\textquotesingle{}s timezone setting for correct time representation.}
\NormalTok{clean\_hourly\_intensities}\SpecialCharTok{$}\NormalTok{activity\_hour\_ymdhms }\OtherTok{\textless{}{-}} \FunctionTok{mdy\_hms}\NormalTok{(clean\_hourly\_intensities}\SpecialCharTok{$}\NormalTok{activity\_hour, }\AttributeTok{tz=}\FunctionTok{Sys.timezone}\NormalTok{())}
\CommentTok{\# Print a header for clarity when viewing output.}
\FunctionTok{print}\NormalTok{(}\StringTok{"{-}{-}{-}{-}{-}Clean Hourly Intensities{-}{-}{-}{-}{-}"}\NormalTok{)}
\end{Highlighting}
\end{Shaded}

\begin{verbatim}
## [1] "-----Clean Hourly Intensities-----"
\end{verbatim}

\begin{Shaded}
\begin{Highlighting}[]
\CommentTok{\# Output the structure of the cleaned data to check the changes and ensure correct formatting.}
\FunctionTok{glimpse}\NormalTok{(clean\_hourly\_intensities)}
\end{Highlighting}
\end{Shaded}

\begin{verbatim}
## Rows: 22,099
## Columns: 5
## $ id                   <dbl> 1503960366, 1503960366, 1503960366, 1503960366, 1~
## $ activity_hour        <chr> "4/12/2016 12:00:00 AM", "4/12/2016 1:00:00 AM", ~
## $ total_intensity      <dbl> 20, 8, 7, 0, 0, 0, 0, 0, 13, 30, 29, 12, 11, 6, 3~
## $ average_intensity    <dbl> 0.333333, 0.133333, 0.116667, 0.000000, 0.000000,~
## $ activity_hour_ymdhms <dttm> 2016-04-12 00:00:00, 2016-04-12 01:00:00, 2016-0~
\end{verbatim}

\textbf{3. Cleaning Hourly Steps Data}

\begin{Shaded}
\begin{Highlighting}[]
\CommentTok{\# Load hourly\_steps, renaming StepTotal to total\_steps for consistency, then apply clean\_names() to all column names.}
\NormalTok{clean\_hourly\_steps }\OtherTok{\textless{}{-}}\NormalTok{ hourly\_steps }\SpecialCharTok{\%\textgreater{}\%} 
  \FunctionTok{rename}\NormalTok{(}\AttributeTok{total\_steps =}\NormalTok{ StepTotal) }\SpecialCharTok{\%\textgreater{}\%} 
  \FunctionTok{clean\_names}\NormalTok{()}
\CommentTok{\# Convert the activity\_hour column into a datetime object with the correct timezone.}
\NormalTok{clean\_hourly\_steps}\SpecialCharTok{$}\NormalTok{activity\_hour\_ymdhms }\OtherTok{\textless{}{-}} \FunctionTok{mdy\_hms}\NormalTok{(clean\_hourly\_steps}\SpecialCharTok{$}\NormalTok{activity\_hour, }\AttributeTok{tz=}\FunctionTok{Sys.timezone}\NormalTok{())}
\CommentTok{\# Print a descriptive message to indicate the completion of cleaning for this specific dataframe.}
\FunctionTok{print}\NormalTok{(}\StringTok{"{-}{-}{-}{-}{-}Clean Hourly Steps{-}{-}{-}{-}{-}"}\NormalTok{)}
\end{Highlighting}
\end{Shaded}

\begin{verbatim}
## [1] "-----Clean Hourly Steps-----"
\end{verbatim}

\begin{Shaded}
\begin{Highlighting}[]
\CommentTok{\# Display the structure and partial contents of the dataframe to verify column names and types.}
\FunctionTok{glimpse}\NormalTok{(clean\_hourly\_steps)}
\end{Highlighting}
\end{Shaded}

\begin{verbatim}
## Rows: 22,099
## Columns: 4
## $ id                   <dbl> 1503960366, 1503960366, 1503960366, 1503960366, 1~
## $ activity_hour        <chr> "4/12/2016 12:00:00 AM", "4/12/2016 1:00:00 AM", ~
## $ total_steps          <dbl> 373, 160, 151, 0, 0, 0, 0, 0, 250, 1864, 676, 360~
## $ activity_hour_ymdhms <dttm> 2016-04-12 00:00:00, 2016-04-12 01:00:00, 2016-0~
\end{verbatim}

\hypertarget{minute-dataframes-1}{%
\subparagraph{Minute DataFrames}\label{minute-dataframes-1}}

\textbf{1. Cleaning Minute-Level Calories Data}

\begin{Shaded}
\begin{Highlighting}[]
\CommentTok{\# Load minute\_calories\_narrow data and standardize column names to be more consistent and easier to handle}
\NormalTok{clean\_minute\_calories }\OtherTok{\textless{}{-}}\NormalTok{ minute\_calories\_narrow }\SpecialCharTok{\%\textgreater{}\%} 
  \FunctionTok{clean\_names}\NormalTok{()}
\CommentTok{\# Convert the activity\_minute column from string format to a POSIXct datetime object considering the system\textquotesingle{}s timezone}
\NormalTok{clean\_minute\_calories}\SpecialCharTok{$}\NormalTok{activity\_minute\_ymdhms }\OtherTok{\textless{}{-}} \FunctionTok{mdy\_hms}\NormalTok{(clean\_minute\_calories}\SpecialCharTok{$}\NormalTok{activity\_minute, }\AttributeTok{tz=}\FunctionTok{Sys.timezone}\NormalTok{())}
\CommentTok{\# Print a statement indicating the dataset being cleaned}
\FunctionTok{print}\NormalTok{(}\StringTok{"{-}{-}{-}{-}{-}Clean Minute Calories{-}{-}{-}{-}{-}"}\NormalTok{)}
\end{Highlighting}
\end{Shaded}

\begin{verbatim}
## [1] "-----Clean Minute Calories-----"
\end{verbatim}

\begin{Shaded}
\begin{Highlighting}[]
\CommentTok{\# Display the structure and a brief preview of the data}
\FunctionTok{glimpse}\NormalTok{(clean\_minute\_calories)}
\end{Highlighting}
\end{Shaded}

\begin{verbatim}
## Rows: 1,325,580
## Columns: 4
## $ id                     <dbl> 1503960366, 1503960366, 1503960366, 1503960366,~
## $ activity_minute        <chr> "4/12/2016 12:00:00 AM", "4/12/2016 12:01:00 AM~
## $ calories               <dbl> 0.7865, 0.7865, 0.7865, 0.7865, 0.7865, 0.9438,~
## $ activity_minute_ymdhms <dttm> 2016-04-12 00:00:00, 2016-04-12 00:01:00, 2016~
\end{verbatim}

\textbf{2. Cleaning Minute-Level Intensities Data}

\begin{Shaded}
\begin{Highlighting}[]
\CommentTok{\# Standardize column names for the minute\_intensities\_narrow dataset}
\NormalTok{clean\_minute\_intensities }\OtherTok{\textless{}{-}}\NormalTok{ minute\_intensities\_narrow }\SpecialCharTok{\%\textgreater{}\%} 
  \FunctionTok{clean\_names}\NormalTok{()}
\CommentTok{\# Convert the datetime information into a properly formatted datetime object}
\NormalTok{clean\_minute\_intensities}\SpecialCharTok{$}\NormalTok{activity\_minute\_ymdhms }\OtherTok{\textless{}{-}} \FunctionTok{mdy\_hms}\NormalTok{(clean\_minute\_intensities}\SpecialCharTok{$}\NormalTok{activity\_minute, }\AttributeTok{tz=}\FunctionTok{Sys.timezone}\NormalTok{())}
\CommentTok{\# Output the cleaning status}
\FunctionTok{print}\NormalTok{(}\StringTok{"{-}{-}{-}{-}{-}Clean Minute Intensities{-}{-}{-}{-}{-}"}\NormalTok{)}
\end{Highlighting}
\end{Shaded}

\begin{verbatim}
## [1] "-----Clean Minute Intensities-----"
\end{verbatim}

\begin{Shaded}
\begin{Highlighting}[]
\CommentTok{\# Use glimpse to view the structure and confirm changes}
\FunctionTok{glimpse}\NormalTok{(clean\_minute\_intensities)}
\end{Highlighting}
\end{Shaded}

\begin{verbatim}
## Rows: 1,325,580
## Columns: 4
## $ id                     <dbl> 1503960366, 1503960366, 1503960366, 1503960366,~
## $ activity_minute        <chr> "4/12/2016 12:00:00 AM", "4/12/2016 12:01:00 AM~
## $ intensity              <dbl> 0, 0, 0, 0, 0, 0, 0, 0, 0, 0, 0, 0, 0, 0, 0, 0,~
## $ activity_minute_ymdhms <dttm> 2016-04-12 00:00:00, 2016-04-12 00:01:00, 2016~
\end{verbatim}

\textbf{3. Cleaning Minute-Level METs Data}

\begin{Shaded}
\begin{Highlighting}[]
\CommentTok{\# Load minute\_METs\_narrow data, rename the METs column for clarity, and clean all column names}
\NormalTok{clean\_minute\_METs }\OtherTok{\textless{}{-}}\NormalTok{ minute\_METs\_narrow }\SpecialCharTok{\%\textgreater{}\%} 
  \FunctionTok{rename}\NormalTok{(}\AttributeTok{mets =}\NormalTok{ METs) }\SpecialCharTok{\%\textgreater{}\%} 
  \FunctionTok{clean\_names}\NormalTok{()}
\CommentTok{\# Adjust the datetime column for timezone correctness}
\NormalTok{clean\_minute\_METs}\SpecialCharTok{$}\NormalTok{activity\_minute\_ymdhms }\OtherTok{\textless{}{-}} \FunctionTok{mdy\_hms}\NormalTok{(clean\_minute\_METs}\SpecialCharTok{$}\NormalTok{activity\_minute, }\AttributeTok{tz=}\FunctionTok{Sys.timezone}\NormalTok{())}
\CommentTok{\# Print a clean data statement}
\FunctionTok{print}\NormalTok{(}\StringTok{"{-}{-}{-}{-}{-}Clean Minute METs{-}{-}{-}{-}{-}"}\NormalTok{)}
\end{Highlighting}
\end{Shaded}

\begin{verbatim}
## [1] "-----Clean Minute METs-----"
\end{verbatim}

\begin{Shaded}
\begin{Highlighting}[]
\CommentTok{\# Check the structure of the cleaned data}
\FunctionTok{glimpse}\NormalTok{(clean\_minute\_METs)}
\end{Highlighting}
\end{Shaded}

\begin{verbatim}
## Rows: 1,325,580
## Columns: 4
## $ id                     <dbl> 1503960366, 1503960366, 1503960366, 1503960366,~
## $ activity_minute        <chr> "4/12/2016 12:00:00 AM", "4/12/2016 12:01:00 AM~
## $ mets                   <dbl> 10, 10, 10, 10, 10, 12, 12, 12, 12, 12, 12, 12,~
## $ activity_minute_ymdhms <dttm> 2016-04-12 00:00:00, 2016-04-12 00:01:00, 2016~
\end{verbatim}

\textbf{4. Cleaning Minute-Level Sleep Data}

\begin{Shaded}
\begin{Highlighting}[]
\CommentTok{\# Rename and clean columns in the minute\_sleep dataframe for consistency}
\NormalTok{clean\_minute\_sleep }\OtherTok{\textless{}{-}}\NormalTok{ minute\_sleep }\SpecialCharTok{\%\textgreater{}\%} 
  \FunctionTok{rename}\NormalTok{(}\AttributeTok{activity\_minute =}\NormalTok{ date,}
         \AttributeTok{sleep\_state =}\NormalTok{ value) }\SpecialCharTok{\%\textgreater{}\%} 
  \FunctionTok{clean\_names}\NormalTok{()}
\CommentTok{\# Format the activity\_minute column as a datetime object with appropriate timezone}
\NormalTok{clean\_minute\_sleep}\SpecialCharTok{$}\NormalTok{activity\_minute\_ymdhms }\OtherTok{\textless{}{-}} \FunctionTok{mdy\_hms}\NormalTok{(clean\_minute\_sleep}\SpecialCharTok{$}\NormalTok{activity\_minute, }\AttributeTok{tz=}\FunctionTok{Sys.timezone}\NormalTok{())}
\CommentTok{\# Indicate the dataset being cleaned}
\FunctionTok{print}\NormalTok{(}\StringTok{"{-}{-}{-}{-}{-}Clean Minute Sleep{-}{-}{-}{-}{-}"}\NormalTok{)}
\end{Highlighting}
\end{Shaded}

\begin{verbatim}
## [1] "-----Clean Minute Sleep-----"
\end{verbatim}

\begin{Shaded}
\begin{Highlighting}[]
\CommentTok{\# Display the structure to confirm changes}
\FunctionTok{glimpse}\NormalTok{(clean\_minute\_sleep)}
\end{Highlighting}
\end{Shaded}

\begin{verbatim}
## Rows: 188,521
## Columns: 5
## $ id                     <dbl> 1503960366, 1503960366, 1503960366, 1503960366,~
## $ activity_minute        <chr> "4/12/2016 2:47:30 AM", "4/12/2016 2:48:30 AM",~
## $ sleep_state            <dbl> 3, 2, 1, 1, 1, 1, 1, 2, 2, 2, 3, 3, 3, 3, 3, 2,~
## $ log_id                 <dbl> 11380564589, 11380564589, 11380564589, 11380564~
## $ activity_minute_ymdhms <dttm> 2016-04-12 02:47:30, 2016-04-12 02:48:30, 2016~
\end{verbatim}

\textbf{4. Cleaning Minute-Level Steps Data}

\begin{Shaded}
\begin{Highlighting}[]
\CommentTok{\# Rename the Steps column to total\_steps and apply standard naming conventions}
\NormalTok{clean\_minute\_steps }\OtherTok{\textless{}{-}}\NormalTok{ minute\_steps\_narrow }\SpecialCharTok{\%\textgreater{}\%} 
  \FunctionTok{rename}\NormalTok{(}\AttributeTok{total\_steps =}\NormalTok{ Steps) }\SpecialCharTok{\%\textgreater{}\%} 
  \FunctionTok{clean\_names}\NormalTok{()}
\CommentTok{\# Convert the time string to a datetime object including timezone adjustments}
\NormalTok{clean\_minute\_steps}\SpecialCharTok{$}\NormalTok{activity\_minute\_ymdhms }\OtherTok{\textless{}{-}} \FunctionTok{mdy\_hms}\NormalTok{(clean\_minute\_steps}\SpecialCharTok{$}\NormalTok{activity\_minute, }\AttributeTok{tz=}\FunctionTok{Sys.timezone}\NormalTok{())}
\CommentTok{\# Announce the cleaned data}
\FunctionTok{print}\NormalTok{(}\StringTok{"{-}{-}{-}{-}{-}Clean Minute Steps{-}{-}{-}{-}{-}"}\NormalTok{)}
\end{Highlighting}
\end{Shaded}

\begin{verbatim}
## [1] "-----Clean Minute Steps-----"
\end{verbatim}

\begin{Shaded}
\begin{Highlighting}[]
\CommentTok{\# Preview the dataframe\textquotesingle{}s structure post{-}cleaning}
\FunctionTok{glimpse}\NormalTok{(clean\_minute\_steps)}
\end{Highlighting}
\end{Shaded}

\begin{verbatim}
## Rows: 1,325,580
## Columns: 4
## $ id                     <dbl> 1503960366, 1503960366, 1503960366, 1503960366,~
## $ activity_minute        <chr> "4/12/2016 12:00:00 AM", "4/12/2016 12:01:00 AM~
## $ total_steps            <dbl> 0, 0, 0, 0, 0, 0, 0, 0, 0, 0, 0, 0, 0, 0, 0, 0,~
## $ activity_minute_ymdhms <dttm> 2016-04-12 00:00:00, 2016-04-12 00:01:00, 2016~
\end{verbatim}

\hypertarget{second-dataframes}{%
\subparagraph{Second DataFrames}\label{second-dataframes}}

\begin{Shaded}
\begin{Highlighting}[]
\CommentTok{\# Clean the heartrate\_seconds DataFrame}
\NormalTok{clean\_heartrate\_seconds }\OtherTok{\textless{}{-}}\NormalTok{ heartrate\_seconds }\SpecialCharTok{\%\textgreater{}\%}
  \CommentTok{\# Renaming columns for clarity and ease of understanding}
  \FunctionTok{rename}\NormalTok{(}\AttributeTok{activity\_second =}\NormalTok{ Time,     }\CommentTok{\# Renames \textquotesingle{}Time\textquotesingle{} to \textquotesingle{}activity\_second\textquotesingle{} for better context}
         \AttributeTok{heartrate =}\NormalTok{ Value) }\SpecialCharTok{\%\textgreater{}\%}      \CommentTok{\# Renames \textquotesingle{}Value\textquotesingle{} to \textquotesingle{}heartrate\textquotesingle{} to clearly indicate the data it represents}
  \CommentTok{\# Standardize all column names to lower case and replace spaces with underscores}
  \FunctionTok{clean\_names}\NormalTok{() }\SpecialCharTok{\%\textgreater{}\%}
  \CommentTok{\# Convert the activity\_second column from string format to a POSIXct datetime object}
  \CommentTok{\# The timezone is set to the system\textquotesingle{}s current timezone for accurate time representation}
  \FunctionTok{mutate}\NormalTok{(}\AttributeTok{activity\_second\_ymdhms =} \FunctionTok{mdy\_hms}\NormalTok{(activity\_second, }\AttributeTok{tz =} \FunctionTok{Sys.timezone}\NormalTok{()))}

\CommentTok{\# Print a message indicating that the cleaning process is complete}
\FunctionTok{print}\NormalTok{(}\StringTok{"{-}{-}{-}{-}{-}Clean Heartrate Seconds{-}{-}{-}{-}{-}"}\NormalTok{)}
\end{Highlighting}
\end{Shaded}

\begin{verbatim}
## [1] "-----Clean Heartrate Seconds-----"
\end{verbatim}

\begin{Shaded}
\begin{Highlighting}[]
\CommentTok{\# Use glimpse() to provide a quick overview of the dataframe structure,}
\CommentTok{\# confirming that columns have been renamed and datetime formatting applied correctly}
\FunctionTok{glimpse}\NormalTok{(clean\_heartrate\_seconds)}
\end{Highlighting}
\end{Shaded}

\begin{verbatim}
## Rows: 2,483,658
## Columns: 4
## $ id                     <dbl> 2022484408, 2022484408, 2022484408, 2022484408,~
## $ activity_second        <chr> "4/12/2016 7:21:00 AM", "4/12/2016 7:21:05 AM",~
## $ heartrate              <dbl> 97, 102, 105, 103, 101, 95, 91, 93, 94, 93, 92,~
## $ activity_second_ymdhms <dttm> 2016-04-12 07:21:00, 2016-04-12 07:21:05, 2016~
\end{verbatim}

\hypertarget{unique-participant-ids}{%
\paragraph{3.4.2 Unique Participant IDs}\label{unique-participant-ids}}

To confirm the consistency of the data, it is essential to verify that
each dataset includes 30 unique participant IDs as indicated in the data
description. This will be done by applying the \texttt{n\_unique()}
function to count and check the number of unique participants in each
dataset.

\hypertarget{daily-dataframes-2}{%
\subparagraph{Daily DataFrames}\label{daily-dataframes-2}}

\begin{Shaded}
\begin{Highlighting}[]
\CommentTok{\# Unique participant IDs: Daily DataFrames}
\FunctionTok{print}\NormalTok{(}\StringTok{"Daily Activity"}\NormalTok{)}
\end{Highlighting}
\end{Shaded}

\begin{verbatim}
## [1] "Daily Activity"
\end{verbatim}

\begin{Shaded}
\begin{Highlighting}[]
\FunctionTok{n\_unique}\NormalTok{(clean\_daily\_activity}\SpecialCharTok{$}\NormalTok{id)}
\end{Highlighting}
\end{Shaded}

\begin{verbatim}
## [1] 33
\end{verbatim}

\begin{Shaded}
\begin{Highlighting}[]
\FunctionTok{print}\NormalTok{(}\StringTok{"Daily Calories"}\NormalTok{)}
\end{Highlighting}
\end{Shaded}

\begin{verbatim}
## [1] "Daily Calories"
\end{verbatim}

\begin{Shaded}
\begin{Highlighting}[]
\FunctionTok{n\_unique}\NormalTok{(clean\_daily\_calories}\SpecialCharTok{$}\NormalTok{id)}
\end{Highlighting}
\end{Shaded}

\begin{verbatim}
## [1] 33
\end{verbatim}

\begin{Shaded}
\begin{Highlighting}[]
\FunctionTok{print}\NormalTok{(}\StringTok{"Daily Intensities"}\NormalTok{)}
\end{Highlighting}
\end{Shaded}

\begin{verbatim}
## [1] "Daily Intensities"
\end{verbatim}

\begin{Shaded}
\begin{Highlighting}[]
\FunctionTok{n\_unique}\NormalTok{(clean\_daily\_intensities}\SpecialCharTok{$}\NormalTok{id)}
\end{Highlighting}
\end{Shaded}

\begin{verbatim}
## [1] 33
\end{verbatim}

\begin{Shaded}
\begin{Highlighting}[]
\FunctionTok{print}\NormalTok{(}\StringTok{"Daily Steps"}\NormalTok{)}
\end{Highlighting}
\end{Shaded}

\begin{verbatim}
## [1] "Daily Steps"
\end{verbatim}

\begin{Shaded}
\begin{Highlighting}[]
\FunctionTok{n\_unique}\NormalTok{(clean\_daily\_steps}\SpecialCharTok{$}\NormalTok{id)}
\end{Highlighting}
\end{Shaded}

\begin{verbatim}
## [1] 33
\end{verbatim}

\begin{Shaded}
\begin{Highlighting}[]
\FunctionTok{print}\NormalTok{(}\StringTok{"Daily Sleep"}\NormalTok{)}
\end{Highlighting}
\end{Shaded}

\begin{verbatim}
## [1] "Daily Sleep"
\end{verbatim}

\begin{Shaded}
\begin{Highlighting}[]
\FunctionTok{n\_unique}\NormalTok{(clean\_daily\_sleep}\SpecialCharTok{$}\NormalTok{id)}
\end{Highlighting}
\end{Shaded}

\begin{verbatim}
## [1] 24
\end{verbatim}

\begin{Shaded}
\begin{Highlighting}[]
\FunctionTok{print}\NormalTok{(}\StringTok{"Daily Weight Log Info"}\NormalTok{)}
\end{Highlighting}
\end{Shaded}

\begin{verbatim}
## [1] "Daily Weight Log Info"
\end{verbatim}

\begin{Shaded}
\begin{Highlighting}[]
\FunctionTok{n\_unique}\NormalTok{(clean\_weight\_log\_info}\SpecialCharTok{$}\NormalTok{id)}
\end{Highlighting}
\end{Shaded}

\begin{verbatim}
## [1] 8
\end{verbatim}

\hypertarget{hourly-dataframes-2}{%
\subparagraph{Hourly DataFrames}\label{hourly-dataframes-2}}

\begin{Shaded}
\begin{Highlighting}[]
\CommentTok{\# Unique participant IDs: Hourly DataFrames}
\FunctionTok{print}\NormalTok{(}\StringTok{"Hourly Calories"}\NormalTok{)}
\end{Highlighting}
\end{Shaded}

\begin{verbatim}
## [1] "Hourly Calories"
\end{verbatim}

\begin{Shaded}
\begin{Highlighting}[]
\FunctionTok{n\_unique}\NormalTok{(clean\_hourly\_calories}\SpecialCharTok{$}\NormalTok{id)}
\end{Highlighting}
\end{Shaded}

\begin{verbatim}
## [1] 33
\end{verbatim}

\begin{Shaded}
\begin{Highlighting}[]
\FunctionTok{print}\NormalTok{(}\StringTok{"Hourly Intensities"}\NormalTok{)}
\end{Highlighting}
\end{Shaded}

\begin{verbatim}
## [1] "Hourly Intensities"
\end{verbatim}

\begin{Shaded}
\begin{Highlighting}[]
\FunctionTok{n\_unique}\NormalTok{(clean\_hourly\_intensities}\SpecialCharTok{$}\NormalTok{id)}
\end{Highlighting}
\end{Shaded}

\begin{verbatim}
## [1] 33
\end{verbatim}

\begin{Shaded}
\begin{Highlighting}[]
\FunctionTok{print}\NormalTok{(}\StringTok{"Hourly Steps"}\NormalTok{)}
\end{Highlighting}
\end{Shaded}

\begin{verbatim}
## [1] "Hourly Steps"
\end{verbatim}

\begin{Shaded}
\begin{Highlighting}[]
\FunctionTok{n\_unique}\NormalTok{(clean\_hourly\_steps}\SpecialCharTok{$}\NormalTok{id)}
\end{Highlighting}
\end{Shaded}

\begin{verbatim}
## [1] 33
\end{verbatim}

\hypertarget{minute-dataframes-2}{%
\subparagraph{Minute DataFrames}\label{minute-dataframes-2}}

\begin{Shaded}
\begin{Highlighting}[]
\CommentTok{\# Unique participant IDs: Minute DataFrames}
\FunctionTok{print}\NormalTok{(}\StringTok{"Minute Calories"}\NormalTok{)}
\end{Highlighting}
\end{Shaded}

\begin{verbatim}
## [1] "Minute Calories"
\end{verbatim}

\begin{Shaded}
\begin{Highlighting}[]
\FunctionTok{n\_unique}\NormalTok{(clean\_minute\_calories}\SpecialCharTok{$}\NormalTok{id)}
\end{Highlighting}
\end{Shaded}

\begin{verbatim}
## [1] 33
\end{verbatim}

\begin{Shaded}
\begin{Highlighting}[]
\FunctionTok{print}\NormalTok{(}\StringTok{"Minute Intensities"}\NormalTok{)}
\end{Highlighting}
\end{Shaded}

\begin{verbatim}
## [1] "Minute Intensities"
\end{verbatim}

\begin{Shaded}
\begin{Highlighting}[]
\FunctionTok{n\_unique}\NormalTok{(clean\_minute\_intensities}\SpecialCharTok{$}\NormalTok{id)}
\end{Highlighting}
\end{Shaded}

\begin{verbatim}
## [1] 33
\end{verbatim}

\begin{Shaded}
\begin{Highlighting}[]
\FunctionTok{print}\NormalTok{(}\StringTok{"Minute METs"}\NormalTok{)}
\end{Highlighting}
\end{Shaded}

\begin{verbatim}
## [1] "Minute METs"
\end{verbatim}

\begin{Shaded}
\begin{Highlighting}[]
\FunctionTok{n\_unique}\NormalTok{(clean\_minute\_METs}\SpecialCharTok{$}\NormalTok{id)}
\end{Highlighting}
\end{Shaded}

\begin{verbatim}
## [1] 33
\end{verbatim}

\begin{Shaded}
\begin{Highlighting}[]
\FunctionTok{print}\NormalTok{(}\StringTok{"Minute Sleep"}\NormalTok{)}
\end{Highlighting}
\end{Shaded}

\begin{verbatim}
## [1] "Minute Sleep"
\end{verbatim}

\begin{Shaded}
\begin{Highlighting}[]
\FunctionTok{n\_unique}\NormalTok{(clean\_minute\_sleep}\SpecialCharTok{$}\NormalTok{id)}
\end{Highlighting}
\end{Shaded}

\begin{verbatim}
## [1] 24
\end{verbatim}

\begin{Shaded}
\begin{Highlighting}[]
\FunctionTok{print}\NormalTok{(}\StringTok{"Minute Steps"}\NormalTok{)}
\end{Highlighting}
\end{Shaded}

\begin{verbatim}
## [1] "Minute Steps"
\end{verbatim}

\begin{Shaded}
\begin{Highlighting}[]
\FunctionTok{n\_unique}\NormalTok{(clean\_minute\_steps}\SpecialCharTok{$}\NormalTok{id)}
\end{Highlighting}
\end{Shaded}

\begin{verbatim}
## [1] 33
\end{verbatim}

\hypertarget{second-dataframes-1}{%
\subparagraph{Second DataFrames}\label{second-dataframes-1}}

\begin{Shaded}
\begin{Highlighting}[]
\CommentTok{\# Unique participant IDs: Second DataFrame}
\FunctionTok{print}\NormalTok{(}\StringTok{"Second Heartrate"}\NormalTok{)}
\end{Highlighting}
\end{Shaded}

\begin{verbatim}
## [1] "Second Heartrate"
\end{verbatim}

\begin{Shaded}
\begin{Highlighting}[]
\FunctionTok{n\_unique}\NormalTok{(clean\_heartrate\_seconds}\SpecialCharTok{$}\NormalTok{id)}
\end{Highlighting}
\end{Shaded}

\begin{verbatim}
## [1] 14
\end{verbatim}

\hypertarget{observation}{%
\subparagraph{Observation}\label{observation}}

The examination of the data reveals a significant inconsistency with the
expected number of participant IDs. While the data description specifies
each dataset should contain 30 unique IDs, the majority of the datasets
actually contain 33 distinct IDs, indicating an excess of three
participants. Additionally, there are notable exceptions in the sleep,
weight, and heart rate datasets, which contain only 24, 8, and 14 unique
IDs respectively. This variation suggests these datasets are incomplete.

The observed discrepancies could be attributed to differences in how
data is collected and recorded: - \textbf{Activity data}, which
generally show higher participant counts, are likely recorded
automatically by the device through sensors detecting movement. -
\textbf{Weight data}, which significantly lacks participant
representation, require manual input by the users into their Fitbit
devices, suggesting that many participants did not consistently enter
this information. - \textbf{Heart rate data}, also showing lower numbers
of participants, might necessitate active user engagement to initiate
tracking, which may not always be consistently performed by all users.

This analysis indicates that the completeness of each dataset is
dependent on both the nature of the data collection method and
participant compliance with using the device's features.

\hypertarget{actions-taken}{%
\subparagraph{Actions Taken}\label{actions-taken}}

\begin{itemize}
\tightlist
\item
  Given the limited participant counts in the weight (8) and heart
  datasets (14), these datasets will be excluded from further analysis
  as the sample size is not significant enough to make any conclusions.
\end{itemize}

\hypertarget{number-of-active-days}{%
\paragraph{Number of Active Days:}\label{number-of-active-days}}

Each dataset is expected to span 31 days. To confirm this, the number of
active days in each dataset will be assessed using the n\_unique()
function.

\hypertarget{daily-dataframes-3}{%
\subparagraph{Daily DataFrames}\label{daily-dataframes-3}}

\begin{Shaded}
\begin{Highlighting}[]
\CommentTok{\# Unique number of active days: Daily DataFrames}
\FunctionTok{print}\NormalTok{(}\StringTok{"Daily Activity"}\NormalTok{)}
\end{Highlighting}
\end{Shaded}

\begin{verbatim}
## [1] "Daily Activity"
\end{verbatim}

\begin{Shaded}
\begin{Highlighting}[]
\FunctionTok{n\_unique}\NormalTok{(clean\_daily\_activity}\SpecialCharTok{$}\NormalTok{activity\_date\_ymd)}
\end{Highlighting}
\end{Shaded}

\begin{verbatim}
## [1] 31
\end{verbatim}

\begin{Shaded}
\begin{Highlighting}[]
\FunctionTok{print}\NormalTok{(}\StringTok{"Daily Calories"}\NormalTok{)}
\end{Highlighting}
\end{Shaded}

\begin{verbatim}
## [1] "Daily Calories"
\end{verbatim}

\begin{Shaded}
\begin{Highlighting}[]
\FunctionTok{n\_unique}\NormalTok{(clean\_daily\_calories}\SpecialCharTok{$}\NormalTok{activity\_date\_ymd)}
\end{Highlighting}
\end{Shaded}

\begin{verbatim}
## [1] 31
\end{verbatim}

\begin{Shaded}
\begin{Highlighting}[]
\FunctionTok{print}\NormalTok{(}\StringTok{"Daily Intensities"}\NormalTok{)}
\end{Highlighting}
\end{Shaded}

\begin{verbatim}
## [1] "Daily Intensities"
\end{verbatim}

\begin{Shaded}
\begin{Highlighting}[]
\FunctionTok{n\_unique}\NormalTok{(clean\_daily\_intensities}\SpecialCharTok{$}\NormalTok{activity\_date\_ymd)}
\end{Highlighting}
\end{Shaded}

\begin{verbatim}
## [1] 31
\end{verbatim}

\begin{Shaded}
\begin{Highlighting}[]
\FunctionTok{print}\NormalTok{(}\StringTok{"Daily Steps"}\NormalTok{)}
\end{Highlighting}
\end{Shaded}

\begin{verbatim}
## [1] "Daily Steps"
\end{verbatim}

\begin{Shaded}
\begin{Highlighting}[]
\FunctionTok{n\_unique}\NormalTok{(clean\_daily\_steps}\SpecialCharTok{$}\NormalTok{activity\_date\_ymd)}
\end{Highlighting}
\end{Shaded}

\begin{verbatim}
## [1] 31
\end{verbatim}

\begin{Shaded}
\begin{Highlighting}[]
\FunctionTok{print}\NormalTok{(}\StringTok{"Daily Sleep"}\NormalTok{)}
\end{Highlighting}
\end{Shaded}

\begin{verbatim}
## [1] "Daily Sleep"
\end{verbatim}

\begin{Shaded}
\begin{Highlighting}[]
\FunctionTok{n\_unique}\NormalTok{(clean\_daily\_sleep}\SpecialCharTok{$}\NormalTok{activity\_date\_ymd)}
\end{Highlighting}
\end{Shaded}

\begin{verbatim}
## [1] 31
\end{verbatim}

\hypertarget{hourly-dataframes-3}{%
\subparagraph{Hourly DataFrames}\label{hourly-dataframes-3}}

\begin{Shaded}
\begin{Highlighting}[]
\CommentTok{\# Unique number of active days: Hourly DataFrames}
\FunctionTok{print}\NormalTok{(}\StringTok{"Hourly Calories"}\NormalTok{)}
\end{Highlighting}
\end{Shaded}

\begin{verbatim}
## [1] "Hourly Calories"
\end{verbatim}

\begin{Shaded}
\begin{Highlighting}[]
\FunctionTok{n\_unique}\NormalTok{(}\FunctionTok{date}\NormalTok{(clean\_hourly\_calories}\SpecialCharTok{$}\NormalTok{activity\_hour\_ymdhms))}
\end{Highlighting}
\end{Shaded}

\begin{verbatim}
## [1] 31
\end{verbatim}

\begin{Shaded}
\begin{Highlighting}[]
\FunctionTok{print}\NormalTok{(}\StringTok{"Hourly Intensities"}\NormalTok{)}
\end{Highlighting}
\end{Shaded}

\begin{verbatim}
## [1] "Hourly Intensities"
\end{verbatim}

\begin{Shaded}
\begin{Highlighting}[]
\FunctionTok{n\_unique}\NormalTok{(}\FunctionTok{date}\NormalTok{(clean\_hourly\_intensities}\SpecialCharTok{$}\NormalTok{activity\_hour\_ymdhms))}
\end{Highlighting}
\end{Shaded}

\begin{verbatim}
## [1] 31
\end{verbatim}

\begin{Shaded}
\begin{Highlighting}[]
\FunctionTok{print}\NormalTok{(}\StringTok{"Hourly Steps"}\NormalTok{)}
\end{Highlighting}
\end{Shaded}

\begin{verbatim}
## [1] "Hourly Steps"
\end{verbatim}

\begin{Shaded}
\begin{Highlighting}[]
\FunctionTok{n\_unique}\NormalTok{(}\FunctionTok{date}\NormalTok{(clean\_hourly\_steps}\SpecialCharTok{$}\NormalTok{activity\_hour\_ymdhms))}
\end{Highlighting}
\end{Shaded}

\begin{verbatim}
## [1] 31
\end{verbatim}

\hypertarget{minute-dataframes-3}{%
\subparagraph{Minute DataFrames}\label{minute-dataframes-3}}

\begin{Shaded}
\begin{Highlighting}[]
\CommentTok{\# Unique number of active days: Minute DataFrames}
\FunctionTok{print}\NormalTok{(}\StringTok{"Minute Calories"}\NormalTok{)}
\end{Highlighting}
\end{Shaded}

\begin{verbatim}
## [1] "Minute Calories"
\end{verbatim}

\begin{Shaded}
\begin{Highlighting}[]
\FunctionTok{n\_unique}\NormalTok{(}\FunctionTok{date}\NormalTok{(clean\_minute\_calories}\SpecialCharTok{$}\NormalTok{activity\_minute\_ymdhms))}
\end{Highlighting}
\end{Shaded}

\begin{verbatim}
## [1] 31
\end{verbatim}

\begin{Shaded}
\begin{Highlighting}[]
\FunctionTok{print}\NormalTok{(}\StringTok{"Minute Intensities"}\NormalTok{)}
\end{Highlighting}
\end{Shaded}

\begin{verbatim}
## [1] "Minute Intensities"
\end{verbatim}

\begin{Shaded}
\begin{Highlighting}[]
\FunctionTok{n\_unique}\NormalTok{(}\FunctionTok{date}\NormalTok{(clean\_minute\_intensities}\SpecialCharTok{$}\NormalTok{activity\_minute\_ymdhms))}
\end{Highlighting}
\end{Shaded}

\begin{verbatim}
## [1] 31
\end{verbatim}

\begin{Shaded}
\begin{Highlighting}[]
\FunctionTok{print}\NormalTok{(}\StringTok{"Minute METs"}\NormalTok{)}
\end{Highlighting}
\end{Shaded}

\begin{verbatim}
## [1] "Minute METs"
\end{verbatim}

\begin{Shaded}
\begin{Highlighting}[]
\FunctionTok{n\_unique}\NormalTok{(}\FunctionTok{date}\NormalTok{(clean\_minute\_METs}\SpecialCharTok{$}\NormalTok{activity\_minute\_ymdhms))}
\end{Highlighting}
\end{Shaded}

\begin{verbatim}
## [1] 31
\end{verbatim}

\begin{Shaded}
\begin{Highlighting}[]
\FunctionTok{print}\NormalTok{(}\StringTok{"Minute Sleep"}\NormalTok{)}
\end{Highlighting}
\end{Shaded}

\begin{verbatim}
## [1] "Minute Sleep"
\end{verbatim}

\begin{Shaded}
\begin{Highlighting}[]
\FunctionTok{n\_unique}\NormalTok{(}\FunctionTok{date}\NormalTok{(clean\_minute\_sleep}\SpecialCharTok{$}\NormalTok{activity\_minute\_ymdhms))}
\end{Highlighting}
\end{Shaded}

\begin{verbatim}
## [1] 32
\end{verbatim}

\begin{Shaded}
\begin{Highlighting}[]
\FunctionTok{print}\NormalTok{(}\StringTok{"Minute Steps"}\NormalTok{)}
\end{Highlighting}
\end{Shaded}

\begin{verbatim}
## [1] "Minute Steps"
\end{verbatim}

\begin{Shaded}
\begin{Highlighting}[]
\FunctionTok{n\_unique}\NormalTok{(}\FunctionTok{date}\NormalTok{(clean\_minute\_steps}\SpecialCharTok{$}\NormalTok{activity\_minute\_ymdhms))}
\end{Highlighting}
\end{Shaded}

\begin{verbatim}
## [1] 31
\end{verbatim}

\hypertarget{observation-1}{%
\subparagraph{Observation}\label{observation-1}}

Most datasets exhibit 31 active days, aligning with the expected
duration of the research period. However, the ``minute\_sleep'' dataset
displays 32 active days. The research was supposed to start on
04/12/2016 and end 05/12/2016 (= 31 days). However, some participants
went to bed before 12am on 04/11/2016; therefore, that date was also
included in the data, amounting to 32 days.

\hypertarget{participants-logged-activity}{%
\paragraph{3.4.4 Participants' Logged
Activity}\label{participants-logged-activity}}

To ensure participant engagement throughout the entire 31-day research
period, the number of logged activities will be examined.

\begin{Shaded}
\begin{Highlighting}[]
\NormalTok{clean\_daily\_activity }\SpecialCharTok{\%\textgreater{}\%} 
  \FunctionTok{group\_by}\NormalTok{(id) }\SpecialCharTok{\%\textgreater{}\%} 
  \FunctionTok{count}\NormalTok{() }\SpecialCharTok{\%\textgreater{}\%} 
  \FunctionTok{arrange}\NormalTok{(n) }\SpecialCharTok{\%\textgreater{}\%} 
  \FunctionTok{tibble}\NormalTok{() }\SpecialCharTok{\%\textgreater{}\%} 
  \FunctionTok{print}\NormalTok{(}\AttributeTok{n=}\DecValTok{33}\NormalTok{)}
\end{Highlighting}
\end{Shaded}

\begin{verbatim}
## # A tibble: 33 x 2
##            id     n
##         <dbl> <int>
##  1 4057192912     4
##  2 2347167796    18
##  3 8253242879    19
##  4 3372868164    20
##  5 6775888955    26
##  6 7007744171    26
##  7 6117666160    28
##  8 6290855005    29
##  9 8792009665    29
## 10 1644430081    30
## 11 3977333714    30
## 12 5577150313    30
## 13 1503960366    31
## 14 1624580081    31
## 15 1844505072    31
## 16 1927972279    31
## 17 2022484408    31
## 18 2026352035    31
## 19 2320127002    31
## 20 2873212765    31
## 21 4020332650    31
## 22 4319703577    31
## 23 4388161847    31
## 24 4445114986    31
## 25 4558609924    31
## 26 4702921684    31
## 27 5553957443    31
## 28 6962181067    31
## 29 7086361926    31
## 30 8053475328    31
## 31 8378563200    31
## 32 8583815059    31
## 33 8877689391    31
\end{verbatim}

\hypertarget{observation-2}{%
\subparagraph{Observation}\label{observation-2}}

Participant with ID\# 4057192912 has only 4 logged activities, which is
significantly low. Although considering removal, given the already
limited participant pool, the participant's data will be kept.

\hypertarget{missing-values}{%
\paragraph{Missing Values}\label{missing-values}}

The number of missing values in each dataset will be examined using the
\texttt{sum(is.na(x))} and \texttt{which(is.na(x))} functions.

\hypertarget{daily-dataframes-4}{%
\subparagraph{Daily DataFrames}\label{daily-dataframes-4}}

\begin{Shaded}
\begin{Highlighting}[]
\CommentTok{\# Checking for missing values: Daily DataFrames}
\FunctionTok{print}\NormalTok{ (}\StringTok{"{-}{-}{-}{-}{-}Daily Activity{-}{-}{-}{-}{-}"}\NormalTok{)}
\end{Highlighting}
\end{Shaded}

\begin{verbatim}
## [1] "-----Daily Activity-----"
\end{verbatim}

\begin{Shaded}
\begin{Highlighting}[]
\ControlFlowTok{if}\NormalTok{ (}\FunctionTok{sum}\NormalTok{(}\FunctionTok{is.na}\NormalTok{(clean\_daily\_activity)) }\SpecialCharTok{!=}\DecValTok{0}\NormalTok{) \{}
  \FunctionTok{print}\NormalTok{(}\StringTok{"Position of missing values {-} "}\NormalTok{)}
  \FunctionTok{which}\NormalTok{(}\FunctionTok{is.na}\NormalTok{(clean\_daily\_activity))}
\NormalTok{\} }\ControlFlowTok{else}\NormalTok{ \{}
  \FunctionTok{print}\NormalTok{(}\StringTok{"No missing values"}\NormalTok{)}
\NormalTok{\}}
\end{Highlighting}
\end{Shaded}

\begin{verbatim}
## [1] "No missing values"
\end{verbatim}

\begin{Shaded}
\begin{Highlighting}[]
\FunctionTok{print}\NormalTok{(}\StringTok{"{-}{-}{-}{-}{-}Daily Calories{-}{-}{-}{-}{-}"}\NormalTok{)}
\end{Highlighting}
\end{Shaded}

\begin{verbatim}
## [1] "-----Daily Calories-----"
\end{verbatim}

\begin{Shaded}
\begin{Highlighting}[]
\ControlFlowTok{if}\NormalTok{ (}\FunctionTok{sum}\NormalTok{(}\FunctionTok{is.na}\NormalTok{(clean\_daily\_calories)) }\SpecialCharTok{!=}\DecValTok{0}\NormalTok{) \{}
  \FunctionTok{print}\NormalTok{(}\StringTok{"Position of missing values {-} "}\NormalTok{)}
  \FunctionTok{which}\NormalTok{(}\FunctionTok{is.na}\NormalTok{(clean\_daily\_calories))}
\NormalTok{\} }\ControlFlowTok{else}\NormalTok{ \{}
  \FunctionTok{print}\NormalTok{(}\StringTok{"No missing values"}\NormalTok{)}
\NormalTok{\}}
\end{Highlighting}
\end{Shaded}

\begin{verbatim}
## [1] "No missing values"
\end{verbatim}

\begin{Shaded}
\begin{Highlighting}[]
\FunctionTok{print}\NormalTok{(}\StringTok{"{-}{-}{-}{-}{-}Daily Intensities{-}{-}{-}{-}{-}"}\NormalTok{)}
\end{Highlighting}
\end{Shaded}

\begin{verbatim}
## [1] "-----Daily Intensities-----"
\end{verbatim}

\begin{Shaded}
\begin{Highlighting}[]
\ControlFlowTok{if}\NormalTok{ (}\FunctionTok{sum}\NormalTok{(}\FunctionTok{is.na}\NormalTok{(clean\_daily\_intensities)) }\SpecialCharTok{!=}\DecValTok{0}\NormalTok{) \{}
  \FunctionTok{print}\NormalTok{(}\StringTok{"Position of missing values {-} "}\NormalTok{)}
  \FunctionTok{which}\NormalTok{(}\FunctionTok{is.na}\NormalTok{(clean\_daily\_intensities))}
\NormalTok{\} }\ControlFlowTok{else}\NormalTok{ \{}
  \FunctionTok{print}\NormalTok{(}\StringTok{"No missing values"}\NormalTok{)}
\NormalTok{\}}
\end{Highlighting}
\end{Shaded}

\begin{verbatim}
## [1] "No missing values"
\end{verbatim}

\begin{Shaded}
\begin{Highlighting}[]
\FunctionTok{print}\NormalTok{(}\StringTok{"{-}{-}{-}{-}{-}Daily Steps{-}{-}{-}{-}{-}"}\NormalTok{)}
\end{Highlighting}
\end{Shaded}

\begin{verbatim}
## [1] "-----Daily Steps-----"
\end{verbatim}

\begin{Shaded}
\begin{Highlighting}[]
\ControlFlowTok{if}\NormalTok{ (}\FunctionTok{sum}\NormalTok{(}\FunctionTok{is.na}\NormalTok{(clean\_daily\_steps)) }\SpecialCharTok{!=}\DecValTok{0}\NormalTok{) \{}
  \FunctionTok{print}\NormalTok{(}\StringTok{"Position of missing values {-} "}\NormalTok{)}
  \FunctionTok{which}\NormalTok{(}\FunctionTok{is.na}\NormalTok{(clean\_daily\_steps))}
\NormalTok{\} }\ControlFlowTok{else}\NormalTok{ \{}
  \FunctionTok{print}\NormalTok{(}\StringTok{"No missing values"}\NormalTok{)}
\NormalTok{\}}
\end{Highlighting}
\end{Shaded}

\begin{verbatim}
## [1] "No missing values"
\end{verbatim}

\begin{Shaded}
\begin{Highlighting}[]
\FunctionTok{print}\NormalTok{(}\StringTok{"{-}{-}{-}{-}{-}Daily Sleep{-}{-}{-}{-}{-}"}\NormalTok{)}
\end{Highlighting}
\end{Shaded}

\begin{verbatim}
## [1] "-----Daily Sleep-----"
\end{verbatim}

\begin{Shaded}
\begin{Highlighting}[]
\ControlFlowTok{if}\NormalTok{ (}\FunctionTok{sum}\NormalTok{(}\FunctionTok{is.na}\NormalTok{(clean\_daily\_sleep)) }\SpecialCharTok{!=}\DecValTok{0}\NormalTok{) \{}
  \FunctionTok{print}\NormalTok{(}\StringTok{"Position of missing values {-} "}\NormalTok{)}
  \FunctionTok{which}\NormalTok{(}\FunctionTok{is.na}\NormalTok{(clean\_daily\_sleep))}
\NormalTok{\} }\ControlFlowTok{else}\NormalTok{ \{}
  \FunctionTok{print}\NormalTok{(}\StringTok{"No missing values"}\NormalTok{)}
\NormalTok{\}}
\end{Highlighting}
\end{Shaded}

\begin{verbatim}
## [1] "No missing values"
\end{verbatim}

\hypertarget{hourly-dataframes-4}{%
\subparagraph{Hourly DataFrames}\label{hourly-dataframes-4}}

\begin{Shaded}
\begin{Highlighting}[]
\CommentTok{\# Checking for missing values: Hourly DataFrames}
\FunctionTok{print}\NormalTok{(}\StringTok{"{-}{-}{-}{-}{-}Hourly Calories{-}{-}{-}{-}{-}"}\NormalTok{)}
\end{Highlighting}
\end{Shaded}

\begin{verbatim}
## [1] "-----Hourly Calories-----"
\end{verbatim}

\begin{Shaded}
\begin{Highlighting}[]
\ControlFlowTok{if}\NormalTok{ (}\FunctionTok{sum}\NormalTok{(}\FunctionTok{is.na}\NormalTok{(clean\_hourly\_calories)) }\SpecialCharTok{!=}\DecValTok{0}\NormalTok{) \{}
  \FunctionTok{print}\NormalTok{(}\StringTok{"Position of missing values {-} "}\NormalTok{)}
  \FunctionTok{which}\NormalTok{(}\FunctionTok{is.na}\NormalTok{(clean\_hourly\_calories))}
\NormalTok{\} }\ControlFlowTok{else}\NormalTok{ \{}
  \FunctionTok{print}\NormalTok{(}\StringTok{"No missing values"}\NormalTok{)}
\NormalTok{\}}
\end{Highlighting}
\end{Shaded}

\begin{verbatim}
## [1] "No missing values"
\end{verbatim}

\begin{Shaded}
\begin{Highlighting}[]
\FunctionTok{print}\NormalTok{(}\StringTok{"{-}{-}{-}{-}{-}Hourly Intensities{-}{-}{-}{-}{-}"}\NormalTok{)}
\end{Highlighting}
\end{Shaded}

\begin{verbatim}
## [1] "-----Hourly Intensities-----"
\end{verbatim}

\begin{Shaded}
\begin{Highlighting}[]
\ControlFlowTok{if}\NormalTok{ (}\FunctionTok{sum}\NormalTok{(}\FunctionTok{is.na}\NormalTok{(clean\_hourly\_intensities)) }\SpecialCharTok{!=}\DecValTok{0}\NormalTok{) \{}
  \FunctionTok{print}\NormalTok{(}\StringTok{"Position of missing values {-} "}\NormalTok{)}
  \FunctionTok{which}\NormalTok{(}\FunctionTok{is.na}\NormalTok{(clean\_hourly\_intensities))}
\NormalTok{\} }\ControlFlowTok{else}\NormalTok{ \{}
  \FunctionTok{print}\NormalTok{(}\StringTok{"No missing values"}\NormalTok{)}
\NormalTok{\}}
\end{Highlighting}
\end{Shaded}

\begin{verbatim}
## [1] "No missing values"
\end{verbatim}

\begin{Shaded}
\begin{Highlighting}[]
\FunctionTok{print}\NormalTok{(}\StringTok{"{-}{-}{-}{-}{-}Hourly Steps{-}{-}{-}{-}{-}"}\NormalTok{)}
\end{Highlighting}
\end{Shaded}

\begin{verbatim}
## [1] "-----Hourly Steps-----"
\end{verbatim}

\begin{Shaded}
\begin{Highlighting}[]
\ControlFlowTok{if}\NormalTok{ (}\FunctionTok{sum}\NormalTok{(}\FunctionTok{is.na}\NormalTok{(clean\_hourly\_steps)) }\SpecialCharTok{!=}\DecValTok{0}\NormalTok{) \{}
  \FunctionTok{print}\NormalTok{(}\StringTok{"Position of missing values {-} "}\NormalTok{)}
  \FunctionTok{which}\NormalTok{(}\FunctionTok{is.na}\NormalTok{(clean\_hourly\_steps))}
\NormalTok{\} }\ControlFlowTok{else}\NormalTok{ \{}
  \FunctionTok{print}\NormalTok{(}\StringTok{"No missing values"}\NormalTok{)}
\NormalTok{\}}
\end{Highlighting}
\end{Shaded}

\begin{verbatim}
## [1] "No missing values"
\end{verbatim}

\hypertarget{minute-dataframes-4}{%
\subparagraph{Minute DataFrames}\label{minute-dataframes-4}}

\begin{Shaded}
\begin{Highlighting}[]
\CommentTok{\# Checking for missing values: Minute DataFrames}
\FunctionTok{print}\NormalTok{(}\StringTok{"{-}{-}{-}{-}{-}Minute Calories{-}{-}{-}{-}{-}"}\NormalTok{)}
\end{Highlighting}
\end{Shaded}

\begin{verbatim}
## [1] "-----Minute Calories-----"
\end{verbatim}

\begin{Shaded}
\begin{Highlighting}[]
\ControlFlowTok{if}\NormalTok{ (}\FunctionTok{sum}\NormalTok{(}\FunctionTok{is.na}\NormalTok{(clean\_minute\_calories)) }\SpecialCharTok{!=}\DecValTok{0}\NormalTok{) \{}
  \FunctionTok{print}\NormalTok{(}\StringTok{"Position of missing values {-} "}\NormalTok{)}
  \FunctionTok{which}\NormalTok{(}\FunctionTok{is.na}\NormalTok{(clean\_minute\_calories))}
\NormalTok{\} }\ControlFlowTok{else}\NormalTok{ \{}
  \FunctionTok{print}\NormalTok{(}\StringTok{"No missing values"}\NormalTok{)}
\NormalTok{\}}
\end{Highlighting}
\end{Shaded}

\begin{verbatim}
## [1] "No missing values"
\end{verbatim}

\begin{Shaded}
\begin{Highlighting}[]
\FunctionTok{print}\NormalTok{(}\StringTok{"{-}{-}{-}{-}{-}Minute Intensities{-}{-}{-}{-}{-}"}\NormalTok{)}
\end{Highlighting}
\end{Shaded}

\begin{verbatim}
## [1] "-----Minute Intensities-----"
\end{verbatim}

\begin{Shaded}
\begin{Highlighting}[]
\ControlFlowTok{if}\NormalTok{ (}\FunctionTok{sum}\NormalTok{(}\FunctionTok{is.na}\NormalTok{(clean\_minute\_intensities)) }\SpecialCharTok{!=}\DecValTok{0}\NormalTok{) \{}
  \FunctionTok{print}\NormalTok{(}\StringTok{"Position of missing values {-} "}\NormalTok{)}
  \FunctionTok{which}\NormalTok{(}\FunctionTok{is.na}\NormalTok{(clean\_minute\_intensities))}
\NormalTok{\} }\ControlFlowTok{else}\NormalTok{ \{}
  \FunctionTok{print}\NormalTok{(}\StringTok{"No missing values"}\NormalTok{)}
\NormalTok{\}}
\end{Highlighting}
\end{Shaded}

\begin{verbatim}
## [1] "No missing values"
\end{verbatim}

\begin{Shaded}
\begin{Highlighting}[]
\FunctionTok{print}\NormalTok{(}\StringTok{"{-}{-}{-}{-}{-}Minute METs{-}{-}{-}{-}{-}"}\NormalTok{)}
\end{Highlighting}
\end{Shaded}

\begin{verbatim}
## [1] "-----Minute METs-----"
\end{verbatim}

\begin{Shaded}
\begin{Highlighting}[]
\ControlFlowTok{if}\NormalTok{ (}\FunctionTok{sum}\NormalTok{(}\FunctionTok{is.na}\NormalTok{(clean\_minute\_METs)) }\SpecialCharTok{!=}\DecValTok{0}\NormalTok{) \{}
  \FunctionTok{print}\NormalTok{(}\StringTok{"Position of missing values {-} "}\NormalTok{)}
  \FunctionTok{which}\NormalTok{(}\FunctionTok{is.na}\NormalTok{(clean\_minute\_METs))}
\NormalTok{\} }\ControlFlowTok{else}\NormalTok{ \{}
  \FunctionTok{print}\NormalTok{(}\StringTok{"No missing values"}\NormalTok{)}
\NormalTok{\}}
\end{Highlighting}
\end{Shaded}

\begin{verbatim}
## [1] "No missing values"
\end{verbatim}

\begin{Shaded}
\begin{Highlighting}[]
\FunctionTok{print}\NormalTok{(}\StringTok{"{-}{-}{-}{-}{-}Minute Sleep{-}{-}{-}{-}{-}"}\NormalTok{)}
\end{Highlighting}
\end{Shaded}

\begin{verbatim}
## [1] "-----Minute Sleep-----"
\end{verbatim}

\begin{Shaded}
\begin{Highlighting}[]
\ControlFlowTok{if}\NormalTok{ (}\FunctionTok{sum}\NormalTok{(}\FunctionTok{is.na}\NormalTok{(clean\_minute\_sleep)) }\SpecialCharTok{!=}\DecValTok{0}\NormalTok{) \{}
  \FunctionTok{print}\NormalTok{(}\StringTok{"Position of missing values {-} "}\NormalTok{)}
  \FunctionTok{which}\NormalTok{(}\FunctionTok{is.na}\NormalTok{(clean\_minute\_sleep))}
\NormalTok{\} }\ControlFlowTok{else}\NormalTok{ \{}
  \FunctionTok{print}\NormalTok{(}\StringTok{"No missing values"}\NormalTok{)}
\NormalTok{\}}
\end{Highlighting}
\end{Shaded}

\begin{verbatim}
## [1] "No missing values"
\end{verbatim}

\begin{Shaded}
\begin{Highlighting}[]
\FunctionTok{print}\NormalTok{(}\StringTok{"{-}{-}{-}{-}{-}Minute Steps{-}{-}{-}{-}{-}"}\NormalTok{)}
\end{Highlighting}
\end{Shaded}

\begin{verbatim}
## [1] "-----Minute Steps-----"
\end{verbatim}

\begin{Shaded}
\begin{Highlighting}[]
\ControlFlowTok{if}\NormalTok{ (}\FunctionTok{sum}\NormalTok{(}\FunctionTok{is.na}\NormalTok{(clean\_minute\_steps)) }\SpecialCharTok{!=}\DecValTok{0}\NormalTok{) \{}
  \FunctionTok{print}\NormalTok{(}\StringTok{"Position of missing values {-} "}\NormalTok{)}
  \FunctionTok{which}\NormalTok{(}\FunctionTok{is.na}\NormalTok{(clean\_minute\_steps))}
\NormalTok{\} }\ControlFlowTok{else}\NormalTok{ \{}
  \FunctionTok{print}\NormalTok{(}\StringTok{"No missing values"}\NormalTok{)}
\NormalTok{\}}
\end{Highlighting}
\end{Shaded}

\begin{verbatim}
## [1] "No missing values"
\end{verbatim}

\hypertarget{observation-3}{%
\subparagraph{Observation}\label{observation-3}}

There are no missing values.

\hypertarget{removing-duplicates}{%
\paragraph{3.4.5 Removing Duplicates}\label{removing-duplicates}}

Duplicate data can cause inaccurate reporting. To find out if there are
any duplicates, the \texttt{sum(duplicated())} function will be used.

\hypertarget{daily-dataframes-5}{%
\subparagraph{Daily DataFrames}\label{daily-dataframes-5}}

\begin{Shaded}
\begin{Highlighting}[]
\CommentTok{\# Checking for duplicates: Daily DataFrames}
\FunctionTok{print}\NormalTok{(}\StringTok{"Daily Activity"}\NormalTok{)}
\end{Highlighting}
\end{Shaded}

\begin{verbatim}
## [1] "Daily Activity"
\end{verbatim}

\begin{Shaded}
\begin{Highlighting}[]
\FunctionTok{sum}\NormalTok{(}\FunctionTok{duplicated}\NormalTok{(clean\_daily\_activity))}
\end{Highlighting}
\end{Shaded}

\begin{verbatim}
## [1] 0
\end{verbatim}

\begin{Shaded}
\begin{Highlighting}[]
\FunctionTok{print}\NormalTok{(}\StringTok{"Daily Calories"}\NormalTok{)}
\end{Highlighting}
\end{Shaded}

\begin{verbatim}
## [1] "Daily Calories"
\end{verbatim}

\begin{Shaded}
\begin{Highlighting}[]
\FunctionTok{sum}\NormalTok{(}\FunctionTok{duplicated}\NormalTok{(clean\_daily\_calories))}
\end{Highlighting}
\end{Shaded}

\begin{verbatim}
## [1] 0
\end{verbatim}

\begin{Shaded}
\begin{Highlighting}[]
\FunctionTok{print}\NormalTok{(}\StringTok{"Daily Intensities"}\NormalTok{)}
\end{Highlighting}
\end{Shaded}

\begin{verbatim}
## [1] "Daily Intensities"
\end{verbatim}

\begin{Shaded}
\begin{Highlighting}[]
\FunctionTok{sum}\NormalTok{(}\FunctionTok{duplicated}\NormalTok{(clean\_daily\_intensities))}
\end{Highlighting}
\end{Shaded}

\begin{verbatim}
## [1] 0
\end{verbatim}

\begin{Shaded}
\begin{Highlighting}[]
\FunctionTok{print}\NormalTok{(}\StringTok{"Daily Steps"}\NormalTok{)}
\end{Highlighting}
\end{Shaded}

\begin{verbatim}
## [1] "Daily Steps"
\end{verbatim}

\begin{Shaded}
\begin{Highlighting}[]
\FunctionTok{sum}\NormalTok{(}\FunctionTok{duplicated}\NormalTok{(clean\_daily\_steps))}
\end{Highlighting}
\end{Shaded}

\begin{verbatim}
## [1] 0
\end{verbatim}

\begin{Shaded}
\begin{Highlighting}[]
\FunctionTok{print}\NormalTok{(}\StringTok{"Daily Sleep"}\NormalTok{)}
\end{Highlighting}
\end{Shaded}

\begin{verbatim}
## [1] "Daily Sleep"
\end{verbatim}

\begin{Shaded}
\begin{Highlighting}[]
\FunctionTok{sum}\NormalTok{(}\FunctionTok{duplicated}\NormalTok{(clean\_daily\_sleep))}
\end{Highlighting}
\end{Shaded}

\begin{verbatim}
## [1] 3
\end{verbatim}

\hypertarget{hourly-dataframes-5}{%
\subparagraph{Hourly DataFrames}\label{hourly-dataframes-5}}

\begin{Shaded}
\begin{Highlighting}[]
\CommentTok{\# Checking for duplicates: Hourly DataFrames}
\FunctionTok{print}\NormalTok{(}\StringTok{"Hourly Calories"}\NormalTok{)}
\end{Highlighting}
\end{Shaded}

\begin{verbatim}
## [1] "Hourly Calories"
\end{verbatim}

\begin{Shaded}
\begin{Highlighting}[]
\FunctionTok{sum}\NormalTok{(}\FunctionTok{duplicated}\NormalTok{(clean\_hourly\_calories))}
\end{Highlighting}
\end{Shaded}

\begin{verbatim}
## [1] 0
\end{verbatim}

\begin{Shaded}
\begin{Highlighting}[]
\FunctionTok{print}\NormalTok{(}\StringTok{"Hourly Intensities"}\NormalTok{)}
\end{Highlighting}
\end{Shaded}

\begin{verbatim}
## [1] "Hourly Intensities"
\end{verbatim}

\begin{Shaded}
\begin{Highlighting}[]
\FunctionTok{sum}\NormalTok{(}\FunctionTok{duplicated}\NormalTok{(clean\_hourly\_intensities))}
\end{Highlighting}
\end{Shaded}

\begin{verbatim}
## [1] 0
\end{verbatim}

\begin{Shaded}
\begin{Highlighting}[]
\FunctionTok{print}\NormalTok{(}\StringTok{"Hourly Steps"}\NormalTok{)}
\end{Highlighting}
\end{Shaded}

\begin{verbatim}
## [1] "Hourly Steps"
\end{verbatim}

\begin{Shaded}
\begin{Highlighting}[]
\FunctionTok{sum}\NormalTok{(}\FunctionTok{duplicated}\NormalTok{(clean\_hourly\_steps))}
\end{Highlighting}
\end{Shaded}

\begin{verbatim}
## [1] 0
\end{verbatim}

\hypertarget{minute-dataframes-5}{%
\subparagraph{Minute DataFrames}\label{minute-dataframes-5}}

\begin{Shaded}
\begin{Highlighting}[]
\CommentTok{\# Checking for duplicates: Minute DataFrames}
\FunctionTok{print}\NormalTok{(}\StringTok{"Minute Calories"}\NormalTok{)}
\end{Highlighting}
\end{Shaded}

\begin{verbatim}
## [1] "Minute Calories"
\end{verbatim}

\begin{Shaded}
\begin{Highlighting}[]
\FunctionTok{sum}\NormalTok{(}\FunctionTok{duplicated}\NormalTok{(clean\_minute\_calories))}
\end{Highlighting}
\end{Shaded}

\begin{verbatim}
## [1] 0
\end{verbatim}

\begin{Shaded}
\begin{Highlighting}[]
\FunctionTok{print}\NormalTok{(}\StringTok{"Minute Intensities"}\NormalTok{)}
\end{Highlighting}
\end{Shaded}

\begin{verbatim}
## [1] "Minute Intensities"
\end{verbatim}

\begin{Shaded}
\begin{Highlighting}[]
\FunctionTok{sum}\NormalTok{(}\FunctionTok{duplicated}\NormalTok{(clean\_minute\_intensities))}
\end{Highlighting}
\end{Shaded}

\begin{verbatim}
## [1] 0
\end{verbatim}

\begin{Shaded}
\begin{Highlighting}[]
\FunctionTok{print}\NormalTok{(}\StringTok{"Minute METs"}\NormalTok{)}
\end{Highlighting}
\end{Shaded}

\begin{verbatim}
## [1] "Minute METs"
\end{verbatim}

\begin{Shaded}
\begin{Highlighting}[]
\FunctionTok{sum}\NormalTok{(}\FunctionTok{duplicated}\NormalTok{(clean\_minute\_METs))}
\end{Highlighting}
\end{Shaded}

\begin{verbatim}
## [1] 0
\end{verbatim}

\begin{Shaded}
\begin{Highlighting}[]
\FunctionTok{print}\NormalTok{(}\StringTok{"Minute Sleep"}\NormalTok{)}
\end{Highlighting}
\end{Shaded}

\begin{verbatim}
## [1] "Minute Sleep"
\end{verbatim}

\begin{Shaded}
\begin{Highlighting}[]
\FunctionTok{sum}\NormalTok{(}\FunctionTok{duplicated}\NormalTok{(clean\_minute\_sleep))}
\end{Highlighting}
\end{Shaded}

\begin{verbatim}
## [1] 543
\end{verbatim}

\begin{Shaded}
\begin{Highlighting}[]
\FunctionTok{print}\NormalTok{(}\StringTok{"Minute Steps"}\NormalTok{)}
\end{Highlighting}
\end{Shaded}

\begin{verbatim}
## [1] "Minute Steps"
\end{verbatim}

\begin{Shaded}
\begin{Highlighting}[]
\FunctionTok{sum}\NormalTok{(}\FunctionTok{duplicated}\NormalTok{(clean\_minute\_steps))}
\end{Highlighting}
\end{Shaded}

\begin{verbatim}
## [1] 0
\end{verbatim}

\hypertarget{observation-4}{%
\subparagraph{Observation}\label{observation-4}}

Duplicates have been identified in the ``daily\_sleep'' and
``minute\_sleep'' dataframes.

\#\#\#\#\#Actions Taken To remove these duplicates, the !duplicated()
function will be used.

\begin{Shaded}
\begin{Highlighting}[]
\CommentTok{\# Removing duplicates from the "Daily Sleep" DataFrame}
\NormalTok{clean\_daily\_sleep }\OtherTok{\textless{}{-}}\NormalTok{ clean\_daily\_sleep[}\SpecialCharTok{!}\FunctionTok{duplicated}\NormalTok{(clean\_daily\_sleep),]}
\CommentTok{\# Checking if the duplicates are removed from the "Daily Sleep" DataFrame}
\FunctionTok{print}\NormalTok{(}\StringTok{"Daily Sleep clean of duplicates"}\NormalTok{)}
\end{Highlighting}
\end{Shaded}

\begin{verbatim}
## [1] "Daily Sleep clean of duplicates"
\end{verbatim}

\begin{Shaded}
\begin{Highlighting}[]
\FunctionTok{sum}\NormalTok{(}\FunctionTok{duplicated}\NormalTok{(clean\_daily\_sleep))}
\end{Highlighting}
\end{Shaded}

\begin{verbatim}
## [1] 0
\end{verbatim}

\begin{Shaded}
\begin{Highlighting}[]
\CommentTok{\# Removing duplicates from the "Minute Sleep" DataFrame}
\NormalTok{clean\_minute\_sleep }\OtherTok{\textless{}{-}}\NormalTok{ clean\_minute\_sleep[}\SpecialCharTok{!}\FunctionTok{duplicated}\NormalTok{(clean\_minute\_sleep),]}
\CommentTok{\# Checking if the duplicates are removed from the "Minute Sleep" DataFrame}
\FunctionTok{print}\NormalTok{(}\StringTok{"Minute Sleep clean of duplicates"}\NormalTok{)}
\end{Highlighting}
\end{Shaded}

\begin{verbatim}
## [1] "Minute Sleep clean of duplicates"
\end{verbatim}

\begin{Shaded}
\begin{Highlighting}[]
\FunctionTok{sum}\NormalTok{(}\FunctionTok{duplicated}\NormalTok{(clean\_minute\_sleep))}
\end{Highlighting}
\end{Shaded}

\begin{verbatim}
## [1] 0
\end{verbatim}

\hypertarget{adding-weekday-column}{%
\paragraph{3.5 Adding `weekday' Column}\label{adding-weekday-column}}

In order to facilitate analysis regarding participant activity levels on
specific days, a ``weekday'' column will be incorporated into all Daily
DataFrames and some Hourly DataFrames.

\begin{Shaded}
\begin{Highlighting}[]
\CommentTok{\# Adding the "weekday" column}
\NormalTok{clean\_daily\_activity}\SpecialCharTok{$}\NormalTok{weekday }\OtherTok{\textless{}{-}} \FunctionTok{weekdays}\NormalTok{(}\FunctionTok{as.Date}\NormalTok{(clean\_daily\_activity}\SpecialCharTok{$}\NormalTok{activity\_date\_ymd))}
\FunctionTok{print}\NormalTok{(}\StringTok{"Daily Activity"}\NormalTok{)}
\end{Highlighting}
\end{Shaded}

\begin{verbatim}
## [1] "Daily Activity"
\end{verbatim}

\begin{Shaded}
\begin{Highlighting}[]
\FunctionTok{glimpse}\NormalTok{(clean\_daily\_activity)}
\end{Highlighting}
\end{Shaded}

\begin{verbatim}
## Rows: 940
## Columns: 17
## $ id                         <dbl> 1503960366, 1503960366, 1503960366, 1503960~
## $ activity_date              <chr> "4/12/2016", "4/13/2016", "4/14/2016", "4/1~
## $ total_steps                <dbl> 13162, 10735, 10460, 9762, 12669, 9705, 130~
## $ total_distance             <dbl> 8.50, 6.97, 6.74, 6.28, 8.16, 6.48, 8.59, 9~
## $ tracker_distance           <dbl> 8.50, 6.97, 6.74, 6.28, 8.16, 6.48, 8.59, 9~
## $ logged_activities_distance <dbl> 0, 0, 0, 0, 0, 0, 0, 0, 0, 0, 0, 0, 0, 0, 0~
## $ very_active_distance       <dbl> 1.88, 1.57, 2.44, 2.14, 2.71, 3.19, 3.25, 3~
## $ moderately_active_distance <dbl> 0.55, 0.69, 0.40, 1.26, 0.41, 0.78, 0.64, 1~
## $ light_active_distance      <dbl> 6.06, 4.71, 3.91, 2.83, 5.04, 2.51, 4.71, 5~
## $ sedentary_active_distance  <dbl> 0, 0, 0, 0, 0, 0, 0, 0, 0, 0, 0, 0, 0, 0, 0~
## $ very_active_minutes        <dbl> 25, 21, 30, 29, 36, 38, 42, 50, 28, 19, 66,~
## $ fairly_active_minutes      <dbl> 13, 19, 11, 34, 10, 20, 16, 31, 12, 8, 27, ~
## $ lightly_active_minutes     <dbl> 328, 217, 181, 209, 221, 164, 233, 264, 205~
## $ sedentary_minutes          <dbl> 728, 776, 1218, 726, 773, 539, 1149, 775, 8~
## $ calories                   <dbl> 1985, 1797, 1776, 1745, 1863, 1728, 1921, 2~
## $ activity_date_ymd          <date> 2016-04-12, 2016-04-13, 2016-04-14, 2016-0~
## $ weekday                    <chr> "Tuesday", "Wednesday", "Thursday", "Friday~
\end{verbatim}

\begin{Shaded}
\begin{Highlighting}[]
\NormalTok{clean\_daily\_calories}\SpecialCharTok{$}\NormalTok{weekday }\OtherTok{\textless{}{-}} \FunctionTok{weekdays}\NormalTok{(}\FunctionTok{as.Date}\NormalTok{(clean\_daily\_calories}\SpecialCharTok{$}\NormalTok{activity\_date\_ymd))}
\FunctionTok{print}\NormalTok{(}\StringTok{"Daily Calories"}\NormalTok{)}
\end{Highlighting}
\end{Shaded}

\begin{verbatim}
## [1] "Daily Calories"
\end{verbatim}

\begin{Shaded}
\begin{Highlighting}[]
\FunctionTok{glimpse}\NormalTok{(clean\_daily\_calories)}
\end{Highlighting}
\end{Shaded}

\begin{verbatim}
## Rows: 940
## Columns: 5
## $ id                <dbl> 1503960366, 1503960366, 1503960366, 1503960366, 1503~
## $ activity_date     <chr> "4/12/2016", "4/13/2016", "4/14/2016", "4/15/2016", ~
## $ calories          <dbl> 1985, 1797, 1776, 1745, 1863, 1728, 1921, 2035, 1786~
## $ activity_date_ymd <date> 2016-04-12, 2016-04-13, 2016-04-14, 2016-04-15, 201~
## $ weekday           <chr> "Tuesday", "Wednesday", "Thursday", "Friday", "Satur~
\end{verbatim}

\begin{Shaded}
\begin{Highlighting}[]
\NormalTok{clean\_daily\_intensities}\SpecialCharTok{$}\NormalTok{weekday }\OtherTok{\textless{}{-}} \FunctionTok{weekdays}\NormalTok{(}\FunctionTok{as.Date}\NormalTok{(clean\_daily\_intensities}\SpecialCharTok{$}\NormalTok{activity\_date\_ymd))}
\FunctionTok{print}\NormalTok{(}\StringTok{"Daily Intensities"}\NormalTok{)}
\end{Highlighting}
\end{Shaded}

\begin{verbatim}
## [1] "Daily Intensities"
\end{verbatim}

\begin{Shaded}
\begin{Highlighting}[]
\FunctionTok{glimpse}\NormalTok{(clean\_daily\_intensities)}
\end{Highlighting}
\end{Shaded}

\begin{verbatim}
## Rows: 940
## Columns: 12
## $ id                         <dbl> 1503960366, 1503960366, 1503960366, 1503960~
## $ activity_date              <chr> "4/12/2016", "4/13/2016", "4/14/2016", "4/1~
## $ sedentary_minutes          <dbl> 728, 776, 1218, 726, 773, 539, 1149, 775, 8~
## $ lightly_active_minutes     <dbl> 328, 217, 181, 209, 221, 164, 233, 264, 205~
## $ fairly_active_minutes      <dbl> 13, 19, 11, 34, 10, 20, 16, 31, 12, 8, 27, ~
## $ very_active_minutes        <dbl> 25, 21, 30, 29, 36, 38, 42, 50, 28, 19, 66,~
## $ sedentary_active_distance  <dbl> 0, 0, 0, 0, 0, 0, 0, 0, 0, 0, 0, 0, 0, 0, 0~
## $ light_active_distance      <dbl> 6.06, 4.71, 3.91, 2.83, 5.04, 2.51, 4.71, 5~
## $ moderately_active_distance <dbl> 0.55, 0.69, 0.40, 1.26, 0.41, 0.78, 0.64, 1~
## $ very_active_distance       <dbl> 1.88, 1.57, 2.44, 2.14, 2.71, 3.19, 3.25, 3~
## $ activity_date_ymd          <date> 2016-04-12, 2016-04-13, 2016-04-14, 2016-0~
## $ weekday                    <chr> "Tuesday", "Wednesday", "Thursday", "Friday~
\end{verbatim}

\begin{Shaded}
\begin{Highlighting}[]
\NormalTok{clean\_daily\_steps}\SpecialCharTok{$}\NormalTok{weekday }\OtherTok{\textless{}{-}} \FunctionTok{weekdays}\NormalTok{(}\FunctionTok{as.Date}\NormalTok{(clean\_daily\_steps}\SpecialCharTok{$}\NormalTok{activity\_date\_ymd))}
\FunctionTok{print}\NormalTok{(}\StringTok{"Daily Steps"}\NormalTok{)}
\end{Highlighting}
\end{Shaded}

\begin{verbatim}
## [1] "Daily Steps"
\end{verbatim}

\begin{Shaded}
\begin{Highlighting}[]
\FunctionTok{glimpse}\NormalTok{(clean\_daily\_steps)}
\end{Highlighting}
\end{Shaded}

\begin{verbatim}
## Rows: 940
## Columns: 5
## $ id                <dbl> 1503960366, 1503960366, 1503960366, 1503960366, 1503~
## $ activity_date     <chr> "4/12/2016", "4/13/2016", "4/14/2016", "4/15/2016", ~
## $ total_steps       <dbl> 13162, 10735, 10460, 9762, 12669, 9705, 13019, 15506~
## $ activity_date_ymd <date> 2016-04-12, 2016-04-13, 2016-04-14, 2016-04-15, 201~
## $ weekday           <chr> "Tuesday", "Wednesday", "Thursday", "Friday", "Satur~
\end{verbatim}

\begin{Shaded}
\begin{Highlighting}[]
\NormalTok{clean\_daily\_sleep}\SpecialCharTok{$}\NormalTok{weekday }\OtherTok{\textless{}{-}} \FunctionTok{weekdays}\NormalTok{(}\FunctionTok{as.Date}\NormalTok{(clean\_daily\_sleep}\SpecialCharTok{$}\NormalTok{activity\_date\_ymd))}
\FunctionTok{print}\NormalTok{(}\StringTok{"Daily Sleep"}\NormalTok{)}
\end{Highlighting}
\end{Shaded}

\begin{verbatim}
## [1] "Daily Sleep"
\end{verbatim}

\begin{Shaded}
\begin{Highlighting}[]
\FunctionTok{glimpse}\NormalTok{(clean\_daily\_sleep)}
\end{Highlighting}
\end{Shaded}

\begin{verbatim}
## Rows: 410
## Columns: 7
## $ id                   <dbl> 1503960366, 1503960366, 1503960366, 1503960366, 1~
## $ activity_date        <chr> "4/12/2016 12:00:00 AM", "4/13/2016 12:00:00 AM",~
## $ total_sleep_records  <dbl> 1, 2, 1, 2, 1, 1, 1, 1, 1, 1, 1, 1, 1, 1, 1, 1, 1~
## $ total_minutes_asleep <dbl> 327, 384, 412, 340, 700, 304, 360, 325, 361, 430,~
## $ total_time_in_bed    <dbl> 346, 407, 442, 367, 712, 320, 377, 364, 384, 449,~
## $ activity_date_ymd    <dttm> 2016-04-12, 2016-04-13, 2016-04-15, 2016-04-16, ~
## $ weekday              <chr> "Monday", "Tuesday", "Thursday", "Friday", "Satur~
\end{verbatim}

\begin{Shaded}
\begin{Highlighting}[]
\NormalTok{clean\_hourly\_intensities}\SpecialCharTok{$}\NormalTok{weekday }\OtherTok{\textless{}{-}} \FunctionTok{weekdays}\NormalTok{(}\FunctionTok{as.Date}\NormalTok{(clean\_hourly\_intensities}\SpecialCharTok{$}\NormalTok{activity\_hour\_ymdhms))}
\FunctionTok{print}\NormalTok{(}\StringTok{"Daily Sleep"}\NormalTok{)}
\end{Highlighting}
\end{Shaded}

\begin{verbatim}
## [1] "Daily Sleep"
\end{verbatim}

\begin{Shaded}
\begin{Highlighting}[]
\FunctionTok{glimpse}\NormalTok{(clean\_hourly\_intensities)}
\end{Highlighting}
\end{Shaded}

\begin{verbatim}
## Rows: 22,099
## Columns: 6
## $ id                   <dbl> 1503960366, 1503960366, 1503960366, 1503960366, 1~
## $ activity_hour        <chr> "4/12/2016 12:00:00 AM", "4/12/2016 1:00:00 AM", ~
## $ total_intensity      <dbl> 20, 8, 7, 0, 0, 0, 0, 0, 13, 30, 29, 12, 11, 6, 3~
## $ average_intensity    <dbl> 0.333333, 0.133333, 0.116667, 0.000000, 0.000000,~
## $ activity_hour_ymdhms <dttm> 2016-04-12 00:00:00, 2016-04-12 01:00:00, 2016-0~
## $ weekday              <chr> "Monday", "Tuesday", "Tuesday", "Tuesday", "Tuesd~
\end{verbatim}

\hypertarget{merging-datasets}{%
\paragraph{3.6 Merging Datasets}\label{merging-datasets}}

Upon examination, it was noted that the daily\_activity dataframe
contains the same variables and attributes present in the
daily\_calories, daily\_intensities, and daily\_steps dataframes.
However, to make sure that those variables are identical, the
all.equal() function will be used. Identical columns will be checked in
all dataframes to identify if any of them can be merged.

\hypertarget{daily-dataframes-6}{%
\subparagraph{Daily DataFrames}\label{daily-dataframes-6}}

\begin{Shaded}
\begin{Highlighting}[]
\CommentTok{\# Checking if columns are identical: Daily DataFrames}
\FunctionTok{print}\NormalTok{(}\StringTok{"{-}{-}{-}{-}{-}Daily Activity, Daily Calories{-}{-}{-}{-}{-}"}\NormalTok{)}
\end{Highlighting}
\end{Shaded}

\begin{verbatim}
## [1] "-----Daily Activity, Daily Calories-----"
\end{verbatim}

\begin{Shaded}
\begin{Highlighting}[]
\ControlFlowTok{if}\NormalTok{ (}\FunctionTok{all.equal}\NormalTok{(clean\_daily\_activity}\SpecialCharTok{$}\NormalTok{id, clean\_daily\_calories}\SpecialCharTok{$}\NormalTok{id) }\SpecialCharTok{!=}\DecValTok{0}\NormalTok{) \{}
  \FunctionTok{print}\NormalTok{(}\StringTok{"Id: identical"}\NormalTok{)}
\NormalTok{\}}
\end{Highlighting}
\end{Shaded}

\begin{verbatim}
## [1] "Id: identical"
\end{verbatim}

\begin{Shaded}
\begin{Highlighting}[]
\ControlFlowTok{if}\NormalTok{ (}\FunctionTok{all.equal}\NormalTok{(clean\_daily\_activity}\SpecialCharTok{$}\NormalTok{activity\_date, clean\_daily\_calories}\SpecialCharTok{$}\NormalTok{activity\_date) }\SpecialCharTok{!=}\DecValTok{0}\NormalTok{) \{}
  \FunctionTok{print}\NormalTok{(}\StringTok{"Activity Date: identical"}\NormalTok{)}
\NormalTok{\}}
\end{Highlighting}
\end{Shaded}

\begin{verbatim}
## [1] "Activity Date: identical"
\end{verbatim}

\begin{Shaded}
\begin{Highlighting}[]
\ControlFlowTok{if}\NormalTok{ (}\FunctionTok{all.equal}\NormalTok{(clean\_daily\_activity}\SpecialCharTok{$}\NormalTok{calories, clean\_daily\_calories}\SpecialCharTok{$}\NormalTok{calories) }\SpecialCharTok{!=}\DecValTok{0}\NormalTok{) \{}
  \FunctionTok{print}\NormalTok{(}\StringTok{"Calories: identical"}\NormalTok{)}
\NormalTok{\}}
\end{Highlighting}
\end{Shaded}

\begin{verbatim}
## [1] "Calories: identical"
\end{verbatim}

\begin{Shaded}
\begin{Highlighting}[]
\ControlFlowTok{if}\NormalTok{ (}\FunctionTok{all.equal}\NormalTok{(clean\_daily\_activity}\SpecialCharTok{$}\NormalTok{weekday, clean\_daily\_calories}\SpecialCharTok{$}\NormalTok{weekday) }\SpecialCharTok{!=}\DecValTok{0}\NormalTok{) \{}
  \FunctionTok{print}\NormalTok{(}\StringTok{"Weekday: identical"}\NormalTok{)}
\NormalTok{\}}
\end{Highlighting}
\end{Shaded}

\begin{verbatim}
## [1] "Weekday: identical"
\end{verbatim}

\begin{Shaded}
\begin{Highlighting}[]
\FunctionTok{print}\NormalTok{(}\StringTok{"{-}{-}{-}{-}{-}Daily Activity, Daily Intensities{-}{-}{-}{-}{-}"}\NormalTok{)}
\end{Highlighting}
\end{Shaded}

\begin{verbatim}
## [1] "-----Daily Activity, Daily Intensities-----"
\end{verbatim}

\begin{Shaded}
\begin{Highlighting}[]
\ControlFlowTok{if}\NormalTok{ (}\FunctionTok{all.equal}\NormalTok{(clean\_daily\_activity}\SpecialCharTok{$}\NormalTok{id, clean\_daily\_intensities}\SpecialCharTok{$}\NormalTok{id) }\SpecialCharTok{!=}\DecValTok{0}\NormalTok{) \{}
  \FunctionTok{print}\NormalTok{(}\StringTok{"Id: identical"}\NormalTok{)}
\NormalTok{\}}
\end{Highlighting}
\end{Shaded}

\begin{verbatim}
## [1] "Id: identical"
\end{verbatim}

\begin{Shaded}
\begin{Highlighting}[]
\ControlFlowTok{if}\NormalTok{ (}\FunctionTok{all.equal}\NormalTok{(clean\_daily\_activity}\SpecialCharTok{$}\NormalTok{activity\_date, clean\_daily\_intensities}\SpecialCharTok{$}\NormalTok{activity\_date) }\SpecialCharTok{!=}\DecValTok{0}\NormalTok{) \{}
  \FunctionTok{print}\NormalTok{(}\StringTok{"Activity Date: identical"}\NormalTok{)}
\NormalTok{\}}
\end{Highlighting}
\end{Shaded}

\begin{verbatim}
## [1] "Activity Date: identical"
\end{verbatim}

\begin{Shaded}
\begin{Highlighting}[]
\ControlFlowTok{if}\NormalTok{ (}\FunctionTok{all.equal}\NormalTok{(clean\_daily\_activity}\SpecialCharTok{$}\NormalTok{sedentary\_minutes, clean\_daily\_intensities}\SpecialCharTok{$}\NormalTok{sedentary\_minutes) }\SpecialCharTok{!=}\DecValTok{0}\NormalTok{) \{}
  \FunctionTok{print}\NormalTok{(}\StringTok{"Sedentary Minutes: identical"}\NormalTok{)}
\NormalTok{\}}
\end{Highlighting}
\end{Shaded}

\begin{verbatim}
## [1] "Sedentary Minutes: identical"
\end{verbatim}

\begin{Shaded}
\begin{Highlighting}[]
\ControlFlowTok{if}\NormalTok{ (}\FunctionTok{all.equal}\NormalTok{(clean\_daily\_activity}\SpecialCharTok{$}\NormalTok{lightly\_active\_minutes, clean\_daily\_intensities}\SpecialCharTok{$}\NormalTok{lightly\_active\_minutes) }\SpecialCharTok{!=}\DecValTok{0}\NormalTok{) \{}
  \FunctionTok{print}\NormalTok{(}\StringTok{"Lightly Active Minutes: identical"}\NormalTok{)}
\NormalTok{\}}
\end{Highlighting}
\end{Shaded}

\begin{verbatim}
## [1] "Lightly Active Minutes: identical"
\end{verbatim}

\begin{Shaded}
\begin{Highlighting}[]
\ControlFlowTok{if}\NormalTok{ (}\FunctionTok{all.equal}\NormalTok{(clean\_daily\_activity}\SpecialCharTok{$}\NormalTok{fairly\_active\_minutes, clean\_daily\_intensities}\SpecialCharTok{$}\NormalTok{fairly\_active\_minutes) }\SpecialCharTok{!=}\DecValTok{0}\NormalTok{) \{}
  \FunctionTok{print}\NormalTok{(}\StringTok{"Fairly Active Minutes: identical"}\NormalTok{)}
\NormalTok{\}}
\end{Highlighting}
\end{Shaded}

\begin{verbatim}
## [1] "Fairly Active Minutes: identical"
\end{verbatim}

\begin{Shaded}
\begin{Highlighting}[]
\ControlFlowTok{if}\NormalTok{ (}\FunctionTok{all.equal}\NormalTok{(clean\_daily\_activity}\SpecialCharTok{$}\NormalTok{very\_active\_minutes, clean\_daily\_intensities}\SpecialCharTok{$}\NormalTok{very\_active\_minutes) }\SpecialCharTok{!=}\DecValTok{0}\NormalTok{) \{}
  \FunctionTok{print}\NormalTok{(}\StringTok{"Very Active Minutes: identical"}\NormalTok{)}
\NormalTok{\}}
\end{Highlighting}
\end{Shaded}

\begin{verbatim}
## [1] "Very Active Minutes: identical"
\end{verbatim}

\begin{Shaded}
\begin{Highlighting}[]
\ControlFlowTok{if}\NormalTok{ (}\FunctionTok{all.equal}\NormalTok{(clean\_daily\_activity}\SpecialCharTok{$}\NormalTok{sedentary\_active\_distance, clean\_daily\_intensities}\SpecialCharTok{$}\NormalTok{sedentary\_active\_distance) }\SpecialCharTok{!=}\DecValTok{0}\NormalTok{) \{}
  \FunctionTok{print}\NormalTok{(}\StringTok{"Sedentary Active Distance: identical"}\NormalTok{)}
\NormalTok{\}}
\end{Highlighting}
\end{Shaded}

\begin{verbatim}
## [1] "Sedentary Active Distance: identical"
\end{verbatim}

\begin{Shaded}
\begin{Highlighting}[]
\ControlFlowTok{if}\NormalTok{ (}\FunctionTok{all.equal}\NormalTok{(clean\_daily\_activity}\SpecialCharTok{$}\NormalTok{light\_active\_distance, clean\_daily\_intensities}\SpecialCharTok{$}\NormalTok{light\_active\_distance) }\SpecialCharTok{!=}\DecValTok{0}\NormalTok{) \{}
  \FunctionTok{print}\NormalTok{(}\StringTok{"Light Active Distance: identical"}\NormalTok{)}
\NormalTok{\}}
\end{Highlighting}
\end{Shaded}

\begin{verbatim}
## [1] "Light Active Distance: identical"
\end{verbatim}

\begin{Shaded}
\begin{Highlighting}[]
\ControlFlowTok{if}\NormalTok{ (}\FunctionTok{all.equal}\NormalTok{(clean\_daily\_activity}\SpecialCharTok{$}\NormalTok{moderately\_active\_distance, clean\_daily\_intensities}\SpecialCharTok{$}\NormalTok{moderately\_active\_distance) }\SpecialCharTok{!=}\DecValTok{0}\NormalTok{) \{}
  \FunctionTok{print}\NormalTok{(}\StringTok{"Moderately Active Distance: identical"}\NormalTok{)}
\NormalTok{\}}
\end{Highlighting}
\end{Shaded}

\begin{verbatim}
## [1] "Moderately Active Distance: identical"
\end{verbatim}

\begin{Shaded}
\begin{Highlighting}[]
\ControlFlowTok{if}\NormalTok{ (}\FunctionTok{all.equal}\NormalTok{(clean\_daily\_activity}\SpecialCharTok{$}\NormalTok{very\_active\_distance, clean\_daily\_intensities}\SpecialCharTok{$}\NormalTok{very\_active\_distance) }\SpecialCharTok{!=}\DecValTok{0}\NormalTok{) \{}
  \FunctionTok{print}\NormalTok{(}\StringTok{"Very Active Distance: identical"}\NormalTok{)}
\NormalTok{\}}
\end{Highlighting}
\end{Shaded}

\begin{verbatim}
## [1] "Very Active Distance: identical"
\end{verbatim}

\begin{Shaded}
\begin{Highlighting}[]
\ControlFlowTok{if}\NormalTok{ (}\FunctionTok{all.equal}\NormalTok{(clean\_daily\_activity}\SpecialCharTok{$}\NormalTok{weekday, clean\_daily\_intensities}\SpecialCharTok{$}\NormalTok{weekday) }\SpecialCharTok{!=}\DecValTok{0}\NormalTok{) \{}
  \FunctionTok{print}\NormalTok{(}\StringTok{"Weekday: identical"}\NormalTok{)}
\NormalTok{\}}
\end{Highlighting}
\end{Shaded}

\begin{verbatim}
## [1] "Weekday: identical"
\end{verbatim}

\begin{Shaded}
\begin{Highlighting}[]
\FunctionTok{print}\NormalTok{(}\StringTok{"{-}{-}{-}{-}{-}Daily Activity, Daily Steps{-}{-}{-}{-}{-}"}\NormalTok{)}
\end{Highlighting}
\end{Shaded}

\begin{verbatim}
## [1] "-----Daily Activity, Daily Steps-----"
\end{verbatim}

\begin{Shaded}
\begin{Highlighting}[]
\ControlFlowTok{if}\NormalTok{ (}\FunctionTok{all.equal}\NormalTok{(clean\_daily\_activity}\SpecialCharTok{$}\NormalTok{id, clean\_daily\_steps}\SpecialCharTok{$}\NormalTok{id) }\SpecialCharTok{!=}\DecValTok{0}\NormalTok{) \{}
  \FunctionTok{print}\NormalTok{(}\StringTok{"Id: identical"}\NormalTok{)}
\NormalTok{\}}
\end{Highlighting}
\end{Shaded}

\begin{verbatim}
## [1] "Id: identical"
\end{verbatim}

\begin{Shaded}
\begin{Highlighting}[]
\ControlFlowTok{if}\NormalTok{ (}\FunctionTok{all.equal}\NormalTok{(clean\_daily\_activity}\SpecialCharTok{$}\NormalTok{activity\_date, clean\_daily\_steps}\SpecialCharTok{$}\NormalTok{activity\_date) }\SpecialCharTok{!=}\DecValTok{0}\NormalTok{) \{}
  \FunctionTok{print}\NormalTok{(}\StringTok{"Activity Date: identical"}\NormalTok{)}
\NormalTok{\}}
\end{Highlighting}
\end{Shaded}

\begin{verbatim}
## [1] "Activity Date: identical"
\end{verbatim}

\begin{Shaded}
\begin{Highlighting}[]
\ControlFlowTok{if}\NormalTok{ (}\FunctionTok{all.equal}\NormalTok{(clean\_daily\_activity}\SpecialCharTok{$}\NormalTok{total\_steps, clean\_daily\_steps}\SpecialCharTok{$}\NormalTok{total\_steps) }\SpecialCharTok{!=}\DecValTok{0}\NormalTok{) \{}
  \FunctionTok{print}\NormalTok{(}\StringTok{"Total Steps: identical"}\NormalTok{)}
\NormalTok{\}}
\end{Highlighting}
\end{Shaded}

\begin{verbatim}
## [1] "Total Steps: identical"
\end{verbatim}

\begin{Shaded}
\begin{Highlighting}[]
\ControlFlowTok{if}\NormalTok{ (}\FunctionTok{all.equal}\NormalTok{(clean\_daily\_activity}\SpecialCharTok{$}\NormalTok{weekday, clean\_daily\_steps}\SpecialCharTok{$}\NormalTok{weekday) }\SpecialCharTok{!=}\DecValTok{0}\NormalTok{) \{}
  \FunctionTok{print}\NormalTok{(}\StringTok{"Weekday: identical"}\NormalTok{)}
\NormalTok{\}}
\end{Highlighting}
\end{Shaded}

\begin{verbatim}
## [1] "Weekday: identical"
\end{verbatim}

\hypertarget{hourly-dataframes-6}{%
\subparagraph{Hourly DataFrames}\label{hourly-dataframes-6}}

\begin{Shaded}
\begin{Highlighting}[]
\CommentTok{\# Checking if columns are identical: Hourly DataFrames}
\FunctionTok{print}\NormalTok{(}\StringTok{"{-}{-}{-}{-}{-}Hourly Calories, Hourly Intensities{-}{-}{-}{-}{-}"}\NormalTok{)}
\end{Highlighting}
\end{Shaded}

\begin{verbatim}
## [1] "-----Hourly Calories, Hourly Intensities-----"
\end{verbatim}

\begin{Shaded}
\begin{Highlighting}[]
\ControlFlowTok{if}\NormalTok{ (}\FunctionTok{all.equal}\NormalTok{(clean\_hourly\_calories}\SpecialCharTok{$}\NormalTok{id, clean\_hourly\_intensities}\SpecialCharTok{$}\NormalTok{id) }\SpecialCharTok{!=}\DecValTok{0}\NormalTok{) \{}
  \FunctionTok{print}\NormalTok{(}\StringTok{"Id: identical"}\NormalTok{)}
\NormalTok{\}}
\end{Highlighting}
\end{Shaded}

\begin{verbatim}
## [1] "Id: identical"
\end{verbatim}

\begin{Shaded}
\begin{Highlighting}[]
\ControlFlowTok{if}\NormalTok{ (}\FunctionTok{all.equal}\NormalTok{(clean\_hourly\_calories}\SpecialCharTok{$}\NormalTok{activity\_hour, clean\_hourly\_intensities}\SpecialCharTok{$}\NormalTok{activity\_hour) }\SpecialCharTok{!=}\DecValTok{0}\NormalTok{) \{}
  \FunctionTok{print}\NormalTok{(}\StringTok{"Activity Hour: identical"}\NormalTok{)}
\NormalTok{\}}
\end{Highlighting}
\end{Shaded}

\begin{verbatim}
## [1] "Activity Hour: identical"
\end{verbatim}

\begin{Shaded}
\begin{Highlighting}[]
\FunctionTok{print}\NormalTok{(}\StringTok{"{-}{-}{-}{-}{-}Hourly Calories, Hourly Steps{-}{-}{-}{-}{-}"}\NormalTok{)}
\end{Highlighting}
\end{Shaded}

\begin{verbatim}
## [1] "-----Hourly Calories, Hourly Steps-----"
\end{verbatim}

\begin{Shaded}
\begin{Highlighting}[]
\ControlFlowTok{if}\NormalTok{ (}\FunctionTok{all.equal}\NormalTok{(clean\_hourly\_calories}\SpecialCharTok{$}\NormalTok{id, clean\_hourly\_steps}\SpecialCharTok{$}\NormalTok{id) }\SpecialCharTok{!=}\DecValTok{0}\NormalTok{) \{}
  \FunctionTok{print}\NormalTok{(}\StringTok{"Id: identical"}\NormalTok{)}
\NormalTok{\}}
\end{Highlighting}
\end{Shaded}

\begin{verbatim}
## [1] "Id: identical"
\end{verbatim}

\begin{Shaded}
\begin{Highlighting}[]
\ControlFlowTok{if}\NormalTok{ (}\FunctionTok{all.equal}\NormalTok{(clean\_hourly\_calories}\SpecialCharTok{$}\NormalTok{activity\_hour, clean\_hourly\_steps}\SpecialCharTok{$}\NormalTok{activity\_hour) }\SpecialCharTok{!=}\DecValTok{0}\NormalTok{) \{}
  \FunctionTok{print}\NormalTok{(}\StringTok{"Activity Hour: identical"}\NormalTok{)}
\NormalTok{\}}
\end{Highlighting}
\end{Shaded}

\begin{verbatim}
## [1] "Activity Hour: identical"
\end{verbatim}

\hypertarget{minute-dataframes-6}{%
\subparagraph{Minute DataFrames}\label{minute-dataframes-6}}

\begin{Shaded}
\begin{Highlighting}[]
\CommentTok{\# Checking if columns are identical: Minute DataFrames}
\FunctionTok{print}\NormalTok{(}\StringTok{"{-}{-}{-}{-}{-}Minute Calories, Minute Intensities{-}{-}{-}{-}{-}"}\NormalTok{)}
\end{Highlighting}
\end{Shaded}

\begin{verbatim}
## [1] "-----Minute Calories, Minute Intensities-----"
\end{verbatim}

\begin{Shaded}
\begin{Highlighting}[]
\ControlFlowTok{if}\NormalTok{ (}\FunctionTok{all.equal}\NormalTok{(clean\_minute\_calories}\SpecialCharTok{$}\NormalTok{id, clean\_minute\_intensities}\SpecialCharTok{$}\NormalTok{id) }\SpecialCharTok{!=}\DecValTok{0}\NormalTok{) \{}
  \FunctionTok{print}\NormalTok{(}\StringTok{"Id: identical"}\NormalTok{)}
\NormalTok{\}}
\end{Highlighting}
\end{Shaded}

\begin{verbatim}
## [1] "Id: identical"
\end{verbatim}

\begin{Shaded}
\begin{Highlighting}[]
\ControlFlowTok{if}\NormalTok{ (}\FunctionTok{all.equal}\NormalTok{(clean\_minute\_calories}\SpecialCharTok{$}\NormalTok{activity\_minute, clean\_minute\_intensities}\SpecialCharTok{$}\NormalTok{activity\_minute) }\SpecialCharTok{!=}\DecValTok{0}\NormalTok{) \{}
  \FunctionTok{print}\NormalTok{(}\StringTok{"Activity Minute: identical"}\NormalTok{)}
\NormalTok{\}}
\end{Highlighting}
\end{Shaded}

\begin{verbatim}
## [1] "Activity Minute: identical"
\end{verbatim}

\begin{Shaded}
\begin{Highlighting}[]
\FunctionTok{print}\NormalTok{(}\StringTok{"{-}{-}{-}{-}{-}Minute Calories, Minute METs{-}{-}{-}{-}{-}"}\NormalTok{)}
\end{Highlighting}
\end{Shaded}

\begin{verbatim}
## [1] "-----Minute Calories, Minute METs-----"
\end{verbatim}

\begin{Shaded}
\begin{Highlighting}[]
\ControlFlowTok{if}\NormalTok{ (}\FunctionTok{all.equal}\NormalTok{(clean\_minute\_calories}\SpecialCharTok{$}\NormalTok{id, clean\_minute\_METs}\SpecialCharTok{$}\NormalTok{id) }\SpecialCharTok{!=}\DecValTok{0}\NormalTok{) \{}
  \FunctionTok{print}\NormalTok{(}\StringTok{"Id: identical"}\NormalTok{)}
\NormalTok{\}}
\end{Highlighting}
\end{Shaded}

\begin{verbatim}
## [1] "Id: identical"
\end{verbatim}

\begin{Shaded}
\begin{Highlighting}[]
\ControlFlowTok{if}\NormalTok{ (}\FunctionTok{all.equal}\NormalTok{(clean\_minute\_calories}\SpecialCharTok{$}\NormalTok{activity\_minute, clean\_minute\_METs}\SpecialCharTok{$}\NormalTok{activity\_minute) }\SpecialCharTok{!=}\DecValTok{0}\NormalTok{) \{}
  \FunctionTok{print}\NormalTok{(}\StringTok{"Activity Minute: identical"}\NormalTok{)}
\NormalTok{\}}
\end{Highlighting}
\end{Shaded}

\begin{verbatim}
## [1] "Activity Minute: identical"
\end{verbatim}

\begin{Shaded}
\begin{Highlighting}[]
\FunctionTok{print}\NormalTok{(}\StringTok{"{-}{-}{-}{-}{-}Minute Calories, Minute Sleep{-}{-}{-}{-}{-}"}\NormalTok{)}
\end{Highlighting}
\end{Shaded}

\begin{verbatim}
## [1] "-----Minute Calories, Minute Sleep-----"
\end{verbatim}

\begin{Shaded}
\begin{Highlighting}[]
\ControlFlowTok{if}\NormalTok{ (}\FunctionTok{all.equal}\NormalTok{(clean\_minute\_calories}\SpecialCharTok{$}\NormalTok{id, clean\_minute\_sleep}\SpecialCharTok{$}\NormalTok{id) }\SpecialCharTok{!=}\DecValTok{0}\NormalTok{) \{}
  \FunctionTok{print}\NormalTok{(}\StringTok{"Id: identical"}\NormalTok{)}
\NormalTok{\}}
\end{Highlighting}
\end{Shaded}

\begin{verbatim}
## [1] "Id: identical"
\end{verbatim}

\begin{Shaded}
\begin{Highlighting}[]
\ControlFlowTok{if}\NormalTok{ (}\FunctionTok{all.equal}\NormalTok{(clean\_minute\_sleep}\SpecialCharTok{$}\NormalTok{activity\_minute, clean\_minute\_sleep}\SpecialCharTok{$}\NormalTok{activity\_minute) }\SpecialCharTok{!=}\DecValTok{0}\NormalTok{) \{}
  \FunctionTok{print}\NormalTok{(}\StringTok{"Activity Minute: identical"}\NormalTok{)}
\NormalTok{\}}
\end{Highlighting}
\end{Shaded}

\begin{verbatim}
## [1] "Activity Minute: identical"
\end{verbatim}

\begin{Shaded}
\begin{Highlighting}[]
\FunctionTok{print}\NormalTok{(}\StringTok{"{-}{-}{-}{-}{-}Minute Calories, Minute Steps{-}{-}{-}{-}{-}"}\NormalTok{)}
\end{Highlighting}
\end{Shaded}

\begin{verbatim}
## [1] "-----Minute Calories, Minute Steps-----"
\end{verbatim}

\begin{Shaded}
\begin{Highlighting}[]
\ControlFlowTok{if}\NormalTok{ (}\FunctionTok{all.equal}\NormalTok{(clean\_minute\_calories}\SpecialCharTok{$}\NormalTok{id, clean\_minute\_steps}\SpecialCharTok{$}\NormalTok{id) }\SpecialCharTok{!=}\DecValTok{0}\NormalTok{) \{}
  \FunctionTok{print}\NormalTok{(}\StringTok{"Id: identical"}\NormalTok{)}
\NormalTok{\}}
\end{Highlighting}
\end{Shaded}

\begin{verbatim}
## [1] "Id: identical"
\end{verbatim}

\begin{Shaded}
\begin{Highlighting}[]
\ControlFlowTok{if}\NormalTok{ (}\FunctionTok{all.equal}\NormalTok{(clean\_minute\_steps}\SpecialCharTok{$}\NormalTok{activity\_minute, clean\_minute\_steps}\SpecialCharTok{$}\NormalTok{activity\_minute) }\SpecialCharTok{!=}\DecValTok{0}\NormalTok{) \{}
  \FunctionTok{print}\NormalTok{(}\StringTok{"Activity Minute: identical"}\NormalTok{)}
\NormalTok{\}}
\end{Highlighting}
\end{Shaded}

\begin{verbatim}
## [1] "Activity Minute: identical"
\end{verbatim}

\hypertarget{actions-taken-1}{%
\subparagraph{Actions Taken}\label{actions-taken-1}}

In terms of Daily DataFrames, because data in daily\_calories,
daily\_intensities, and daily\_steps is already aggregated within the
daily\_activity dataframe, only the daily\_activity and daily\_sleep
dataframes will be merged. With regard to Hourly DataFrames, all of them
will be merged into one big dataframe. The same will be done with the
Minute DataFrames.

\hypertarget{daily-dataframes-7}{%
\subparagraph{Daily DataFrames}\label{daily-dataframes-7}}

\begin{Shaded}
\begin{Highlighting}[]
\CommentTok{\# Merging data: Daily DataFrames}
\NormalTok{daily\_activity\_merged }\OtherTok{\textless{}{-}} \FunctionTok{merge}\NormalTok{(clean\_daily\_activity, clean\_daily\_sleep, }\AttributeTok{by=}\FunctionTok{c}\NormalTok{(}\StringTok{"id"}\NormalTok{, }\StringTok{"activity\_date\_ymd"}\NormalTok{, }\StringTok{"weekday"}\NormalTok{), }\AttributeTok{all =} \ConstantTok{TRUE}\NormalTok{, }\AttributeTok{no.dups =} \ConstantTok{TRUE}\NormalTok{)}
\FunctionTok{glimpse}\NormalTok{(daily\_activity\_merged)}
\end{Highlighting}
\end{Shaded}

\begin{verbatim}
## Rows: 953
## Columns: 21
## $ id                         <dbl> 1503960366, 1503960366, 1503960366, 1503960~
## $ activity_date_ymd          <date> 2016-04-11, 2016-04-12, 2016-04-13, 2016-0~
## $ weekday                    <chr> "Monday", "Tuesday", "Wednesday", "Thursday~
## $ activity_date.x            <chr> NA, "4/12/2016", "4/13/2016", "4/14/2016", ~
## $ total_steps                <dbl> NA, 13162, 10735, 10460, 9762, 12669, 9705,~
## $ total_distance             <dbl> NA, 8.50, 6.97, 6.74, 6.28, 8.16, 6.48, 8.5~
## $ tracker_distance           <dbl> NA, 8.50, 6.97, 6.74, 6.28, 8.16, 6.48, 8.5~
## $ logged_activities_distance <dbl> NA, 0, 0, 0, 0, 0, 0, 0, 0, 0, 0, 0, 0, 0, ~
## $ very_active_distance       <dbl> NA, 1.88, 1.57, 2.44, 2.14, 2.71, 3.19, 3.2~
## $ moderately_active_distance <dbl> NA, 0.55, 0.69, 0.40, 1.26, 0.41, 0.78, 0.6~
## $ light_active_distance      <dbl> NA, 6.06, 4.71, 3.91, 2.83, 5.04, 2.51, 4.7~
## $ sedentary_active_distance  <dbl> NA, 0, 0, 0, 0, 0, 0, 0, 0, 0, 0, 0, 0, 0, ~
## $ very_active_minutes        <dbl> NA, 25, 21, 30, 29, 36, 38, 42, 50, 28, 19,~
## $ fairly_active_minutes      <dbl> NA, 13, 19, 11, 34, 10, 20, 16, 31, 12, 8, ~
## $ lightly_active_minutes     <dbl> NA, 328, 217, 181, 209, 221, 164, 233, 264,~
## $ sedentary_minutes          <dbl> NA, 728, 776, 1218, 726, 773, 539, 1149, 77~
## $ calories                   <dbl> NA, 1985, 1797, 1776, 1745, 1863, 1728, 192~
## $ activity_date.y            <chr> "4/12/2016 12:00:00 AM", "4/13/2016 12:00:0~
## $ total_sleep_records        <dbl> 1, 2, NA, 1, 2, 1, NA, 1, 1, 1, NA, 1, 1, 1~
## $ total_minutes_asleep       <dbl> 327, 384, NA, 412, 340, 700, NA, 304, 360, ~
## $ total_time_in_bed          <dbl> 346, 407, NA, 442, 367, 712, NA, 320, 377, ~
\end{verbatim}

\hypertarget{hourly-dataframes-7}{%
\subparagraph{Hourly DataFrames}\label{hourly-dataframes-7}}

\begin{Shaded}
\begin{Highlighting}[]
\CommentTok{\# Merging data: Hourly DataFrames}
\NormalTok{hourly\_activity\_m }\OtherTok{\textless{}{-}} \FunctionTok{merge}\NormalTok{(clean\_hourly\_calories, clean\_hourly\_intensities, }\AttributeTok{by=}\FunctionTok{c}\NormalTok{(}\StringTok{"id"}\NormalTok{, }\StringTok{"activity\_hour"}\NormalTok{, }\StringTok{"activity\_hour\_ymdhms"}\NormalTok{), }\AttributeTok{all =} \ConstantTok{TRUE}\NormalTok{, }\AttributeTok{no.dups =} \ConstantTok{TRUE}\NormalTok{)}
\NormalTok{hourly\_activity\_merged }\OtherTok{\textless{}{-}} \FunctionTok{merge}\NormalTok{(hourly\_activity\_m, clean\_hourly\_steps, }\AttributeTok{by=}\FunctionTok{c}\NormalTok{(}\StringTok{"id"}\NormalTok{, }\StringTok{"activity\_hour"}\NormalTok{, }\StringTok{"activity\_hour\_ymdhms"}\NormalTok{), }\AttributeTok{all =} \ConstantTok{TRUE}\NormalTok{, }\AttributeTok{no.dups =} \ConstantTok{TRUE}\NormalTok{)}
\FunctionTok{glimpse}\NormalTok{(hourly\_activity\_merged)}
\end{Highlighting}
\end{Shaded}

\begin{verbatim}
## Rows: 22,099
## Columns: 8
## $ id                   <dbl> 1503960366, 1503960366, 1503960366, 1503960366, 1~
## $ activity_hour        <chr> "4/12/2016 1:00:00 AM", "4/12/2016 1:00:00 PM", "~
## $ activity_hour_ymdhms <dttm> 2016-04-12 01:00:00, 2016-04-12 13:00:00, 2016-0~
## $ calories             <dbl> 61, 66, 99, 65, 76, 81, 81, 73, 59, 110, 47, 151,~
## $ total_intensity      <dbl> 8, 6, 29, 9, 12, 21, 20, 11, 7, 36, 0, 58, 0, 13,~
## $ average_intensity    <dbl> 0.133333, 0.100000, 0.483333, 0.150000, 0.200000,~
## $ weekday              <chr> "Tuesday", "Tuesday", "Tuesday", "Tuesday", "Tues~
## $ total_steps          <dbl> 160, 221, 676, 89, 360, 338, 373, 253, 151, 1166,~
\end{verbatim}

\hypertarget{minute-dataframes-7}{%
\subparagraph{Minute DataFrames}\label{minute-dataframes-7}}

\begin{Shaded}
\begin{Highlighting}[]
\CommentTok{\# Merging data: Minute DataFrames}
\NormalTok{minute\_activity\_m }\OtherTok{\textless{}{-}} \FunctionTok{merge}\NormalTok{(clean\_minute\_calories,clean\_minute\_intensities, }\AttributeTok{by=}\FunctionTok{c}\NormalTok{(}\StringTok{"id"}\NormalTok{, }\StringTok{"activity\_minute"}\NormalTok{, }\StringTok{"activity\_minute\_ymdhms"}\NormalTok{), }\AttributeTok{all =} \ConstantTok{TRUE}\NormalTok{, }\AttributeTok{no.dups =} \ConstantTok{TRUE}\NormalTok{)}
\NormalTok{minute\_activity\_m1 }\OtherTok{\textless{}{-}} \FunctionTok{merge}\NormalTok{(minute\_activity\_m, clean\_minute\_METs, }\AttributeTok{by=}\FunctionTok{c}\NormalTok{(}\StringTok{"id"}\NormalTok{, }\StringTok{"activity\_minute"}\NormalTok{, }\StringTok{"activity\_minute\_ymdhms"}\NormalTok{), }\AttributeTok{all =} \ConstantTok{TRUE}\NormalTok{, }\AttributeTok{no.dups =} \ConstantTok{TRUE}\NormalTok{)}
\NormalTok{minute\_activity\_m2 }\OtherTok{\textless{}{-}} \FunctionTok{merge}\NormalTok{(minute\_activity\_m1, clean\_minute\_sleep, }\AttributeTok{by=}\FunctionTok{c}\NormalTok{(}\StringTok{"id"}\NormalTok{, }\StringTok{"activity\_minute"}\NormalTok{, }\StringTok{"activity\_minute\_ymdhms"}\NormalTok{), }\AttributeTok{all =} \ConstantTok{TRUE}\NormalTok{, }\AttributeTok{no.dups =} \ConstantTok{TRUE}\NormalTok{)}
\NormalTok{minute\_activity\_merged }\OtherTok{\textless{}{-}} \FunctionTok{merge}\NormalTok{(minute\_activity\_m2, clean\_minute\_steps, }\AttributeTok{by=}\FunctionTok{c}\NormalTok{(}\StringTok{"id"}\NormalTok{, }\StringTok{"activity\_minute"}\NormalTok{, }\StringTok{"activity\_minute\_ymdhms"}\NormalTok{), }\AttributeTok{all =} \ConstantTok{TRUE}\NormalTok{, }\AttributeTok{no.dups =} \ConstantTok{TRUE}\NormalTok{)}
\FunctionTok{glimpse}\NormalTok{(minute\_activity\_merged)}
\end{Highlighting}
\end{Shaded}

\begin{verbatim}
## Rows: 1,388,198
## Columns: 9
## $ id                     <dbl> 1503960366, 1503960366, 1503960366, 1503960366,~
## $ activity_minute        <chr> "4/12/2016 1:00:00 AM", "4/12/2016 1:00:00 PM",~
## $ activity_minute_ymdhms <dttm> 2016-04-12 01:00:00, 2016-04-12 13:00:00, 2016~
## $ calories               <dbl> 0.9438, 0.9438, 2.6741, 0.9438, 2.0449, 0.9438,~
## $ intensity              <dbl> 0, 0, 1, 0, 1, 0, 1, 1, 1, 1, 0, 0, 0, 0, 0, 0,~
## $ mets                   <dbl> 12, 12, 34, 12, 26, 12, 34, 30, 30, 26, 12, 12,~
## $ sleep_state            <dbl> NA, NA, NA, NA, NA, NA, NA, NA, NA, NA, NA, NA,~
## $ log_id                 <dbl> NA, NA, NA, NA, NA, NA, NA, NA, NA, NA, NA, NA,~
## $ total_steps            <dbl> 0, 0, 36, 0, 9, 0, 34, 21, 23, 9, 0, 0, 0, 0, 0~
\end{verbatim}

\hypertarget{summary-of-the-datasets}{%
\paragraph{3.7 Summary of the Datasets}\label{summary-of-the-datasets}}

Summarising all the datasets will provide an overview and reveal any
other errors or outliers that were missed in the previous sections.

\begin{Shaded}
\begin{Highlighting}[]
\CommentTok{\# Summary of Daily DataFrames}
\FunctionTok{print}\NormalTok{(}\StringTok{"{-}{-}{-}{-}{-}Daily Calories{-}{-}{-}{-}{-}"}\NormalTok{)}
\end{Highlighting}
\end{Shaded}

\begin{verbatim}
## [1] "-----Daily Calories-----"
\end{verbatim}

\begin{Shaded}
\begin{Highlighting}[]
\NormalTok{clean\_daily\_calories }\SpecialCharTok{\%\textgreater{}\%} 
  \FunctionTok{select}\NormalTok{(calories) }\SpecialCharTok{\%\textgreater{}\%} 
  \FunctionTok{summary}\NormalTok{()}
\end{Highlighting}
\end{Shaded}

\begin{verbatim}
##     calories   
##  Min.   :   0  
##  1st Qu.:1828  
##  Median :2134  
##  Mean   :2304  
##  3rd Qu.:2793  
##  Max.   :4900
\end{verbatim}

\begin{Shaded}
\begin{Highlighting}[]
\FunctionTok{print}\NormalTok{(}\StringTok{"{-}{-}{-}{-}{-}Daily Intensities{-}{-}{-}{-}{-}"}\NormalTok{)}
\end{Highlighting}
\end{Shaded}

\begin{verbatim}
## [1] "-----Daily Intensities-----"
\end{verbatim}

\begin{Shaded}
\begin{Highlighting}[]
\NormalTok{clean\_daily\_intensities }\SpecialCharTok{\%\textgreater{}\%} 
  \FunctionTok{select}\NormalTok{(sedentary\_minutes, lightly\_active\_minutes, fairly\_active\_minutes, very\_active\_minutes) }\SpecialCharTok{\%\textgreater{}\%} 
  \FunctionTok{summary}\NormalTok{()}
\end{Highlighting}
\end{Shaded}

\begin{verbatim}
##  sedentary_minutes lightly_active_minutes fairly_active_minutes
##  Min.   :   0.0    Min.   :  0.0          Min.   :  0.00       
##  1st Qu.: 729.8    1st Qu.:127.0          1st Qu.:  0.00       
##  Median :1057.5    Median :199.0          Median :  6.00       
##  Mean   : 991.2    Mean   :192.8          Mean   : 13.56       
##  3rd Qu.:1229.5    3rd Qu.:264.0          3rd Qu.: 19.00       
##  Max.   :1440.0    Max.   :518.0          Max.   :143.00       
##  very_active_minutes
##  Min.   :  0.00     
##  1st Qu.:  0.00     
##  Median :  4.00     
##  Mean   : 21.16     
##  3rd Qu.: 32.00     
##  Max.   :210.00
\end{verbatim}

\begin{Shaded}
\begin{Highlighting}[]
\NormalTok{clean\_daily\_intensities }\SpecialCharTok{\%\textgreater{}\%} 
  \FunctionTok{select}\NormalTok{(sedentary\_active\_distance) }\SpecialCharTok{\%\textgreater{}\%} 
  \FunctionTok{summary}\NormalTok{()}
\end{Highlighting}
\end{Shaded}

\begin{verbatim}
##  sedentary_active_distance
##  Min.   :0.000000         
##  1st Qu.:0.000000         
##  Median :0.000000         
##  Mean   :0.001606         
##  3rd Qu.:0.000000         
##  Max.   :0.110000
\end{verbatim}

\begin{Shaded}
\begin{Highlighting}[]
\FunctionTok{print}\NormalTok{(}\StringTok{"{-}{-}{-}{-}{-}Daily Steps{-}{-}{-}{-}{-}"}\NormalTok{)}
\end{Highlighting}
\end{Shaded}

\begin{verbatim}
## [1] "-----Daily Steps-----"
\end{verbatim}

\begin{Shaded}
\begin{Highlighting}[]
\NormalTok{clean\_daily\_steps }\SpecialCharTok{\%\textgreater{}\%} 
  \FunctionTok{select}\NormalTok{(total\_steps) }\SpecialCharTok{\%\textgreater{}\%} 
  \FunctionTok{summary}\NormalTok{()}
\end{Highlighting}
\end{Shaded}

\begin{verbatim}
##   total_steps   
##  Min.   :    0  
##  1st Qu.: 3790  
##  Median : 7406  
##  Mean   : 7638  
##  3rd Qu.:10727  
##  Max.   :36019
\end{verbatim}

\begin{Shaded}
\begin{Highlighting}[]
\FunctionTok{print}\NormalTok{(}\StringTok{"{-}{-}{-}{-}{-}Daily Sleep{-}{-}{-}{-}{-}"}\NormalTok{)}
\end{Highlighting}
\end{Shaded}

\begin{verbatim}
## [1] "-----Daily Sleep-----"
\end{verbatim}

\begin{Shaded}
\begin{Highlighting}[]
\NormalTok{clean\_daily\_sleep }\SpecialCharTok{\%\textgreater{}\%} 
  \FunctionTok{select}\NormalTok{(total\_sleep\_records, total\_minutes\_asleep, total\_time\_in\_bed) }\SpecialCharTok{\%\textgreater{}\%} 
  \FunctionTok{summary}\NormalTok{()}
\end{Highlighting}
\end{Shaded}

\begin{verbatim}
##  total_sleep_records total_minutes_asleep total_time_in_bed
##  Min.   :1.00        Min.   : 58.0        Min.   : 61.0    
##  1st Qu.:1.00        1st Qu.:361.0        1st Qu.:403.8    
##  Median :1.00        Median :432.5        Median :463.0    
##  Mean   :1.12        Mean   :419.2        Mean   :458.5    
##  3rd Qu.:1.00        3rd Qu.:490.0        3rd Qu.:526.0    
##  Max.   :3.00        Max.   :796.0        Max.   :961.0
\end{verbatim}

\hypertarget{hourly-dataframes-8}{%
\subparagraph{Hourly DataFrames}\label{hourly-dataframes-8}}

\begin{Shaded}
\begin{Highlighting}[]
\CommentTok{\# Summary of Hourly DataFrames}
\FunctionTok{print}\NormalTok{(}\StringTok{"{-}{-}{-}{-}{-}Hourly Calories{-}{-}{-}{-}{-}"}\NormalTok{)}
\end{Highlighting}
\end{Shaded}

\begin{verbatim}
## [1] "-----Hourly Calories-----"
\end{verbatim}

\begin{Shaded}
\begin{Highlighting}[]
\NormalTok{hourly\_activity\_merged }\SpecialCharTok{\%\textgreater{}\%} 
  \FunctionTok{select}\NormalTok{(calories) }\SpecialCharTok{\%\textgreater{}\%} 
  \FunctionTok{drop\_na}\NormalTok{() }\SpecialCharTok{\%\textgreater{}\%} 
  \FunctionTok{summary}\NormalTok{()}
\end{Highlighting}
\end{Shaded}

\begin{verbatim}
##     calories     
##  Min.   : 42.00  
##  1st Qu.: 63.00  
##  Median : 83.00  
##  Mean   : 97.39  
##  3rd Qu.:108.00  
##  Max.   :948.00
\end{verbatim}

\begin{Shaded}
\begin{Highlighting}[]
\FunctionTok{print}\NormalTok{(}\StringTok{"{-}{-}{-}{-}{-}Hourly Intensities{-}{-}{-}{-}{-}"}\NormalTok{)}
\end{Highlighting}
\end{Shaded}

\begin{verbatim}
## [1] "-----Hourly Intensities-----"
\end{verbatim}

\begin{Shaded}
\begin{Highlighting}[]
\NormalTok{hourly\_activity\_merged }\SpecialCharTok{\%\textgreater{}\%} 
  \FunctionTok{select}\NormalTok{(total\_intensity, average\_intensity) }\SpecialCharTok{\%\textgreater{}\%} 
  \FunctionTok{drop\_na}\NormalTok{() }\SpecialCharTok{\%\textgreater{}\%} 
  \FunctionTok{summary}\NormalTok{()}
\end{Highlighting}
\end{Shaded}

\begin{verbatim}
##  total_intensity  average_intensity
##  Min.   :  0.00   Min.   :0.0000   
##  1st Qu.:  0.00   1st Qu.:0.0000   
##  Median :  3.00   Median :0.0500   
##  Mean   : 12.04   Mean   :0.2006   
##  3rd Qu.: 16.00   3rd Qu.:0.2667   
##  Max.   :180.00   Max.   :3.0000
\end{verbatim}

\begin{Shaded}
\begin{Highlighting}[]
\FunctionTok{print}\NormalTok{(}\StringTok{"{-}{-}{-}{-}{-}Hourly Steps{-}{-}{-}{-}{-}"}\NormalTok{)}
\end{Highlighting}
\end{Shaded}

\begin{verbatim}
## [1] "-----Hourly Steps-----"
\end{verbatim}

\begin{Shaded}
\begin{Highlighting}[]
\NormalTok{hourly\_activity\_merged }\SpecialCharTok{\%\textgreater{}\%} 
  \FunctionTok{select}\NormalTok{(total\_steps) }\SpecialCharTok{\%\textgreater{}\%} 
  \FunctionTok{drop\_na}\NormalTok{() }\SpecialCharTok{\%\textgreater{}\%} 
  \FunctionTok{summary}\NormalTok{()}
\end{Highlighting}
\end{Shaded}

\begin{verbatim}
##   total_steps     
##  Min.   :    0.0  
##  1st Qu.:    0.0  
##  Median :   40.0  
##  Mean   :  320.2  
##  3rd Qu.:  357.0  
##  Max.   :10554.0
\end{verbatim}

\hypertarget{minute-dataframes-8}{%
\subparagraph{Minute DataFrames}\label{minute-dataframes-8}}

\begin{Shaded}
\begin{Highlighting}[]
\CommentTok{\# Summary of Minute DataFrames}
\FunctionTok{print}\NormalTok{(}\StringTok{"{-}{-}{-}{-}{-}Minute Calories{-}{-}{-}{-}{-}"}\NormalTok{)}
\end{Highlighting}
\end{Shaded}

\begin{verbatim}
## [1] "-----Minute Calories-----"
\end{verbatim}

\begin{Shaded}
\begin{Highlighting}[]
\NormalTok{minute\_activity\_merged }\SpecialCharTok{\%\textgreater{}\%} 
  \FunctionTok{select}\NormalTok{(calories) }\SpecialCharTok{\%\textgreater{}\%} 
  \FunctionTok{drop\_na}\NormalTok{() }\SpecialCharTok{\%\textgreater{}\%} 
  \FunctionTok{summary}\NormalTok{()}
\end{Highlighting}
\end{Shaded}

\begin{verbatim}
##     calories      
##  Min.   : 0.0000  
##  1st Qu.: 0.9357  
##  Median : 1.2176  
##  Mean   : 1.6231  
##  3rd Qu.: 1.4327  
##  Max.   :19.7499
\end{verbatim}

\begin{Shaded}
\begin{Highlighting}[]
\FunctionTok{print}\NormalTok{(}\StringTok{"{-}{-}{-}{-}{-}Minute Intensities{-}{-}{-}{-}{-}"}\NormalTok{)}
\end{Highlighting}
\end{Shaded}

\begin{verbatim}
## [1] "-----Minute Intensities-----"
\end{verbatim}

\begin{Shaded}
\begin{Highlighting}[]
\NormalTok{minute\_activity\_merged }\SpecialCharTok{\%\textgreater{}\%} 
  \FunctionTok{select}\NormalTok{(intensity, mets, sleep\_state) }\SpecialCharTok{\%\textgreater{}\%} 
  \FunctionTok{drop\_na}\NormalTok{() }\SpecialCharTok{\%\textgreater{}\%} 
  \FunctionTok{summary}\NormalTok{()}
\end{Highlighting}
\end{Shaded}

\begin{verbatim}
##    intensity            mets        sleep_state 
##  Min.   :0.00000   Min.   : 0.00   Min.   :1.0  
##  1st Qu.:0.00000   1st Qu.:10.00   1st Qu.:1.0  
##  Median :0.00000   Median :10.00   Median :1.0  
##  Mean   :0.01395   Mean   :10.34   Mean   :1.1  
##  3rd Qu.:0.00000   3rd Qu.:10.00   3rd Qu.:1.0  
##  Max.   :3.00000   Max.   :90.00   Max.   :3.0
\end{verbatim}

\begin{Shaded}
\begin{Highlighting}[]
\FunctionTok{print}\NormalTok{(}\StringTok{"{-}{-}{-}{-}{-}Minute Steps{-}{-}{-}{-}{-}"}\NormalTok{)}
\end{Highlighting}
\end{Shaded}

\begin{verbatim}
## [1] "-----Minute Steps-----"
\end{verbatim}

\begin{Shaded}
\begin{Highlighting}[]
\NormalTok{minute\_activity\_merged }\SpecialCharTok{\%\textgreater{}\%} 
  \FunctionTok{select}\NormalTok{(total\_steps) }\SpecialCharTok{\%\textgreater{}\%} 
  \FunctionTok{drop\_na}\NormalTok{() }\SpecialCharTok{\%\textgreater{}\%} 
  \FunctionTok{summary}\NormalTok{()}
\end{Highlighting}
\end{Shaded}

\begin{verbatim}
##   total_steps     
##  Min.   :  0.000  
##  1st Qu.:  0.000  
##  Median :  0.000  
##  Mean   :  5.336  
##  3rd Qu.:  0.000  
##  Max.   :220.000
\end{verbatim}

\hypertarget{removing-0-total-steps-and-1440-sedentary-minutes}{%
\paragraph{3.7.1 Removing 0 Total Steps and 1,440 Sedentary
Minutes}\label{removing-0-total-steps-and-1440-sedentary-minutes}}

The clean\_daily\_steps summary has revealed that there were 0 steps
taken throughout a day. Furthermore, in the clean\_daily\_intensities
dataframe, 1,440 minutes (equivalent to 24 hours) of sedentary time was
recorded. This might suggest that some participants may have activated
the device but not worn it. This idea is further investigated by
comparing the sedentary\_minute and total\_steps columns.

\begin{Shaded}
\begin{Highlighting}[]
\CommentTok{\# Comparing the total\_steps and sedentary\_minutes}
\NormalTok{daily\_activity\_merged }\SpecialCharTok{\%\textgreater{}\%} 
  \FunctionTok{group\_by}\NormalTok{(id) }\SpecialCharTok{\%\textgreater{}\%} 
  \FunctionTok{filter}\NormalTok{(sedentary\_minutes }\SpecialCharTok{==} \DecValTok{1440} \SpecialCharTok{\&}\NormalTok{ total\_steps }\SpecialCharTok{==} \DecValTok{0}\NormalTok{) }\SpecialCharTok{\%\textgreater{}\%} 
  \FunctionTok{select}\NormalTok{(activity\_date\_ymd, sedentary\_minutes, total\_steps) }\SpecialCharTok{\%\textgreater{}\%} 
  \FunctionTok{print}\NormalTok{(}\AttributeTok{n=}\DecValTok{100}\NormalTok{)}
\end{Highlighting}
\end{Shaded}

\begin{verbatim}
## Adding missing grouping variables: `id`
\end{verbatim}

\begin{verbatim}
## # A tibble: 72 x 4
## # Groups:   id [15]
##            id activity_date_ymd sedentary_minutes total_steps
##         <dbl> <date>                        <dbl>       <dbl>
##  1 1503960366 2016-05-12                     1440           0
##  2 1844505072 2016-04-24                     1440           0
##  3 1844505072 2016-04-25                     1440           0
##  4 1844505072 2016-04-26                     1440           0
##  5 1844505072 2016-05-02                     1440           0
##  6 1844505072 2016-05-07                     1440           0
##  7 1844505072 2016-05-08                     1440           0
##  8 1844505072 2016-05-09                     1440           0
##  9 1844505072 2016-05-10                     1440           0
## 10 1844505072 2016-05-11                     1440           0
## 11 1927972279 2016-04-16                     1440           0
## 12 1927972279 2016-04-17                     1440           0
## 13 1927972279 2016-04-19                     1440           0
## 14 1927972279 2016-04-20                     1440           0
## 15 1927972279 2016-04-21                     1440           0
## 16 1927972279 2016-04-27                     1440           0
## 17 1927972279 2016-04-29                     1440           0
## 18 1927972279 2016-04-30                     1440           0
## 19 1927972279 2016-05-05                     1440           0
## 20 1927972279 2016-05-08                     1440           0
## 21 1927972279 2016-05-09                     1440           0
## 22 1927972279 2016-05-10                     1440           0
## 23 1927972279 2016-05-11                     1440           0
## 24 4020332650 2016-04-13                     1440           0
## 25 4020332650 2016-04-19                     1440           0
## 26 4020332650 2016-04-20                     1440           0
## 27 4020332650 2016-04-21                     1440           0
## 28 4020332650 2016-04-22                     1440           0
## 29 4020332650 2016-04-23                     1440           0
## 30 4020332650 2016-04-24                     1440           0
## 31 4020332650 2016-04-25                     1440           0
## 32 4020332650 2016-04-26                     1440           0
## 33 4020332650 2016-04-27                     1440           0
## 34 4020332650 2016-04-28                     1440           0
## 35 4020332650 2016-04-29                     1440           0
## 36 4020332650 2016-04-30                     1440           0
## 37 4020332650 2016-05-01                     1440           0
## 38 4057192912 2016-04-14                     1440           0
## 39 4702921684 2016-05-01                     1440           0
## 40 5577150313 2016-05-07                     1440           0
## 41 5577150313 2016-05-08                     1440           0
## 42 6117666160 2016-04-12                     1440           0
## 43 6117666160 2016-04-13                     1440           0
## 44 6117666160 2016-04-14                     1440           0
## 45 6117666160 2016-04-25                     1440           0
## 46 6117666160 2016-05-03                     1440           0
## 47 6290855005 2016-04-21                     1440           0
## 48 6290855005 2016-04-26                     1440           0
## 49 6290855005 2016-04-29                     1440           0
## 50 6290855005 2016-05-10                     1440           0
## 51 6775888955 2016-04-12                     1440           0
## 52 6775888955 2016-04-19                     1440           0
## 53 6775888955 2016-04-21                     1440           0
## 54 6775888955 2016-04-23                     1440           0
## 55 6775888955 2016-04-27                     1440           0
## 56 6775888955 2016-04-29                     1440           0
## 57 6775888955 2016-05-02                     1440           0
## 58 6775888955 2016-05-04                     1440           0
## 59 6775888955 2016-05-05                     1440           0
## 60 7007744171 2016-05-04                     1440           0
## 61 7086361926 2016-04-17                     1440           0
## 62 8253242879 2016-04-30                     1440           0
## 63 8583815059 2016-05-12                     1440           0
## 64 8792009665 2016-04-17                     1440           0
## 65 8792009665 2016-04-18                     1440           0
## 66 8792009665 2016-04-19                     1440           0
## 67 8792009665 2016-04-25                     1440           0
## 68 8792009665 2016-05-05                     1440           0
## 69 8792009665 2016-05-06                     1440           0
## 70 8792009665 2016-05-07                     1440           0
## 71 8792009665 2016-05-08                     1440           0
## 72 8792009665 2016-05-09                     1440           0
\end{verbatim}

\hypertarget{observation-5}{%
\subparagraph{Observation}\label{observation-5}}

It appears that the 0 number of total steps and 1,440 minutes of
sedentary time are related.

\hypertarget{actions-taken-2}{%
\subparagraph{Actions Taken}\label{actions-taken-2}}

Data entries showing 0 total steps and 1,440 minutes of sedentary time
will be excluded from the analysis as they indicate inactivity, which
could skew the results.

\begin{Shaded}
\begin{Highlighting}[]
\CommentTok{\# Removing 0 total\_steps and 1,440 minutes of sedentary time}
\NormalTok{daily\_activity\_merged }\OtherTok{\textless{}{-}}\NormalTok{ daily\_activity\_merged[daily\_activity\_merged}\SpecialCharTok{$}\NormalTok{total\_steps }\SpecialCharTok{\textgreater{}} \DecValTok{0} \SpecialCharTok{\&}\NormalTok{ daily\_activity\_merged}\SpecialCharTok{$}\NormalTok{sedentary\_minutes }\SpecialCharTok{\textless{}} \DecValTok{1400}\NormalTok{,]}
\FunctionTok{head}\NormalTok{(daily\_activity\_merged)}
\end{Highlighting}
\end{Shaded}

\begin{verbatim}
##            id activity_date_ymd   weekday activity_date.x total_steps
## NA         NA              <NA>      <NA>            <NA>          NA
## 2  1503960366        2016-04-12   Tuesday       4/12/2016       13162
## 3  1503960366        2016-04-13 Wednesday       4/13/2016       10735
## 4  1503960366        2016-04-14  Thursday       4/14/2016       10460
## 5  1503960366        2016-04-15    Friday       4/15/2016        9762
## 6  1503960366        2016-04-16  Saturday       4/16/2016       12669
##    total_distance tracker_distance logged_activities_distance
## NA             NA               NA                         NA
## 2            8.50             8.50                          0
## 3            6.97             6.97                          0
## 4            6.74             6.74                          0
## 5            6.28             6.28                          0
## 6            8.16             8.16                          0
##    very_active_distance moderately_active_distance light_active_distance
## NA                   NA                         NA                    NA
## 2                  1.88                       0.55                  6.06
## 3                  1.57                       0.69                  4.71
## 4                  2.44                       0.40                  3.91
## 5                  2.14                       1.26                  2.83
## 6                  2.71                       0.41                  5.04
##    sedentary_active_distance very_active_minutes fairly_active_minutes
## NA                        NA                  NA                    NA
## 2                          0                  25                    13
## 3                          0                  21                    19
## 4                          0                  30                    11
## 5                          0                  29                    34
## 6                          0                  36                    10
##    lightly_active_minutes sedentary_minutes calories       activity_date.y
## NA                     NA                NA       NA                  <NA>
## 2                     328               728     1985 4/13/2016 12:00:00 AM
## 3                     217               776     1797                  <NA>
## 4                     181              1218     1776 4/15/2016 12:00:00 AM
## 5                     209               726     1745 4/16/2016 12:00:00 AM
## 6                     221               773     1863 4/17/2016 12:00:00 AM
##    total_sleep_records total_minutes_asleep total_time_in_bed
## NA                  NA                   NA                NA
## 2                    2                  384               407
## 3                   NA                   NA                NA
## 4                    1                  412               442
## 5                    2                  340               367
## 6                    1                  700               712
\end{verbatim}

\hypertarget{accurate-met-values}{%
\paragraph{3.7.2 Accurate MET Values}\label{accurate-met-values}}

The MET values appear to be too high to be correct. According to the
Fitbase Data Dictionary, all MET values exported from Fitbase are
multiplied by 10. Therefore, to obtain accurate MET values, they will be
divided by 10.

\hypertarget{getting-accurate-met-values}{%
\subsection{Getting accurate MET
values}\label{getting-accurate-met-values}}

\begin{Shaded}
\begin{Highlighting}[]
\NormalTok{clean\_minute\_METs }\OtherTok{\textless{}{-}} \FunctionTok{mutate}\NormalTok{(clean\_minute\_METs, }\AttributeTok{mets10 =}\NormalTok{ mets}\SpecialCharTok{/}\DecValTok{10}\NormalTok{)}


\CommentTok{\# Updating Merged Minute DataFrames}
\NormalTok{minute\_activity\_m }\OtherTok{\textless{}{-}} \FunctionTok{merge}\NormalTok{(clean\_minute\_calories, clean\_minute\_intensities, }\AttributeTok{by=}\FunctionTok{c}\NormalTok{(}\StringTok{"id"}\NormalTok{, }\StringTok{"activity\_minute"}\NormalTok{, }\StringTok{"activity\_minute\_ymdhms"}\NormalTok{), }\AttributeTok{all =} \ConstantTok{TRUE}\NormalTok{, }\AttributeTok{no.dups =} \ConstantTok{TRUE}\NormalTok{)}
\NormalTok{minute\_activity\_m1 }\OtherTok{\textless{}{-}} \FunctionTok{merge}\NormalTok{(minute\_activity\_m, clean\_minute\_METs, }\AttributeTok{by=}\FunctionTok{c}\NormalTok{(}\StringTok{"id"}\NormalTok{, }\StringTok{"activity\_minute"}\NormalTok{, }\StringTok{"activity\_minute\_ymdhms"}\NormalTok{), }\AttributeTok{all =} \ConstantTok{TRUE}\NormalTok{, }\AttributeTok{no.dups =} \ConstantTok{TRUE}\NormalTok{)}
\NormalTok{minute\_activity\_m2 }\OtherTok{\textless{}{-}} \FunctionTok{merge}\NormalTok{(minute\_activity\_m1, clean\_minute\_sleep, }\AttributeTok{by=}\FunctionTok{c}\NormalTok{(}\StringTok{"id"}\NormalTok{, }\StringTok{"activity\_minute"}\NormalTok{, }\StringTok{"activity\_minute\_ymdhms"}\NormalTok{), }\AttributeTok{all =} \ConstantTok{TRUE}\NormalTok{, }\AttributeTok{no.dups =} \ConstantTok{TRUE}\NormalTok{)}
\NormalTok{minute\_activity\_merged }\OtherTok{\textless{}{-}} \FunctionTok{merge}\NormalTok{(minute\_activity\_m2, clean\_minute\_steps, }\AttributeTok{by=}\FunctionTok{c}\NormalTok{(}\StringTok{"id"}\NormalTok{, }\StringTok{"activity\_minute"}\NormalTok{, }\StringTok{"activity\_minute\_ymdhms"}\NormalTok{), }\AttributeTok{all =} \ConstantTok{TRUE}\NormalTok{, }\AttributeTok{no.dups =} \ConstantTok{TRUE}\NormalTok{)}

\FunctionTok{head}\NormalTok{(clean\_minute\_METs)}
\end{Highlighting}
\end{Shaded}

\begin{verbatim}
## # A tibble: 6 x 5
##           id activity_minute        mets activity_minute_ymdhms mets10
##        <dbl> <chr>                 <dbl> <dttm>                  <dbl>
## 1 1503960366 4/12/2016 12:00:00 AM    10 2016-04-12 00:00:00       1  
## 2 1503960366 4/12/2016 12:01:00 AM    10 2016-04-12 00:01:00       1  
## 3 1503960366 4/12/2016 12:02:00 AM    10 2016-04-12 00:02:00       1  
## 4 1503960366 4/12/2016 12:03:00 AM    10 2016-04-12 00:03:00       1  
## 5 1503960366 4/12/2016 12:04:00 AM    10 2016-04-12 00:04:00       1  
## 6 1503960366 4/12/2016 12:05:00 AM    12 2016-04-12 00:05:00       1.2
\end{verbatim}

\hypertarget{section-4-analyse-and-share}{%
\subsubsection{Section 4: Analyse and
Share}\label{section-4-analyse-and-share}}

In this section, the cleaned data will be transformed and organised to
identify patterns and to answer key questions relevant to our business
task. The results will be visualised and interpreted to facilitate
data-driven decisions.

\hypertarget{summary-statistics}{%
\paragraph{4.1 Summary Statistics}\label{summary-statistics}}

Presented below are summary statistics of the cleaned data, offering a
comprehensive overview and providing guidance for the upcoming analysis.

\#\#\#\#\#Daily DataFrames

\begin{Shaded}
\begin{Highlighting}[]
\CommentTok{\# Summary of Daily DataFrames}
\FunctionTok{print}\NormalTok{(}\StringTok{"{-}{-}{-}{-}{-}Daily Calories{-}{-}{-}{-}{-}"}\NormalTok{)}
\end{Highlighting}
\end{Shaded}

\begin{verbatim}
## [1] "-----Daily Calories-----"
\end{verbatim}

\begin{Shaded}
\begin{Highlighting}[]
\NormalTok{daily\_activity\_merged }\SpecialCharTok{\%\textgreater{}\%} 
  \FunctionTok{select}\NormalTok{(calories) }\SpecialCharTok{\%\textgreater{}\%} 
  \FunctionTok{drop\_na}\NormalTok{() }\SpecialCharTok{\%\textgreater{}\%} 
  \FunctionTok{summary}\NormalTok{()}
\end{Highlighting}
\end{Shaded}

\begin{verbatim}
##     calories   
##  Min.   :  52  
##  1st Qu.:1861  
##  Median :2225  
##  Mean   :2372  
##  3rd Qu.:2844  
##  Max.   :4900
\end{verbatim}

\begin{Shaded}
\begin{Highlighting}[]
\FunctionTok{print}\NormalTok{(}\StringTok{"{-}{-}{-}{-}{-}Daily Intensities{-}{-}{-}{-}{-}"}\NormalTok{)}
\end{Highlighting}
\end{Shaded}

\begin{verbatim}
## [1] "-----Daily Intensities-----"
\end{verbatim}

\begin{Shaded}
\begin{Highlighting}[]
\NormalTok{daily\_activity\_merged }\SpecialCharTok{\%\textgreater{}\%} 
  \FunctionTok{select}\NormalTok{(sedentary\_minutes, lightly\_active\_minutes, fairly\_active\_minutes, very\_active\_distance) }\SpecialCharTok{\%\textgreater{}\%} 
  \FunctionTok{drop\_na}\NormalTok{() }\SpecialCharTok{\%\textgreater{}\%} 
  \FunctionTok{summary}\NormalTok{()}
\end{Highlighting}
\end{Shaded}

\begin{verbatim}
##  sedentary_minutes lightly_active_minutes fairly_active_minutes
##  Min.   :   0.0    Min.   :  2.0          Min.   :  0.00       
##  1st Qu.: 717.8    1st Qu.:152.0          1st Qu.:  0.00       
##  Median : 991.5    Median :214.0          Median :  8.00       
##  Mean   : 940.5    Mean   :216.5          Mean   : 15.21       
##  3rd Qu.:1174.2    3rd Qu.:275.0          3rd Qu.: 22.00       
##  Max.   :1395.0    Max.   :518.0          Max.   :143.00       
##  very_active_distance
##  Min.   : 0.000      
##  1st Qu.: 0.000      
##  Median : 0.470      
##  Mean   : 1.688      
##  3rd Qu.: 2.340      
##  Max.   :21.920
\end{verbatim}

\begin{Shaded}
\begin{Highlighting}[]
\FunctionTok{print}\NormalTok{(}\StringTok{"{-}{-}{-}{-}{-}Daily Steps{-}{-}{-}{-}{-}"}\NormalTok{)}
\end{Highlighting}
\end{Shaded}

\begin{verbatim}
## [1] "-----Daily Steps-----"
\end{verbatim}

\begin{Shaded}
\begin{Highlighting}[]
\NormalTok{daily\_activity\_merged }\SpecialCharTok{\%\textgreater{}\%} 
  \FunctionTok{select}\NormalTok{(total\_steps) }\SpecialCharTok{\%\textgreater{}\%} 
  \FunctionTok{drop\_na}\NormalTok{() }\SpecialCharTok{\%\textgreater{}\%} 
  \FunctionTok{summary}\NormalTok{()}
\end{Highlighting}
\end{Shaded}

\begin{verbatim}
##   total_steps   
##  Min.   :   17  
##  1st Qu.: 5078  
##  Median : 8198  
##  Mean   : 8518  
##  3rd Qu.:11178  
##  Max.   :36019
\end{verbatim}

\begin{Shaded}
\begin{Highlighting}[]
\FunctionTok{print}\NormalTok{(}\StringTok{"{-}{-}{-}{-}{-}Daily Sleep{-}{-}{-}{-}{-}"}\NormalTok{)}
\end{Highlighting}
\end{Shaded}

\begin{verbatim}
## [1] "-----Daily Sleep-----"
\end{verbatim}

\begin{Shaded}
\begin{Highlighting}[]
\NormalTok{daily\_activity\_merged }\SpecialCharTok{\%\textgreater{}\%} 
  \FunctionTok{select}\NormalTok{(total\_sleep\_records, total\_minutes\_asleep, total\_time\_in\_bed) }\SpecialCharTok{\%\textgreater{}\%} 
  \FunctionTok{drop\_na}\NormalTok{() }\SpecialCharTok{\%\textgreater{}\%} 
  \FunctionTok{summary}\NormalTok{()}
\end{Highlighting}
\end{Shaded}

\begin{verbatim}
##  total_sleep_records total_minutes_asleep total_time_in_bed
##  Min.   :1.000       Min.   : 58.0        Min.   : 61.0    
##  1st Qu.:1.000       1st Qu.:361.0        1st Qu.:406.0    
##  Median :1.000       Median :433.0        Median :463.0    
##  Mean   :1.117       Mean   :419.3        Mean   :458.9    
##  3rd Qu.:1.000       3rd Qu.:490.0        3rd Qu.:526.0    
##  Max.   :3.000       Max.   :796.0        Max.   :961.0
\end{verbatim}

\hypertarget{hourly-dataframes-9}{%
\subparagraph{Hourly DataFrames}\label{hourly-dataframes-9}}

\begin{Shaded}
\begin{Highlighting}[]
\CommentTok{\# Summary of Hourly DataFrames}
\FunctionTok{print}\NormalTok{(}\StringTok{"{-}{-}{-}{-}{-}Hourly Calories{-}{-}{-}{-}{-}"}\NormalTok{)}
\end{Highlighting}
\end{Shaded}

\begin{verbatim}
## [1] "-----Hourly Calories-----"
\end{verbatim}

\begin{Shaded}
\begin{Highlighting}[]
\NormalTok{hourly\_activity\_merged }\SpecialCharTok{\%\textgreater{}\%} 
 \FunctionTok{select}\NormalTok{(calories) }\SpecialCharTok{\%\textgreater{}\%} 
 \FunctionTok{drop\_na}\NormalTok{() }\SpecialCharTok{\%\textgreater{}\%} 
 \FunctionTok{summary}\NormalTok{()}
\end{Highlighting}
\end{Shaded}

\begin{verbatim}
##     calories     
##  Min.   : 42.00  
##  1st Qu.: 63.00  
##  Median : 83.00  
##  Mean   : 97.39  
##  3rd Qu.:108.00  
##  Max.   :948.00
\end{verbatim}

\begin{Shaded}
\begin{Highlighting}[]
\FunctionTok{print}\NormalTok{(}\StringTok{"{-}{-}{-}{-}{-}Hourly Intensities{-}{-}{-}{-}{-}"}\NormalTok{)}
\end{Highlighting}
\end{Shaded}

\begin{verbatim}
## [1] "-----Hourly Intensities-----"
\end{verbatim}

\begin{Shaded}
\begin{Highlighting}[]
\NormalTok{hourly\_activity\_merged }\SpecialCharTok{\%\textgreater{}\%} 
 \FunctionTok{select}\NormalTok{(total\_intensity, average\_intensity) }\SpecialCharTok{\%\textgreater{}\%} 
 \FunctionTok{drop\_na}\NormalTok{() }\SpecialCharTok{\%\textgreater{}\%} 
 \FunctionTok{summary}\NormalTok{()}
\end{Highlighting}
\end{Shaded}

\begin{verbatim}
##  total_intensity  average_intensity
##  Min.   :  0.00   Min.   :0.0000   
##  1st Qu.:  0.00   1st Qu.:0.0000   
##  Median :  3.00   Median :0.0500   
##  Mean   : 12.04   Mean   :0.2006   
##  3rd Qu.: 16.00   3rd Qu.:0.2667   
##  Max.   :180.00   Max.   :3.0000
\end{verbatim}

\begin{Shaded}
\begin{Highlighting}[]
\FunctionTok{print}\NormalTok{(}\StringTok{"{-}{-}{-}{-}{-}Hourly Steps{-}{-}{-}{-}{-}"}\NormalTok{)}
\end{Highlighting}
\end{Shaded}

\begin{verbatim}
## [1] "-----Hourly Steps-----"
\end{verbatim}

\begin{Shaded}
\begin{Highlighting}[]
\NormalTok{hourly\_activity\_merged }\SpecialCharTok{\%\textgreater{}\%} 
 \FunctionTok{select}\NormalTok{(total\_steps) }\SpecialCharTok{\%\textgreater{}\%} 
 \FunctionTok{drop\_na}\NormalTok{() }\SpecialCharTok{\%\textgreater{}\%} 
 \FunctionTok{summary}\NormalTok{()}
\end{Highlighting}
\end{Shaded}

\begin{verbatim}
##   total_steps     
##  Min.   :    0.0  
##  1st Qu.:    0.0  
##  Median :   40.0  
##  Mean   :  320.2  
##  3rd Qu.:  357.0  
##  Max.   :10554.0
\end{verbatim}

\hypertarget{minute-dataframes-9}{%
\subparagraph{Minute DataFrames}\label{minute-dataframes-9}}

\begin{Shaded}
\begin{Highlighting}[]
\CommentTok{\# Summary of Minute DataFrames}
\FunctionTok{print}\NormalTok{(}\StringTok{"{-}{-}{-}{-}{-}Minute Calories{-}{-}{-}{-}{-}"}\NormalTok{)}
\end{Highlighting}
\end{Shaded}

\begin{verbatim}
## [1] "-----Minute Calories-----"
\end{verbatim}

\begin{Shaded}
\begin{Highlighting}[]
\NormalTok{minute\_activity\_merged }\SpecialCharTok{\%\textgreater{}\%} 
 \FunctionTok{select}\NormalTok{(calories) }\SpecialCharTok{\%\textgreater{}\%} 
 \FunctionTok{drop\_na}\NormalTok{() }\SpecialCharTok{\%\textgreater{}\%} 
 \FunctionTok{summary}\NormalTok{()}
\end{Highlighting}
\end{Shaded}

\begin{verbatim}
##     calories      
##  Min.   : 0.0000  
##  1st Qu.: 0.9357  
##  Median : 1.2176  
##  Mean   : 1.6231  
##  3rd Qu.: 1.4327  
##  Max.   :19.7499
\end{verbatim}

\begin{Shaded}
\begin{Highlighting}[]
\FunctionTok{print}\NormalTok{(}\StringTok{"{-}{-}{-}{-}{-}Minute Intensities{-}{-}{-}{-}{-}"}\NormalTok{)}
\end{Highlighting}
\end{Shaded}

\begin{verbatim}
## [1] "-----Minute Intensities-----"
\end{verbatim}

\begin{Shaded}
\begin{Highlighting}[]
\NormalTok{minute\_activity\_merged }\SpecialCharTok{\%\textgreater{}\%} 
 \FunctionTok{select}\NormalTok{(intensity, mets10, sleep\_state) }\SpecialCharTok{\%\textgreater{}\%} 
 \FunctionTok{drop\_na}\NormalTok{() }\SpecialCharTok{\%\textgreater{}\%} 
 \FunctionTok{summary}\NormalTok{()}
\end{Highlighting}
\end{Shaded}

\begin{verbatim}
##    intensity           mets10       sleep_state 
##  Min.   :0.00000   Min.   :0.000   Min.   :1.0  
##  1st Qu.:0.00000   1st Qu.:1.000   1st Qu.:1.0  
##  Median :0.00000   Median :1.000   Median :1.0  
##  Mean   :0.01395   Mean   :1.034   Mean   :1.1  
##  3rd Qu.:0.00000   3rd Qu.:1.000   3rd Qu.:1.0  
##  Max.   :3.00000   Max.   :9.000   Max.   :3.0
\end{verbatim}

\begin{Shaded}
\begin{Highlighting}[]
\FunctionTok{print}\NormalTok{(}\StringTok{"{-}{-}{-}{-}{-}Minute Steps{-}{-}{-}{-}{-}"}\NormalTok{)}
\end{Highlighting}
\end{Shaded}

\begin{verbatim}
## [1] "-----Minute Steps-----"
\end{verbatim}

\begin{Shaded}
\begin{Highlighting}[]
\NormalTok{minute\_activity\_merged }\SpecialCharTok{\%\textgreater{}\%} 
 \FunctionTok{select}\NormalTok{(total\_steps) }\SpecialCharTok{\%\textgreater{}\%} 
 \FunctionTok{drop\_na}\NormalTok{() }\SpecialCharTok{\%\textgreater{}\%} 
 \FunctionTok{summary}\NormalTok{()}
\end{Highlighting}
\end{Shaded}

\begin{verbatim}
##   total_steps     
##  Min.   :  0.000  
##  1st Qu.:  0.000  
##  Median :  0.000  
##  Mean   :  5.336  
##  3rd Qu.:  0.000  
##  Max.   :220.000
\end{verbatim}

\hypertarget{key-findings-summary}{%
\subparagraph{Key Findings Summary:}\label{key-findings-summary}}

\begin{enumerate}
\def\labelenumi{\arabic{enumi}.}
\item
  \textbf{Caloric Burn:} Participants in the study burned an average of
  2,372 calories per day. This range is higher than the typical daily
  caloric burn without exercise, which varies from 1,300 to over 2,000
  calories.
\item
  \textbf{Sedentary Time:} On average, participants were sedentary for
  about 940.5 minutes per day, or approximately 16 hours. This duration
  far exceeds the typical daily sedentary time reported for the American
  population, which is about 7.7 hours. Extended sedentary periods over
  10 hours a day are linked to various health issues. High sedentary
  time in this study could be attributed to factors such as limited
  exercise spaces, predominantly sedentary occupations, and the
  widespread use of televisions and other video devices.
\item
  \textbf{Light Activity:} Participants engaged in light physical
  activities for an average of 216.5 minutes per day, or nearly 4 hours.
\item
  \textbf{Moderate Activity:} The average time spent on moderately
  intense activities was 15.21 minutes per day. This is below the World
  Health Organization's (WHO) recommended minimum of 21-43 minutes per
  day.
\item
  \textbf{Intense Activity:} The average time spent on highly intense
  activities was 23.73 minutes per day, which is above the WHO's minimum
  guideline of 11-21 minutes per day.
\item
  \textbf{Steps:} The average number of steps taken per day was 8,518,
  which is below the widely recommended target of 10,000 steps per day.
\item
  \textbf{Sleep:} Participants averaged about 419.3 minutes of sleep per
  day, which is around 7 hours, meeting the minimum recommended sleep
  duration for adults.
\item
  \textbf{METs Value:} The average Metabolic Equivalent of Task (METs)
  value was 1.034, indicating a high level of sedentary behavior, as
  METs values of 1.5 or lower characterize sedentary activities.
\end{enumerate}

\hypertarget{main-topics}{%
\paragraph{4.2 Main Topics}\label{main-topics}}

Based on the summary statistics, this analysis will focus on:

\begin{enumerate}
\def\labelenumi{\arabic{enumi}.}
\item
  \textbf{Physical Activity and Intensity Levels:} We'll look at how
  active the participants were, ranging from not active at all
  (sedentary) to very active. We'll also examine the intensity of these
  activities, whether they were low, moderate, or high intensity.
\item
  \textbf{Device Usage Patterns:} We'll explore how frequently and at
  what times the participants used the device.
\end{enumerate}

\hypertarget{physical-activity-and-intensity-levels}{%
\paragraph{4.2.1 Physical Activity and Intensity
Levels}\label{physical-activity-and-intensity-levels}}

It would be worth analysing how physically active the users were and
what intensity they used throughout the duration of the study to
discover any patterns that could help establish trends in smart device
usage.

\hypertarget{percentage-of-activity-levels}{%
\subparagraph{Percentage of Activity
Levels}\label{percentage-of-activity-levels}}

Based on the summary statistics, there is a high amount of time spent
being inactive compared to other types of activities. To better
understand this, we'll create a pie chart that shows the exact
percentage of sedentary time compared to other activities.

\begin{Shaded}
\begin{Highlighting}[]
\CommentTok{\# Percentage of Activity Levels}
\NormalTok{activity\_level\_total }\OtherTok{\textless{}{-}}\NormalTok{ daily\_activity\_merged }\SpecialCharTok{\%\textgreater{}\%}
 \FunctionTok{summarise}\NormalTok{(}
   \AttributeTok{sedentary\_minutes\_total =} \FunctionTok{sum}\NormalTok{(daily\_activity\_merged}\SpecialCharTok{$}\NormalTok{sedentary\_minutes, }\AttributeTok{na.rm =} \ConstantTok{TRUE}\NormalTok{),}
   \AttributeTok{fairly\_active\_minutes\_total =} \FunctionTok{sum}\NormalTok{(daily\_activity\_merged}\SpecialCharTok{$}\NormalTok{fairly\_active\_minutes, }\AttributeTok{na.rm =} \ConstantTok{TRUE}\NormalTok{),}
   \AttributeTok{lightly\_active\_minutes\_total =} \FunctionTok{sum}\NormalTok{(daily\_activity\_merged}\SpecialCharTok{$}\NormalTok{lightly\_active\_minutes, }\AttributeTok{na.rm =} \ConstantTok{TRUE}\NormalTok{),}
   \AttributeTok{very\_active\_minutes\_total =} \FunctionTok{sum}\NormalTok{(daily\_activity\_merged}\SpecialCharTok{$}\NormalTok{very\_active\_minutes, }\AttributeTok{na.rm =} \ConstantTok{TRUE}\NormalTok{))}

\NormalTok{activity\_level }\OtherTok{\textless{}{-}} \FunctionTok{c}\NormalTok{(activity\_level\_total}\SpecialCharTok{$}\NormalTok{sedentary\_minutes\_total, activity\_level\_total}\SpecialCharTok{$}\NormalTok{lightly\_active\_minutes\_total, activity\_level\_total}\SpecialCharTok{$}\NormalTok{fairly\_active\_minutes\_total, activity\_level\_total}\SpecialCharTok{$}\NormalTok{very\_active\_minutes\_total)}
\NormalTok{labels }\OtherTok{\textless{}{-}} \FunctionTok{c}\NormalTok{(}\StringTok{"inactive"}\NormalTok{, }\StringTok{"Lightly Active"}\NormalTok{, }\StringTok{"Fairly Active"}\NormalTok{, }\StringTok{"Very Active"}\NormalTok{)}
\NormalTok{colours }\OtherTok{\textless{}{-}} \FunctionTok{c}\NormalTok{(}\StringTok{"\#cebdbf"}\NormalTok{, }\StringTok{"\#e3d4cb"}\NormalTok{, }\StringTok{"\#996855"}\NormalTok{, }\StringTok{"\#e7a589"}\NormalTok{)}
\NormalTok{percentage }\OtherTok{\textless{}{-}}\NormalTok{ (activity\_level}\SpecialCharTok{/}\FunctionTok{sum}\NormalTok{(activity\_level)}\SpecialCharTok{*}\DecValTok{100}\NormalTok{) }\SpecialCharTok{\%\textgreater{}\%} 
 \FunctionTok{round}\NormalTok{(}\DecValTok{2}\NormalTok{)}
\NormalTok{labels }\OtherTok{\textless{}{-}} \FunctionTok{paste}\NormalTok{(labels, percentage)}
\FunctionTok{options}\NormalTok{(}\AttributeTok{repr.plot.width =} \DecValTok{20}\NormalTok{, }\AttributeTok{repr.plot.height =} \DecValTok{700}\NormalTok{)}
\FunctionTok{pie}\NormalTok{(activity\_level, }\AttributeTok{labels=}\FunctionTok{paste}\NormalTok{(labels, }\AttributeTok{sep=}\StringTok{" "}\NormalTok{, }\StringTok{"\%"}\NormalTok{), }\AttributeTok{col =}\NormalTok{ colours, }\AttributeTok{border =} \StringTok{"white"}\NormalTok{, }\AttributeTok{radius =} \DecValTok{1}\NormalTok{, }\AttributeTok{cex =} \FloatTok{0.8}\NormalTok{)}
\FunctionTok{title}\NormalTok{(}\AttributeTok{main=}\StringTok{"Percentage of Activity Levels"}\NormalTok{, }\AttributeTok{cex.main =} \DecValTok{2}\NormalTok{, }\AttributeTok{line =} \DecValTok{2}\NormalTok{)}
\end{Highlighting}
\end{Shaded}

\includegraphics{Bellabeat-Case-Study-R-Project_files/figure-latex/unnamed-chunk-54-1.pdf}

Key Findings

Most participants, about 78.64\%, were inactive during the study, which
is concerning because being inactive for long periods can lead to health
problems like heart disease, diabetes, and some cancers.

Only 3.25\% of the participants actively engaged in exercise (1.27\%
were fairly active and 1.98\% were very active) throughout the 31 days.
This suggests that only a few people used the Fitbit mainly to track
their workouts, while most used it to monitor their daily routines.

It's likely that many of the participants have jobs that require them to
sit a lot or do only light physical activities. According to research by
Church and others, more than 80\% of jobs in the US are sedentary or
involve only light activity.

\hypertarget{correlation-between-activity-levels-and-calories-burnt-per-day}{%
\subparagraph{Correlation between Activity Levels and Calories Burnt per
Day}\label{correlation-between-activity-levels-and-calories-burnt-per-day}}

To explore potential correlations between activity levels and daily
caloric expenditure, a visualisation will be created using ggplot2.

\begin{Shaded}
\begin{Highlighting}[]
\CommentTok{\# Correlation between Activity Level and Calories Burnt}
\NormalTok{p1 }\OtherTok{\textless{}{-}} \FunctionTok{ggplot}\NormalTok{(}\AttributeTok{data =}\NormalTok{ daily\_activity\_merged, }\FunctionTok{aes}\NormalTok{(}\AttributeTok{x =}\NormalTok{ sedentary\_minutes, }\AttributeTok{y =}\NormalTok{ calories)) }\SpecialCharTok{+} \FunctionTok{geom\_point}\NormalTok{(}\AttributeTok{color =} \StringTok{"\#cebdbf"}\NormalTok{) }\SpecialCharTok{+} \FunctionTok{geom\_smooth}\NormalTok{(}\AttributeTok{method =} \StringTok{"lm"}\NormalTok{, }\AttributeTok{color =} \StringTok{"\#996855"}\NormalTok{, }\AttributeTok{fill =} \StringTok{"\#e7a589"}\NormalTok{) }\SpecialCharTok{+} \FunctionTok{xlab}\NormalTok{(}\StringTok{"Sedentary Minutes"}\NormalTok{) }\SpecialCharTok{+} \FunctionTok{ylab}\NormalTok{(}\StringTok{"Calories Burnt"}\NormalTok{)}
\NormalTok{p2 }\OtherTok{\textless{}{-}} \FunctionTok{ggplot}\NormalTok{(}\AttributeTok{data =}\NormalTok{ daily\_activity\_merged, }\FunctionTok{aes}\NormalTok{(}\AttributeTok{x =}\NormalTok{ lightly\_active\_minutes, }\AttributeTok{y =}\NormalTok{ calories)) }\SpecialCharTok{+} \FunctionTok{geom\_point}\NormalTok{(}\AttributeTok{color =} \StringTok{"\#cebdbf"}\NormalTok{) }\SpecialCharTok{+} \FunctionTok{geom\_smooth}\NormalTok{(}\AttributeTok{method =} \StringTok{"lm"}\NormalTok{, }\AttributeTok{color =} \StringTok{"\#996855"}\NormalTok{, }\AttributeTok{fill =} \StringTok{"\#e7a589"}\NormalTok{) }\SpecialCharTok{+} \FunctionTok{xlab}\NormalTok{(}\StringTok{"Lightly Active Minutes"}\NormalTok{) }\SpecialCharTok{+} \FunctionTok{ylab}\NormalTok{(}\StringTok{"Calories Burnt"}\NormalTok{)}
\NormalTok{p3 }\OtherTok{\textless{}{-}} \FunctionTok{ggplot}\NormalTok{(}\AttributeTok{data =}\NormalTok{ daily\_activity\_merged, }\FunctionTok{aes}\NormalTok{(}\AttributeTok{x =}\NormalTok{ fairly\_active\_minutes, }\AttributeTok{y =}\NormalTok{ calories)) }\SpecialCharTok{+} \FunctionTok{geom\_point}\NormalTok{(}\AttributeTok{color =} \StringTok{"\#cebdbf"}\NormalTok{) }\SpecialCharTok{+} \FunctionTok{geom\_smooth}\NormalTok{(}\AttributeTok{method =} \StringTok{"lm"}\NormalTok{, }\AttributeTok{color =} \StringTok{"\#996855"}\NormalTok{, }\AttributeTok{fill =} \StringTok{"\#e7a589"}\NormalTok{) }\SpecialCharTok{+} \FunctionTok{xlab}\NormalTok{(}\StringTok{"Fairly Active Minutes"}\NormalTok{) }\SpecialCharTok{+} \FunctionTok{ylab}\NormalTok{(}\StringTok{"Calories Burnt"}\NormalTok{)}
\NormalTok{p4 }\OtherTok{\textless{}{-}} \FunctionTok{ggplot}\NormalTok{(}\AttributeTok{data =}\NormalTok{ daily\_activity\_merged, }\FunctionTok{aes}\NormalTok{(}\AttributeTok{x =}\NormalTok{ very\_active\_minutes, }\AttributeTok{y =}\NormalTok{ calories)) }\SpecialCharTok{+} \FunctionTok{geom\_point}\NormalTok{(}\AttributeTok{color =} \StringTok{"\#cebdbf"}\NormalTok{) }\SpecialCharTok{+} \FunctionTok{geom\_smooth}\NormalTok{(}\AttributeTok{method =} \StringTok{"lm"}\NormalTok{, }\AttributeTok{color =} \StringTok{"\#996855"}\NormalTok{, }\AttributeTok{fill =} \StringTok{"\#e7a589"}\NormalTok{) }\SpecialCharTok{+} \FunctionTok{xlab}\NormalTok{(}\StringTok{"Very Active Minutes"}\NormalTok{) }\SpecialCharTok{+} \FunctionTok{ylab}\NormalTok{(}\StringTok{"Calories Burnt"}\NormalTok{)}

\FunctionTok{options}\NormalTok{(}\AttributeTok{repr.plot.width =} \DecValTok{20}\NormalTok{, }\AttributeTok{repr.plot.height =} \DecValTok{10}\NormalTok{)}
\FunctionTok{grid.arrange}\NormalTok{(p1, p2, p3, p4,}
            \AttributeTok{nrow =} \DecValTok{1}\NormalTok{,}
            \AttributeTok{top =} \StringTok{"Correlation between Activity Levels and Calories Burnt per Day"}
\NormalTok{)}
\end{Highlighting}
\end{Shaded}

\begin{verbatim}
## `geom_smooth()` using formula = 'y ~ x'
\end{verbatim}

\begin{verbatim}
## Warning: Removed 13 rows containing non-finite outside the scale range
## (`stat_smooth()`).
\end{verbatim}

\begin{verbatim}
## Warning: Removed 13 rows containing missing values or values outside the scale range
## (`geom_point()`).
\end{verbatim}

\begin{verbatim}
## `geom_smooth()` using formula = 'y ~ x'
\end{verbatim}

\begin{verbatim}
## Warning: Removed 13 rows containing non-finite outside the scale range
## (`stat_smooth()`).
## Removed 13 rows containing missing values or values outside the scale range
## (`geom_point()`).
\end{verbatim}

\begin{verbatim}
## `geom_smooth()` using formula = 'y ~ x'
\end{verbatim}

\begin{verbatim}
## Warning: Removed 13 rows containing non-finite outside the scale range
## (`stat_smooth()`).
## Removed 13 rows containing missing values or values outside the scale range
## (`geom_point()`).
\end{verbatim}

\begin{verbatim}
## `geom_smooth()` using formula = 'y ~ x'
\end{verbatim}

\begin{verbatim}
## Warning: Removed 13 rows containing non-finite outside the scale range
## (`stat_smooth()`).
## Removed 13 rows containing missing values or values outside the scale range
## (`geom_point()`).
\end{verbatim}

\includegraphics{Bellabeat-Case-Study-R-Project_files/figure-latex/unnamed-chunk-55-1.pdf}

\hypertarget{key-findings-1}{%
\subparagraph{Key Findings}\label{key-findings-1}}

\begin{itemize}
\tightlist
\item
  The less active a person is, the fewer calories they will burn.
\item
  The more active a person is, the more calories they will burn.
\item
  The correlation coefficient between Very Active Minutes and Burnt
  Calories is 0.61, indicating a moderate level of correlation.
  Additionally, with a p-value close to 0, this correlation is highly
  statistically significant.
\item
  Despite fairly active and very active levels representing the smallest
  categories, individuals in these activity levels burn more calories
  compared to those in sedentary and lightly active levels.
\end{itemize}

\hypertarget{average-hourly-physical-intensity-throughout-the-day}{%
\subparagraph{Average Hourly Physical Intensity Throughout the
Day}\label{average-hourly-physical-intensity-throughout-the-day}}

To discern peak times of physical intensity throughout the day, a bar
chart will be generated.

\begin{Shaded}
\begin{Highlighting}[]
\CommentTok{\#Average Hourly Intensity throughout the Day}
\NormalTok{hourly\_activity\_merged}\SpecialCharTok{$}\NormalTok{activity\_hour\_hms }\OtherTok{\textless{}{-}} \FunctionTok{format}\NormalTok{(hourly\_activity\_merged}\SpecialCharTok{$}\NormalTok{activity\_hour\_ymdhms, }\AttributeTok{format =} \StringTok{"\%H:\%M:\%S"}\NormalTok{)}

\NormalTok{hourly\_activity\_merged }\SpecialCharTok{\%\textgreater{}\%}
 \FunctionTok{group\_by}\NormalTok{(activity\_hour\_hms) }\SpecialCharTok{\%\textgreater{}\%}
 \FunctionTok{summarise}\NormalTok{(}\AttributeTok{average\_hourly\_intensity =} \FunctionTok{mean}\NormalTok{(total\_intensity)) }\SpecialCharTok{\%\textgreater{}\%} 
 \FunctionTok{ggplot}\NormalTok{(}\FunctionTok{aes}\NormalTok{(}\AttributeTok{x =}\NormalTok{ activity\_hour\_hms, }\AttributeTok{y =}\NormalTok{ average\_hourly\_intensity, }\AttributeTok{fill =}\NormalTok{ average\_hourly\_intensity)) }\SpecialCharTok{+}
 \FunctionTok{geom\_col}\NormalTok{() }\SpecialCharTok{+} 
 \FunctionTok{scale\_fill\_gradient}\NormalTok{(}\AttributeTok{low =} \StringTok{"\#cebdbf"}\NormalTok{,}
                     \AttributeTok{high =} \StringTok{"\#9F7175"}\NormalTok{) }\SpecialCharTok{+}
 \FunctionTok{labs}\NormalTok{(}\AttributeTok{x =} \StringTok{"Time of Day"}\NormalTok{, }\AttributeTok{y =}\StringTok{"Average Intensity"}\NormalTok{, }\AttributeTok{title =}\StringTok{"Average Hourly Intensity throughout the Day"}\NormalTok{)}\SpecialCharTok{+}
 \FunctionTok{theme}\NormalTok{(}\AttributeTok{plot.title =} \FunctionTok{element\_text}\NormalTok{(}\AttributeTok{size =} \DecValTok{20}\NormalTok{, }\AttributeTok{hjust=}\FloatTok{0.5}\NormalTok{, }\AttributeTok{face=}\StringTok{\textquotesingle{}bold\textquotesingle{}}\NormalTok{),}
       \AttributeTok{axis.text.x =} \FunctionTok{element\_text}\NormalTok{(}\AttributeTok{angle =} \DecValTok{90}\NormalTok{, }\AttributeTok{size =} \DecValTok{18}\NormalTok{), }
       \AttributeTok{axis.title.x =} \FunctionTok{element\_text}\NormalTok{(}\AttributeTok{size =} \DecValTok{20}\NormalTok{),}
       \AttributeTok{axis.title.y =} \FunctionTok{element\_text}\NormalTok{(}\AttributeTok{size =} \DecValTok{20}\NormalTok{),}
       \AttributeTok{legend.title =} \FunctionTok{element\_text}\NormalTok{(}\AttributeTok{size=}\DecValTok{20}\NormalTok{),}
       \AttributeTok{legend.text =} \FunctionTok{element\_text}\NormalTok{(}\AttributeTok{size=}\DecValTok{18}\NormalTok{)) }\SpecialCharTok{+}
 \FunctionTok{guides}\NormalTok{(}\AttributeTok{fill =} \FunctionTok{guide\_legend}\NormalTok{(}\AttributeTok{title=}\StringTok{"Average Hourly Intensity"}\NormalTok{))}
\end{Highlighting}
\end{Shaded}

\includegraphics{Bellabeat-Case-Study-R-Project_files/figure-latex/unnamed-chunk-56-1.pdf}

\begin{Shaded}
\begin{Highlighting}[]
\FunctionTok{options}\NormalTok{(}\AttributeTok{repr.plot.width =} \DecValTok{20}\NormalTok{, }\AttributeTok{repr.plot.height =} \DecValTok{12}\NormalTok{)}
\end{Highlighting}
\end{Shaded}

\hypertarget{key-findings-2}{%
\subparagraph{Key Findings}\label{key-findings-2}}

\begin{itemize}
\tightlist
\item
  The lowest physical intensities occur from 23:00 to 5:00,
  corresponding to typical sleeping hour.
\item
  Peak intensity is observed from 17:00 to 19:00, coinciding with the
  time when people usually return home from work.
\item
  Another peak in intensity is observed from 12:00 to 14:00, indicative
  of lunch breaks.
\end{itemize}

\hypertarget{average-physical-intensity-throughout-the-week}{%
\subparagraph{Average Physical Intensity Throughout the
Week}\label{average-physical-intensity-throughout-the-week}}

To determine the most active day of the week, a weekly bar chart was
generated.

\begin{Shaded}
\begin{Highlighting}[]
\CommentTok{\#Average Intensity throughout the Week}
\NormalTok{hourly\_activity\_merged}\SpecialCharTok{$}\NormalTok{weekday }\OtherTok{\textless{}{-}} \FunctionTok{factor}\NormalTok{(hourly\_activity\_merged}\SpecialCharTok{$}\NormalTok{weekday, }\AttributeTok{levels=}\FunctionTok{c}\NormalTok{(}\StringTok{"Monday"}\NormalTok{, }\StringTok{"Tuesday"}\NormalTok{, }\StringTok{"Wednesday"}\NormalTok{, }\StringTok{"Thursday"}\NormalTok{, }\StringTok{"Friday"}\NormalTok{, }\StringTok{"Saturday"}\NormalTok{, }\StringTok{"Sunday"}\NormalTok{))}

\NormalTok{hourly\_activity\_merged }\SpecialCharTok{\%\textgreater{}\%}
 \FunctionTok{group\_by}\NormalTok{(weekday) }\SpecialCharTok{\%\textgreater{}\%}
 \FunctionTok{summarise}\NormalTok{(}\AttributeTok{average\_intensity =} \FunctionTok{mean}\NormalTok{(total\_intensity)) }\SpecialCharTok{\%\textgreater{}\%} 
 \FunctionTok{ggplot}\NormalTok{(}\FunctionTok{aes}\NormalTok{(}\AttributeTok{x =}\NormalTok{ weekday, }\AttributeTok{y =}\NormalTok{ average\_intensity, }\AttributeTok{fill =}\NormalTok{ average\_intensity)) }\SpecialCharTok{+}
 \FunctionTok{geom\_col}\NormalTok{() }\SpecialCharTok{+} 
 \FunctionTok{scale\_fill\_gradient}\NormalTok{(}\AttributeTok{low =} \StringTok{"\#cebdbf"}\NormalTok{,}
                     \AttributeTok{high =} \StringTok{"\#9F7175"}\NormalTok{) }\SpecialCharTok{+}
 \FunctionTok{labs}\NormalTok{(}\AttributeTok{x =} \StringTok{"Day of Week"}\NormalTok{, }\AttributeTok{y =}\StringTok{"Average Intensity"}\NormalTok{, }\AttributeTok{title =}\StringTok{"Average Intensity throughout the Week"}\NormalTok{)}\SpecialCharTok{+}
 \FunctionTok{guides}\NormalTok{(}\AttributeTok{fill =} \FunctionTok{guide\_legend}\NormalTok{(}\AttributeTok{title=}\StringTok{"Average Intensity"}\NormalTok{)) }\SpecialCharTok{+}
 \FunctionTok{theme}\NormalTok{(}\AttributeTok{plot.title =} \FunctionTok{element\_text}\NormalTok{(}\AttributeTok{hjust =} \FloatTok{0.5}\NormalTok{, }\AttributeTok{vjust =} \FloatTok{0.8}\NormalTok{, }\AttributeTok{size =} \DecValTok{20}\NormalTok{, }\AttributeTok{face =} \StringTok{\textquotesingle{}bold\textquotesingle{}}\NormalTok{)) }\SpecialCharTok{+}
 \FunctionTok{theme}\NormalTok{(}\AttributeTok{axis.text.x =} \FunctionTok{element\_text}\NormalTok{(}\AttributeTok{angle =} \DecValTok{90}\NormalTok{, , }\AttributeTok{size =} \DecValTok{14}\NormalTok{)) }\SpecialCharTok{+}
 \FunctionTok{theme}\NormalTok{(}\AttributeTok{axis.title.x =} \FunctionTok{element\_text}\NormalTok{(}\AttributeTok{size =} \DecValTok{16}\NormalTok{),}
       \AttributeTok{axis.title.y =} \FunctionTok{element\_text}\NormalTok{(}\AttributeTok{size =} \DecValTok{16}\NormalTok{),}
       \AttributeTok{legend.title =} \FunctionTok{element\_text}\NormalTok{(}\AttributeTok{size=}\DecValTok{20}\NormalTok{),}
       \AttributeTok{legend.text =} \FunctionTok{element\_text}\NormalTok{(}\AttributeTok{size=}\DecValTok{18}\NormalTok{))}
\end{Highlighting}
\end{Shaded}

\includegraphics{Bellabeat-Case-Study-R-Project_files/figure-latex/unnamed-chunk-57-1.pdf}

\hypertarget{key-findings-3}{%
\subparagraph{Key Findings}\label{key-findings-3}}

\begin{itemize}
\tightlist
\item
  Saturday emerges as the most active day, likely due to more free time.
\item
  An unexpected peak in activity is observed on Tuesday, while intensity
  levels on other weekdays reamin relatively consistent.
\item
  Sunday shows the lowest activity levels, possibly due to individuals
  taking rest before the start of the workweek.
\end{itemize}

\hypertarget{average-hourly-physical-intensity-throughout-the-week}{%
\subparagraph{Average Hourly Physical Intensity Throughout the
Week}\label{average-hourly-physical-intensity-throughout-the-week}}

Below, the average hourly intensity for each day of the week is
presented, offering further insights into intensity levels over the
week.

\begin{Shaded}
\begin{Highlighting}[]
\CommentTok{\# Average Hourly Intensity throughout the Week}
\NormalTok{hourly\_activity\_merged }\SpecialCharTok{\%\textgreater{}\%}
 \FunctionTok{group\_by}\NormalTok{(weekday, activity\_hour\_hms) }\SpecialCharTok{\%\textgreater{}\%}
 \FunctionTok{summarise}\NormalTok{(}\AttributeTok{average\_hourly\_intensity =} \FunctionTok{mean}\NormalTok{(total\_intensity)) }\SpecialCharTok{\%\textgreater{}\%} 
 \FunctionTok{ggplot}\NormalTok{(}\FunctionTok{aes}\NormalTok{(}\AttributeTok{x =}\NormalTok{ activity\_hour\_hms, }\AttributeTok{y =}\NormalTok{ weekday, }\AttributeTok{fill =}\NormalTok{ average\_hourly\_intensity)) }\SpecialCharTok{+}
 \FunctionTok{theme}\NormalTok{(}\AttributeTok{axis.text.x =} \FunctionTok{element\_text}\NormalTok{(}\AttributeTok{angle=}\DecValTok{90}\NormalTok{)) }\SpecialCharTok{+}
 \FunctionTok{scale\_fill\_continuous}\NormalTok{(}\AttributeTok{low=}\StringTok{"white"}\NormalTok{, }\AttributeTok{high =} \StringTok{"\#9F7175"}\NormalTok{) }\SpecialCharTok{+}
 \FunctionTok{geom\_tile}\NormalTok{(}\AttributeTok{colour =} \StringTok{"white"}\NormalTok{, }\AttributeTok{lwd=}\NormalTok{ .}\DecValTok{5}\NormalTok{, }\AttributeTok{linetype =} \DecValTok{1}\NormalTok{) }\SpecialCharTok{+} 
 \FunctionTok{coord\_fixed}\NormalTok{() }\SpecialCharTok{+} 
 \FunctionTok{labs}\NormalTok{(}\AttributeTok{x =} \StringTok{"Time of Day"}\NormalTok{, }\AttributeTok{y =}\StringTok{"Weekday"}\NormalTok{, }\AttributeTok{title =}\StringTok{"Average Hourly Intensity throughout the Week"}\NormalTok{, }\AttributeTok{fill =} \StringTok{"Average Hourly Intensity"}\NormalTok{) }\SpecialCharTok{+}
 \FunctionTok{theme}\NormalTok{(}\AttributeTok{plot.title =} \FunctionTok{element\_text}\NormalTok{(}\AttributeTok{hjust =} \FloatTok{0.5}\NormalTok{, }\AttributeTok{vjust =} \FloatTok{0.8}\NormalTok{, }\AttributeTok{size =} \DecValTok{15}\NormalTok{, }\AttributeTok{face =} \StringTok{\textquotesingle{}bold\textquotesingle{}}\NormalTok{), }\AttributeTok{panel.background =} \FunctionTok{element\_blank}\NormalTok{()) }\SpecialCharTok{+}
 \FunctionTok{theme}\NormalTok{(}\AttributeTok{axis.text.x =} \FunctionTok{element\_text}\NormalTok{(}\AttributeTok{angle =} \DecValTok{90}\NormalTok{, , }\AttributeTok{size =} \DecValTok{14}\NormalTok{)) }\SpecialCharTok{+}
 \FunctionTok{theme}\NormalTok{(}\AttributeTok{axis.title.x =} \FunctionTok{element\_text}\NormalTok{(}\AttributeTok{size =} \DecValTok{16}\NormalTok{),}
       \AttributeTok{axis.title.y =} \FunctionTok{element\_text}\NormalTok{(}\AttributeTok{size =} \DecValTok{16}\NormalTok{),}
       \AttributeTok{legend.title =} \FunctionTok{element\_text}\NormalTok{(}\AttributeTok{size=}\DecValTok{20}\NormalTok{),}
       \AttributeTok{legend.text =} \FunctionTok{element\_text}\NormalTok{(}\AttributeTok{size=}\DecValTok{10}\NormalTok{))}
\end{Highlighting}
\end{Shaded}

\begin{verbatim}
## `summarise()` has grouped output by 'weekday'. You can override using the
## `.groups` argument.
\end{verbatim}

\includegraphics{Bellabeat-Case-Study-R-Project_files/figure-latex/unnamed-chunk-58-1.pdf}

\hypertarget{key-findings-4}{%
\subparagraph{Key Findings}\label{key-findings-4}}

\begin{itemize}
\tightlist
\item
  Consistent with earlier observations, intensity peaks from 17:00 to
  19:00, post-work hours.
\item
  The midday peak from 12:00 to 14:00 on weekdays (Monday - Friday) is
  less pronounced than the previous graph had shown. The peak during
  that period actually comes from high activity and intensity levels on
  Saturday when people have more time to workout or do other activities
  that require higher intensities.
\item
  Users go to bed and wake up later during weekends.
\end{itemize}

\hypertarget{device-usage-levels}{%
\paragraph{4.2.2 Device Usage Levels}\label{device-usage-levels}}

Daily usage of the device will be examined to understand how users use
the device and how often they use it.

\hypertarget{daily-device-usage-percentage}{%
\subparagraph{Daily Device Usage
Percentage}\label{daily-device-usage-percentage}}

To determine device usage frequency, participants will be categorized
based on the number of days they used the device. The resulting
percentages will be visualised using a pie chart.

\begin{Shaded}
\begin{Highlighting}[]
\NormalTok{activity\_usage }\OtherTok{\textless{}{-}}\NormalTok{ daily\_activity\_merged }\SpecialCharTok{\%\textgreater{}\%} 
 \FunctionTok{group\_by}\NormalTok{(id) }\SpecialCharTok{\%\textgreater{}\%} 
 \FunctionTok{summarize}\NormalTok{(}\AttributeTok{days\_used =} \FunctionTok{sum}\NormalTok{(}\FunctionTok{n}\NormalTok{())) }\SpecialCharTok{\%\textgreater{}\%} 
 \FunctionTok{mutate}\NormalTok{(}\AttributeTok{daily\_usage\_level =} \FunctionTok{case\_when}\NormalTok{(}
\NormalTok{   days\_used }\SpecialCharTok{\textgreater{}=} \DecValTok{1} \SpecialCharTok{\&}\NormalTok{ days\_used }\SpecialCharTok{\textless{}=} \DecValTok{10} \SpecialCharTok{\textasciitilde{}} \StringTok{\textquotesingle{}Low Usage\textquotesingle{}}\NormalTok{,}
\NormalTok{   days\_used }\SpecialCharTok{\textgreater{}} \DecValTok{10} \SpecialCharTok{\&}\NormalTok{ days\_used }\SpecialCharTok{\textless{}=} \DecValTok{20} \SpecialCharTok{\textasciitilde{}} \StringTok{\textquotesingle{}Moderate Usage\textquotesingle{}}\NormalTok{,}
\NormalTok{   days\_used }\SpecialCharTok{\textgreater{}} \DecValTok{20} \SpecialCharTok{\textasciitilde{}} \StringTok{\textquotesingle{}High Usage\textquotesingle{}}
\NormalTok{ )) }\SpecialCharTok{\%\textgreater{}\%} 
 \FunctionTok{drop\_na}\NormalTok{() }\SpecialCharTok{\%\textgreater{}\%} 
 \FunctionTok{group\_by}\NormalTok{(daily\_usage\_level) }\SpecialCharTok{\%\textgreater{}\%} 
 \FunctionTok{summarize}\NormalTok{(}\AttributeTok{users\_total =} \FunctionTok{n}\NormalTok{())}

\FunctionTok{head}\NormalTok{(activity\_usage)}
\end{Highlighting}
\end{Shaded}

\begin{verbatim}
## # A tibble: 3 x 2
##   daily_usage_level users_total
##   <chr>                   <int>
## 1 High Usage                 24
## 2 Low Usage                   1
## 3 Moderate Usage              8
\end{verbatim}

\begin{Shaded}
\begin{Highlighting}[]
\CommentTok{\# Daily Device Usage Percentage}
\NormalTok{slices\_v2 }\OtherTok{\textless{}{-}} \FunctionTok{c}\NormalTok{(}\DecValTok{1}\NormalTok{, }\DecValTok{8}\NormalTok{, }\DecValTok{24}\NormalTok{)}
\NormalTok{labels\_v2 }\OtherTok{\textless{}{-}} \FunctionTok{c}\NormalTok{(}\StringTok{"Low Usage"}\NormalTok{, }\StringTok{"Moderate Usage"}\NormalTok{, }\StringTok{"High Usage"}\NormalTok{)}
\NormalTok{pct }\OtherTok{\textless{}{-}}\NormalTok{ slices\_v2}\SpecialCharTok{/}\FunctionTok{sum}\NormalTok{(slices\_v2)}\SpecialCharTok{*}\DecValTok{100}
\NormalTok{pct }\OtherTok{\textless{}{-}} \FunctionTok{round}\NormalTok{(pct, }\DecValTok{2}\NormalTok{)}
\NormalTok{labels\_v2 }\OtherTok{\textless{}{-}} \FunctionTok{paste}\NormalTok{(labels\_v2, pct)}
\NormalTok{labels\_v2 }\OtherTok{\textless{}{-}} \FunctionTok{paste}\NormalTok{(labels\_v2, }\StringTok{"\%"}\NormalTok{, }\AttributeTok{sep =} \StringTok{""}\NormalTok{)}
\NormalTok{colours }\OtherTok{\textless{}{-}} \FunctionTok{c}\NormalTok{(}\StringTok{"\#996855"}\NormalTok{, }\StringTok{"\#e7a589"}\NormalTok{, }\StringTok{"\#cebdbf"}\NormalTok{)}
\FunctionTok{options}\NormalTok{(}\AttributeTok{repr.plot.width =} \DecValTok{20}\NormalTok{, }\AttributeTok{repr.plot.height =} \DecValTok{13}\NormalTok{)}
\FunctionTok{pie}\NormalTok{(slices\_v2, }\AttributeTok{labels =}\NormalTok{ labels\_v2,}
    \AttributeTok{main=}\StringTok{"Daily Usage Levels"}\NormalTok{,}
    \AttributeTok{col =}\NormalTok{ colours, }
    \AttributeTok{border =} \StringTok{"white"}\NormalTok{, }
    \AttributeTok{cex.main =} \DecValTok{2}\NormalTok{,}
    \AttributeTok{cex =} \FloatTok{0.8}\NormalTok{,}
    \AttributeTok{radius =} \DecValTok{1}\NormalTok{)}
\FunctionTok{legend}\NormalTok{(}\StringTok{\textquotesingle{}topleft\textquotesingle{}}\NormalTok{, }\FunctionTok{c}\NormalTok{(}\StringTok{"Low Usage: 1 {-} 10 days"}\NormalTok{,}\StringTok{"Moderate Usage: 11 {-} 20 days"}\NormalTok{,}\StringTok{"High Usage: 21 {-} 31 days"}\NormalTok{), }\AttributeTok{cex =} \FloatTok{0.5}\NormalTok{, }\AttributeTok{fill =}\NormalTok{ colours)}
\end{Highlighting}
\end{Shaded}

\includegraphics{Bellabeat-Case-Study-R-Project_files/figure-latex/unnamed-chunk-60-1.pdf}

\hypertarget{device-worn-in-a-day-percentage}{%
\subparagraph{Device Worn in a Day
Percentage}\label{device-worn-in-a-day-percentage}}

The pie chart below illustrates the percentage with which the device was
worn all day.

\begin{Shaded}
\begin{Highlighting}[]
\NormalTok{daily\_activity\_merged}\SpecialCharTok{$}\NormalTok{device\_worn }\OtherTok{\textless{}{-}}\NormalTok{ daily\_activity\_merged}\SpecialCharTok{$}\NormalTok{sedentary\_minutes }\SpecialCharTok{+}\NormalTok{ daily\_activity\_merged}\SpecialCharTok{$}\NormalTok{fairly\_active\_minutes }\SpecialCharTok{+}\NormalTok{ daily\_activity\_merged}\SpecialCharTok{$}\NormalTok{lightly\_active\_minutes }\SpecialCharTok{+}\NormalTok{ daily\_activity\_merged}\SpecialCharTok{$}\NormalTok{very\_active\_minutes}
\NormalTok{daily\_activity\_merged}\SpecialCharTok{$}\NormalTok{device\_worn }\OtherTok{\textless{}{-}}\NormalTok{ daily\_activity\_merged}\SpecialCharTok{$}\NormalTok{device\_worn}\SpecialCharTok{/}\DecValTok{60}
\NormalTok{daily\_activity\_merged}\SpecialCharTok{$}\NormalTok{device\_worn\_24 }\OtherTok{\textless{}{-}}\NormalTok{ daily\_activity\_merged}\SpecialCharTok{$}\NormalTok{device\_worn }\SpecialCharTok{==} \DecValTok{24}
\NormalTok{device\_worn }\OtherTok{\textless{}{-}} \FunctionTok{count}\NormalTok{(daily\_activity\_merged, device\_worn\_24) }\SpecialCharTok{\%\textgreater{}\%} 
  \FunctionTok{drop\_na}\NormalTok{()}

\FunctionTok{glimpse}\NormalTok{(device\_worn)}
\end{Highlighting}
\end{Shaded}

\begin{verbatim}
## Rows: 2
## Columns: 2
## $ device_worn_24 <lgl> FALSE, TRUE
## $ n              <int> 458, 378
\end{verbatim}

\begin{Shaded}
\begin{Highlighting}[]
\NormalTok{slices\_v3 }\OtherTok{\textless{}{-}} \FunctionTok{c}\NormalTok{(}\DecValTok{378}\NormalTok{, }\DecValTok{458}\NormalTok{)}
\NormalTok{labels\_v3 }\OtherTok{\textless{}{-}} \FunctionTok{c}\NormalTok{(}\StringTok{"Worn all day"}\NormalTok{, }\StringTok{"Not worn all day"}\NormalTok{)}
\NormalTok{pct\_v1 }\OtherTok{\textless{}{-}}\NormalTok{ slices\_v3}\SpecialCharTok{/}\FunctionTok{sum}\NormalTok{(slices\_v3)}\SpecialCharTok{*}\DecValTok{100}
\NormalTok{pct\_v1 }\OtherTok{\textless{}{-}} \FunctionTok{round}\NormalTok{(pct\_v1, }\DecValTok{2}\NormalTok{)}
\NormalTok{labels\_v3 }\OtherTok{\textless{}{-}} \FunctionTok{paste}\NormalTok{(labels\_v3, pct\_v1)}
\NormalTok{labels\_v3 }\OtherTok{\textless{}{-}} \FunctionTok{paste}\NormalTok{(labels\_v3, }\StringTok{"\%"}\NormalTok{, }\AttributeTok{sep =} \StringTok{""}\NormalTok{)}
\NormalTok{colours }\OtherTok{\textless{}{-}} \FunctionTok{c}\NormalTok{(}\StringTok{"\#996855"}\NormalTok{, }\StringTok{"\#cebdbf"}\NormalTok{)}
\FunctionTok{options}\NormalTok{(}\AttributeTok{repr.plot.width =} \DecValTok{20}\NormalTok{, }\AttributeTok{repr.plot.height =} \DecValTok{13}\NormalTok{)}
\FunctionTok{pie}\NormalTok{(slices\_v3, }\AttributeTok{labels =}\NormalTok{ labels\_v3,}
    \AttributeTok{main=}\StringTok{"Percentage of the Device Being Worn the Whole Day"}\NormalTok{, }\AttributeTok{cex =} \DecValTok{1}\NormalTok{, }\AttributeTok{cex.main =} \FloatTok{1.4}\NormalTok{, }\AttributeTok{col =}\NormalTok{ colours, }\AttributeTok{border =} \StringTok{"white"}\NormalTok{, }\AttributeTok{radius =} \DecValTok{1}\NormalTok{)}
\end{Highlighting}
\end{Shaded}

\includegraphics{Bellabeat-Case-Study-R-Project_files/figure-latex/unnamed-chunk-62-1.pdf}

Key Findings

\begin{itemize}
\tightlist
\item
  54.78\% of participants did not wear the device for the entire day.
\end{itemize}

\hypertarget{the-number-of-hours-the-device-was-worn-throughout-the-week}{%
\subparagraph{The Number of Hours the Device was Worn Throughout the
Week}\label{the-number-of-hours-the-device-was-worn-throughout-the-week}}

Given that over 50\% of participants did not wear the device for the
entire 24 hours per day, it is pertinent to examine the number of hours
the device was worn per day.

\begin{Shaded}
\begin{Highlighting}[]
\CommentTok{\# How many Hours was the Device Worn per Day}
\NormalTok{device\_worn\_24\_7 }\OtherTok{\textless{}{-}}\NormalTok{ daily\_activity\_merged[}\FunctionTok{c}\NormalTok{(}\StringTok{"id"}\NormalTok{, }\StringTok{"weekday"}\NormalTok{, }\StringTok{"device\_worn"}\NormalTok{, }\StringTok{"activity\_date\_ymd"}\NormalTok{)]}
\NormalTok{device\_worn\_24\_7}\SpecialCharTok{$}\NormalTok{weekday }\OtherTok{\textless{}{-}} \FunctionTok{factor}\NormalTok{(device\_worn\_24\_7}\SpecialCharTok{$}\NormalTok{weekday, }\AttributeTok{levels =} \FunctionTok{c}\NormalTok{(}\StringTok{"Monday"}\NormalTok{, }\StringTok{"Tuesday"}\NormalTok{, }\StringTok{"Wednesday"}\NormalTok{, }\StringTok{"Thursday"}\NormalTok{, }\StringTok{"Friday"}\NormalTok{, }\StringTok{"Saturday"}\NormalTok{, }\StringTok{"Sunday"}\NormalTok{))}

\NormalTok{device\_worn\_24\_7 }\SpecialCharTok{\%\textgreater{}\%} 
  \FunctionTok{drop\_na}\NormalTok{() }\SpecialCharTok{\%\textgreater{}\%} 
  \FunctionTok{ggplot}\NormalTok{(}\FunctionTok{aes}\NormalTok{(}\AttributeTok{y =}\NormalTok{ device\_worn, }\AttributeTok{x =}\NormalTok{ weekday, }\AttributeTok{group =} \DecValTok{1}\NormalTok{)) }\SpecialCharTok{+} 
  \FunctionTok{geom\_point}\NormalTok{(}\AttributeTok{color =} \StringTok{"\#cebdbf"}\NormalTok{, }\AttributeTok{size =} \DecValTok{1}\NormalTok{) }\SpecialCharTok{+}
  \FunctionTok{geom\_smooth}\NormalTok{(}\AttributeTok{method =} \StringTok{"loess"}\NormalTok{, }\AttributeTok{span =} \FloatTok{0.2}\NormalTok{, }\AttributeTok{color =} \StringTok{"\#996855"}\NormalTok{, }\AttributeTok{fill =} \StringTok{"\#e7a589"}\NormalTok{) }\SpecialCharTok{+}
  \FunctionTok{coord\_cartesian}\NormalTok{(}\AttributeTok{ylim =} \FunctionTok{c}\NormalTok{(}\DecValTok{0}\NormalTok{, }\DecValTok{24}\NormalTok{)) }\SpecialCharTok{+}
  \FunctionTok{ggtitle}\NormalTok{(}\StringTok{"The Number of Hours the Device was Worn }\SpecialCharTok{\textbackslash{}n}\StringTok{throughout the Week"}\NormalTok{)}\SpecialCharTok{+}
  \FunctionTok{ylab}\NormalTok{(}\StringTok{"The Number of Hours Worn within a Day"}\NormalTok{) }\SpecialCharTok{+}
  \FunctionTok{xlab}\NormalTok{(}\StringTok{"Day of Week"}\NormalTok{) }\SpecialCharTok{+}
  \FunctionTok{theme}\NormalTok{(}\AttributeTok{plot.title =} \FunctionTok{element\_text}\NormalTok{(}\AttributeTok{size =} \DecValTok{20}\NormalTok{, }\AttributeTok{hjust =} \FloatTok{0.5}\NormalTok{, }\AttributeTok{face =} \StringTok{\textquotesingle{}bold\textquotesingle{}}\NormalTok{),}
        \AttributeTok{axis.text.x =} \FunctionTok{element\_text}\NormalTok{(}\AttributeTok{angle =} \DecValTok{90}\NormalTok{, }\AttributeTok{size =} \DecValTok{14}\NormalTok{), }
        \AttributeTok{axis.title.x =} \FunctionTok{element\_text}\NormalTok{(}\AttributeTok{size =} \DecValTok{16}\NormalTok{),}
        \AttributeTok{axis.title.y =} \FunctionTok{element\_text}\NormalTok{(}\AttributeTok{size =} \DecValTok{16}\NormalTok{)}
\NormalTok{        )}
\end{Highlighting}
\end{Shaded}

\begin{verbatim}
## `geom_smooth()` using formula = 'y ~ x'
\end{verbatim}

\begin{verbatim}
## Warning in simpleLoess(y, x, w, span, degree = degree, parametric = parametric,
## : pseudoinverse used at 0.97
\end{verbatim}

\begin{verbatim}
## Warning in simpleLoess(y, x, w, span, degree = degree, parametric = parametric,
## : neighborhood radius 1.03
\end{verbatim}

\begin{verbatim}
## Warning in simpleLoess(y, x, w, span, degree = degree, parametric = parametric,
## : reciprocal condition number 1.0465e-30
\end{verbatim}

\begin{verbatim}
## Warning in simpleLoess(y, x, w, span, degree = degree, parametric = parametric,
## : There are other near singularities as well. 1.0609
\end{verbatim}

\begin{verbatim}
## Warning in predLoess(object$y, object$x, newx = if (is.null(newdata)) object$x
## else if (is.data.frame(newdata))
## as.matrix(model.frame(delete.response(terms(object)), : pseudoinverse used at
## 0.97
\end{verbatim}

\begin{verbatim}
## Warning in predLoess(object$y, object$x, newx = if (is.null(newdata)) object$x
## else if (is.data.frame(newdata))
## as.matrix(model.frame(delete.response(terms(object)), : neighborhood radius
## 1.03
\end{verbatim}

\begin{verbatim}
## Warning in predLoess(object$y, object$x, newx = if (is.null(newdata)) object$x
## else if (is.data.frame(newdata))
## as.matrix(model.frame(delete.response(terms(object)), : reciprocal condition
## number 1.0465e-30
\end{verbatim}

\begin{verbatim}
## Warning in predLoess(object$y, object$x, newx = if (is.null(newdata)) object$x
## else if (is.data.frame(newdata))
## as.matrix(model.frame(delete.response(terms(object)), : There are other near
## singularities as well. 1.0609
\end{verbatim}

\includegraphics{Bellabeat-Case-Study-R-Project_files/figure-latex/unnamed-chunk-63-1.pdf}

\begin{Shaded}
\begin{Highlighting}[]
\FunctionTok{options}\NormalTok{(}\AttributeTok{repr.plot.width =} \DecValTok{25}\NormalTok{, }\AttributeTok{repr.plot.height =} \DecValTok{12}\NormalTok{)}
\end{Highlighting}
\end{Shaded}

It appears that many participants wore the tracker for less than 20
hours per day. To determine the exact percentage of how many times the
device was worn for less than 20 hours per day, the count() function
will be used and a pie chart will be created.

\begin{Shaded}
\begin{Highlighting}[]
\CommentTok{\# Count of the device being worn \textless{} 20 hours}
\NormalTok{worn\_less\_20 }\OtherTok{\textless{}{-}}\NormalTok{ daily\_activity\_merged }\SpecialCharTok{\%\textgreater{}\%} 
  \FunctionTok{count}\NormalTok{(device\_worn }\SpecialCharTok{\textless{}} \DecValTok{20}\NormalTok{) }\SpecialCharTok{\%\textgreater{}\%} 
  \FunctionTok{drop\_na}\NormalTok{()}

\FunctionTok{head}\NormalTok{(worn\_less\_20)}
\end{Highlighting}
\end{Shaded}

\begin{verbatim}
##   device_worn < 20   n
## 1            FALSE 433
## 2             TRUE 403
\end{verbatim}

\begin{Shaded}
\begin{Highlighting}[]
\CommentTok{\# Percentage of the device being worn \textless{} 20 hours}
\NormalTok{slices\_v3 }\OtherTok{\textless{}{-}} \FunctionTok{c}\NormalTok{(}\DecValTok{406}\NormalTok{, }\DecValTok{433}\NormalTok{)}
\NormalTok{labels\_v3 }\OtherTok{\textless{}{-}} \FunctionTok{c}\NormalTok{(}\StringTok{"Device worn \textless{} 20 hours"}\NormalTok{, }\StringTok{"Device worn \textgreater{} 20 hours"}\NormalTok{)}
\NormalTok{pct\_v1 }\OtherTok{\textless{}{-}}\NormalTok{ slices\_v3}\SpecialCharTok{/}\FunctionTok{sum}\NormalTok{(slices\_v3)}\SpecialCharTok{*}\DecValTok{100}
\NormalTok{pct\_v1 }\OtherTok{\textless{}{-}} \FunctionTok{round}\NormalTok{(pct\_v1, }\DecValTok{2}\NormalTok{)}
\NormalTok{labels\_v3 }\OtherTok{\textless{}{-}} \FunctionTok{paste}\NormalTok{(labels\_v3, pct\_v1)}
\NormalTok{labels\_v3 }\OtherTok{\textless{}{-}} \FunctionTok{paste}\NormalTok{(labels\_v3, }\StringTok{"\%"}\NormalTok{, }\AttributeTok{sep =} \StringTok{""}\NormalTok{)}
\NormalTok{colours }\OtherTok{\textless{}{-}} \FunctionTok{c}\NormalTok{(}\StringTok{"\#996855"}\NormalTok{, }\StringTok{"\#cebdbf"}\NormalTok{)}
\FunctionTok{options}\NormalTok{(}\AttributeTok{repr.plot.width =} \DecValTok{20}\NormalTok{, }\AttributeTok{repr.plot.height =} \DecValTok{13}\NormalTok{)}
\FunctionTok{pie}\NormalTok{(slices\_v3, }\AttributeTok{labels =}\NormalTok{ labels\_v3,}
    \AttributeTok{main =} \StringTok{"Percentage of the Device Being Worn }\SpecialCharTok{\textbackslash{}n}\StringTok{Less than 20 Hours per Day"}\NormalTok{, }\AttributeTok{col =}\NormalTok{ colours, }\AttributeTok{border =} \StringTok{"white"}\NormalTok{, }\AttributeTok{radius =} \DecValTok{1}\NormalTok{,}
   \AttributeTok{cex.main =} \DecValTok{1}\NormalTok{, }\AttributeTok{cex =} \DecValTok{1}\NormalTok{)}
\end{Highlighting}
\end{Shaded}

\includegraphics{Bellabeat-Case-Study-R-Project_files/figure-latex/unnamed-chunk-65-1.pdf}

\#\#\#\#\#Key Findings - A significant portion of participants did not
wear the device for the entire day during the study. - 48.39\% of
participants wore the device for less than 20 hours per day.

\hypertarget{activity-device-usage-vs-sleep-device-usage}{%
\subparagraph{Activity Device Usage vs Sleep Device
Usage}\label{activity-device-usage-vs-sleep-device-usage}}

Given that almost half of the participants wore the device for less than
20 hours per day, it is plausible that these unaccounted hours
correspond to sleep periods when users may remove the device for
comfort. To investigate this hypothesis, two bar graphs will be plotted
comparing activity usage levels (users tracking their daily activity)
and sleep usage levels (users tracking their sleep).

\begin{Shaded}
\begin{Highlighting}[]
\CommentTok{\# Activity usage per participant}
\NormalTok{activity\_usage\_v2 }\OtherTok{\textless{}{-}}\NormalTok{ clean\_daily\_calories }\SpecialCharTok{\%\textgreater{}\%} 
  \FunctionTok{group\_by}\NormalTok{(id) }\SpecialCharTok{\%\textgreater{}\%} 
  \FunctionTok{count}\NormalTok{() }\SpecialCharTok{\%\textgreater{}\%} 
  \FunctionTok{arrange}\NormalTok{(n) }\SpecialCharTok{\%\textgreater{}\%}
  \FunctionTok{mutate}\NormalTok{(}\AttributeTok{activity\_percentage =}\NormalTok{ (n}\SpecialCharTok{/}\DecValTok{31}\NormalTok{)}\SpecialCharTok{*}\DecValTok{100}\NormalTok{)}
\FunctionTok{head}\NormalTok{(activity\_usage\_v2)}
\end{Highlighting}
\end{Shaded}

\begin{verbatim}
## # A tibble: 6 x 3
## # Groups:   id [6]
##           id     n activity_percentage
##        <dbl> <int>               <dbl>
## 1 4057192912     4                12.9
## 2 2347167796    18                58.1
## 3 8253242879    19                61.3
## 4 3372868164    20                64.5
## 5 6775888955    26                83.9
## 6 7007744171    26                83.9
\end{verbatim}

\begin{Shaded}
\begin{Highlighting}[]
\NormalTok{p5 }\OtherTok{\textless{}{-}} \FunctionTok{ggplot}\NormalTok{(activity\_usage\_v2, }\FunctionTok{aes}\NormalTok{(}\AttributeTok{x =} \FunctionTok{reorder}\NormalTok{(id, n), }\AttributeTok{y =}\NormalTok{ activity\_percentage)) }\SpecialCharTok{+}
  \FunctionTok{geom\_col}\NormalTok{(}\AttributeTok{fill =} \StringTok{"\#cebdbf"}\NormalTok{)  }\SpecialCharTok{+} 
  \FunctionTok{labs}\NormalTok{(}\AttributeTok{title=}\StringTok{"Activity Device Usage"}\NormalTok{, }\AttributeTok{x =} \StringTok{"Particpant \#"}\NormalTok{, }\AttributeTok{y =} \StringTok{"Percentage of Logged Activity"}\NormalTok{, }\AttributeTok{caption =} \StringTok{"33 participants"}\NormalTok{) }\SpecialCharTok{+}
  \FunctionTok{theme}\NormalTok{(}\AttributeTok{plot.title =} \FunctionTok{element\_text}\NormalTok{(}\AttributeTok{hjust =} \FloatTok{0.5}\NormalTok{, }\AttributeTok{vjust =} \FloatTok{0.8}\NormalTok{, }\AttributeTok{size =} \DecValTok{10}\NormalTok{, }\AttributeTok{face =} \StringTok{\textquotesingle{}bold\textquotesingle{}}\NormalTok{)) }\SpecialCharTok{+}
  \FunctionTok{theme}\NormalTok{(}\AttributeTok{axis.text.x =} \FunctionTok{element\_text}\NormalTok{(}\AttributeTok{angle =} \DecValTok{90}\NormalTok{, }\AttributeTok{size =} \DecValTok{8}\NormalTok{)) }\SpecialCharTok{+}
  \FunctionTok{theme}\NormalTok{(}\AttributeTok{axis.title.x =} \FunctionTok{element\_text}\NormalTok{(}\AttributeTok{size =} \DecValTok{5}\NormalTok{)) }\SpecialCharTok{+}
  \FunctionTok{theme}\NormalTok{(}\AttributeTok{axis.title.y =} \FunctionTok{element\_text}\NormalTok{(}\AttributeTok{size =} \DecValTok{8}\NormalTok{)) }\SpecialCharTok{+} 
  \FunctionTok{theme}\NormalTok{(}\AttributeTok{plot.caption =} \FunctionTok{element\_text}\NormalTok{(}\AttributeTok{size =} \DecValTok{6}\NormalTok{))}

\CommentTok{\# Sleep usage per participant}
\NormalTok{sleep\_usage }\OtherTok{\textless{}{-}}\NormalTok{ clean\_daily\_sleep }\SpecialCharTok{\%\textgreater{}\%} 
  \FunctionTok{group\_by}\NormalTok{(id) }\SpecialCharTok{\%\textgreater{}\%} 
  \FunctionTok{count}\NormalTok{() }\SpecialCharTok{\%\textgreater{}\%} 
  \FunctionTok{arrange}\NormalTok{(n) }\SpecialCharTok{\%\textgreater{}\%}
  \FunctionTok{mutate}\NormalTok{(}\AttributeTok{sleep\_percentage =}\NormalTok{ (n}\SpecialCharTok{/}\DecValTok{31}\NormalTok{)}\SpecialCharTok{*}\DecValTok{100}\NormalTok{) }\CommentTok{\#\% of nights users monitored sleep}
\FunctionTok{head}\NormalTok{(sleep\_usage)}
\end{Highlighting}
\end{Shaded}

\begin{verbatim}
## # A tibble: 6 x 3
## # Groups:   id [6]
##           id     n sleep_percentage
##        <dbl> <int>            <dbl>
## 1 2320127002     1             3.23
## 2 7007744171     2             6.45
## 3 1844505072     3             9.68
## 4 6775888955     3             9.68
## 5 8053475328     3             9.68
## 6 1644430081     4            12.9
\end{verbatim}

\begin{Shaded}
\begin{Highlighting}[]
\NormalTok{p6 }\OtherTok{\textless{}{-}} \FunctionTok{ggplot}\NormalTok{(sleep\_usage, }\FunctionTok{aes}\NormalTok{(}\AttributeTok{x=} \FunctionTok{reorder}\NormalTok{(id, n), }\AttributeTok{y =}\NormalTok{ sleep\_percentage)) }\SpecialCharTok{+}
  \FunctionTok{geom\_col}\NormalTok{(}\AttributeTok{fill =} \StringTok{"\#cebdbf"}\NormalTok{) }\SpecialCharTok{+} 
  \FunctionTok{labs}\NormalTok{(}\AttributeTok{title=}\StringTok{"Sleep Device Usage"}\NormalTok{, }\AttributeTok{x=} \StringTok{"Particpant \#"}\NormalTok{, }\AttributeTok{y=}\StringTok{"Percentage of Logged Sleep"}\NormalTok{, }\AttributeTok{caption =} \StringTok{"24 participants"}\NormalTok{) }\SpecialCharTok{+}
  \FunctionTok{theme}\NormalTok{(}\AttributeTok{plot.title =} \FunctionTok{element\_text}\NormalTok{(}\AttributeTok{hjust =} \FloatTok{0.5}\NormalTok{, }\AttributeTok{vjust =} \FloatTok{0.8}\NormalTok{, }\AttributeTok{size =} \DecValTok{10}\NormalTok{, }\AttributeTok{face =} \StringTok{\textquotesingle{}bold\textquotesingle{}}\NormalTok{)) }\SpecialCharTok{+}
  \FunctionTok{theme}\NormalTok{(}\AttributeTok{axis.text.x =} \FunctionTok{element\_text}\NormalTok{(}\AttributeTok{angle =} \DecValTok{90}\NormalTok{, }\AttributeTok{size =} \DecValTok{8}\NormalTok{)) }\SpecialCharTok{+}
  \FunctionTok{theme}\NormalTok{(}\AttributeTok{axis.title.x =} \FunctionTok{element\_text}\NormalTok{(}\AttributeTok{size =} \DecValTok{5}\NormalTok{)) }\SpecialCharTok{+}
  \FunctionTok{theme}\NormalTok{(}\AttributeTok{axis.title.y =} \FunctionTok{element\_text}\NormalTok{(}\AttributeTok{size =} \DecValTok{8}\NormalTok{)) }\SpecialCharTok{+}
  \FunctionTok{theme}\NormalTok{(}\AttributeTok{plot.caption =} \FunctionTok{element\_text}\NormalTok{(}\AttributeTok{size =} \DecValTok{6}\NormalTok{))}

\FunctionTok{options}\NormalTok{(}\AttributeTok{repr.plot.width =} \DecValTok{20}\NormalTok{, }\AttributeTok{repr.plot.height =} \DecValTok{12}\NormalTok{)}
\FunctionTok{grid.arrange}\NormalTok{(p5, p6,}
             \AttributeTok{ncol =} \DecValTok{1}
\NormalTok{)}
\end{Highlighting}
\end{Shaded}

\includegraphics{Bellabeat-Case-Study-R-Project_files/figure-latex/unnamed-chunk-66-1.pdf}

\hypertarget{key-findings-6}{%
\subparagraph{Key Findings}\label{key-findings-6}}

\begin{itemize}
\tightlist
\item
  33 participants tracked their daily activity, whereas 24 participants
  tracked their sleep. This indicates that all participants tracked
  their daily activity, but 9 of them did not track their sleep.
\item
  Over 21 participants wore the device throughout the whole day to track
  their daily activity, but only 3 wore it to bed consistently during
  the study. This suggests that many users remove their Fitbit tracker
  before sleeping.
\end{itemize}

\hypertarget{section-5-act}{%
\subsubsection{Section 5: Act}\label{section-5-act}}

\hypertarget{bellabeat-marketing-strategy-recommendations}{%
\paragraph{Bellabeat Marketing Strategy
Recommendations}\label{bellabeat-marketing-strategy-recommendations}}

\hypertarget{personalised-workout-schedule-for-female-working-professionals}{%
\subparagraph{personalised Workout Schedule for Female Working
Professionals}\label{personalised-workout-schedule-for-female-working-professionals}}

Many Bellabeat customers are working professionals who might often be
too busy to engage in regular exercise. Studies propose that people with
fewer opportunities for daily workouts during their work week can either
workout for longer periods during the weekends and/or work for shorter
periods during the week to maintain the recommended 75 - 150 minutes of
moderate to vigorous activity a week. Bellabeat can provide personalised
workout schedules that would not only reach the recommended amount of
activity but also accommodate the user's busy schedule.

\hypertarget{gentle-reminders-and-inactivity-alerts}{%
\subparagraph{Gentle Reminders and Inactivity
Alerts}\label{gentle-reminders-and-inactivity-alerts}}

To reduce the amount of sedentary time, Bellabeat added the inactivity
alert feature which sends out alerts to the users when they have been
inactive for too long, to remind them to get up and move around. This
feature, however, is only available to Leaf and Time but not the Ivy
tracker. Since Ivy focuses on wellness, vibrations and reminders might
be too distracting which can induce anxiety. Instead, Ivy allows users
to set their own daily goals for steps and activity to motivate them to
make better choices for their health. However, knowing that the target
market are female professionals who might be involved in daily prolonged
sitting, gentle inactivity alerts should still be an option as high
sedentary behaviour has negative health effects.

\hypertarget{comfortable-and-lightweight}{%
\subparagraph{Comfortable and
Lightweight}\label{comfortable-and-lightweight}}

Despite the importance of sleep tracking, some Fitbit users took off
their tracking device during the night as they found it uncomfortable to
sleep with. The marketing team should emphasise that Bellabeat Ivy is
incredibly lightweight making it very comfortable to wear anywhere and
anytime including to bed.

\hypertarget{strong-battery-life}{%
\subparagraph{Strong Battery Life}\label{strong-battery-life}}

Bellabeat Ivy also boasts a long battery life that can last up to 8
days. Therefore, users would have to worry less about recharging the
device overnight.

\#\#Conclusion The purpose of this case study was to identify potential
growth opportunities for Bellabeat, a wellness company, and improve
their marketing strategy by analysing smart device usage among Fitbit
consumers. Based on the analysis conducted, it can be concluded that the
majority of fitness smart device users are working professionals who
lack the recommended active minutes as they are often too busy to engage
in regular exercise. Therefore, it is recommended to tailor Bellabeat's
marketing strategies and products to working professionals by providing
personalised workout schedules as well as implementing gentle inactivity
alerts. Moreover, it was suggested to emphasise the importance of sleep
because of its numerous benefits. These are the recommendations and
conclusions drawn from the analysis of the Fitbit data. However, further
analysis into other smart device companies such as the Apple Watch or
Garmin is advised to support these results and provide additional
insight into Bellabeat's strategies.

\end{document}
